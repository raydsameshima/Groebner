\documentclass[11pt]{book}
\usepackage{amsmath,amssymb}
\usepackage{amsfonts}
\usepackage{graphicx}
\usepackage[all]{xy}
\usepackage{stmaryrd}
\usepackage[yyyymmdd]{datetime}
%\usepackage{listings}

\usepackage{hyperref}

\usepackage{tikz}

% To show all the labels. Remove this when finish.
\usepackage{showkeys}

\DeclareMathOperator*{\colim}{colim}
\DeclareMathOperator*{\coker}{coker}

\begin{document}

\newcommand{\Slash}[1]{{\ooalign{\hfil/\crcr$#1$}}}

\title{ A note on Gr\"obner bases}
\author{Ray D. Sameshima}
\date{2015/02/10 $\sim$ \today \, \currenttime}
\maketitle

\tableofcontents

%%%%%%body
\setcounter{chapter}{-2}
\chapter{Preface}

\section{Reference}
\begin{enumerate}
\item Ideals, Varieties, and Algorithms\marginpar{Our source book.}\\
An Introduction to Computational Algebraic Geometry and Commutative Algebra

Authors: David Cox, John Little, Donal O'Shea\\
ISBN: 978-0-387-35650-1 (Print) 978-0-387-35651-8 (Online)

\item nLab

\verb|https://ncatlab.org/|

\item Learn You a Haskell for Great Good!

\verb|http://learnyouahaskell.com/chapters|

\item ASCENDING CHAIN CONDITION
DENNIS S. KEELER

\verb|http://www.users.miamioh.edu/keelerds/705/chain.pdf|

\item A basic linear algebra(Masayoshi Nagata, et al) (written in Japanese)

\item Maxima, a Computer Algebra System

\verb|http://maxima.sourceforge.net|

\end{enumerate}

\chapter{Basics}
We assume living (working) knowledge on mathematics.

\section{Set theoretical gadgets}
Our set theory is ZFC.

%\setcounter{subsection}{-1}

\subsection{Binary relations}
A binary relation $\rho$ on a set $S$ is a function\footnote{This is Haskell type annotation. Haskell is pure, lazy, functional programming language. www.haskell.org}
\begin{eqnarray}
\verb|rho :: S -> S -> Bool|
\end{eqnarray}
i.e., $\forall a,b \in S$, we can determine whether $a \rho b \left( = \rho(a,b) \right)$ is \verb|True| or \verb|False|.

A set theoretical implementation is a subset in the product
\begin{eqnarray}
\rho \subset A \times A,
\end{eqnarray}
and
\begin{eqnarray}
a \rho b :\Leftrightarrow (a,b) \in \rho.
\end{eqnarray}

\subsection{Partially ordered sets}
Let $\leq$ be a binary relation on a set $S$.
A structured set $(S, \leq)$ is a partially ordered set iff
\begin{eqnarray}
\forall a \in S, a \leq a & \text{ (reflexivity)} \\
\forall a,b,c \in S, (a \leq b, b \leq c \Rightarrow a \leq c) & \text{ (transitivity)} \\
\forall a,b \in S, (a \leq b \leq a \Rightarrow a = b) & \text{ (antisymmetry)}
\end{eqnarray}

\subsection{Totally ordered sets}
The partial order $(S, \leq)$ is called total (linear) order iff
\begin{eqnarray}
\forall a,b \in S, \text{either } a \leq b \text{ or } b \leq a
\end{eqnarray}
holds.
That is, all two elements are comparable in a totally ordered set.

\subsection{Well-ordered sets}
A partially ordered set $(S, \leq)$ is well-ordered iff an arbitrary subset $T \subset S$ has a minimum element.
That is,
\begin{eqnarray}
\forall T \subset S, \exists t_0 \in T \text{ s.t. } \forall t \in T, t_0 \leq t.
\end{eqnarray}

\subsubsection{Well-ordered sets are totally ordered}
A well-ordered set $(S, \leq)$ is indeed totally ordered, since an arbitrary pair
\begin{eqnarray}
\{a, b\} \subset S
\end{eqnarray}
has the minimum, that is, either $a \leq b$ or $b \leq a$.

\subsection{Rings}
A ring $(R,+,*)$ is a structured set with two binary operations
\begin{eqnarray}
\verb|(+) :: R -> R -> R|\\
\verb|(*) :: R -> R -> R|
\end{eqnarray}
satisfying the following 3 (ring) axioms:
\begin{enumerate}
\item $(R,+)$ is an abelian, i.e., commutative group,
i.e.,
\begin{eqnarray}
\forall a,b,c \in R, (a+b) + c = a + (b + c) & \text{ (associativity for $+$)} \\
\forall a, b, \in R, a+b = b+a & \text{ (commutativity)} \\
\exists 0 \in R, \text{ s.t. } \forall a \in R, a + 0 = a & \text{ (additive identity) } \\
\forall a \in R, \exists (-a) \in R \text{ s.t. } a + (-a) = 0 & \text{ (additive inverse)}
\end{eqnarray}

\item $(R,*)$ is a monoid, i.e.,
\begin{eqnarray}
\forall a,b,c \in R, (a*b) * c = a * (b * c) & \text{ (associativity for $*$)} \\
\exists 1 \in R, \text{ s.t. } \forall a \in R, a * 1 = a = 1*a & \text{ (multiplicative identity) } 
\end{eqnarray}

\item Multiplication is distributive w.r.t addition, i.e., $\forall a,b,c \in R$, 
\begin{eqnarray}
a*(b+c) = (a*b) + (a*c) & \text{ (left distributivity)} \\
(a+b)*c = (a*c) + (b*c) & \text{ (right distributivity)} 
\end{eqnarray}

\end{enumerate}

\subsection{Fields}
A field is a ring $(\mathbb{K},+,*)$ whose non-zero elements form an abelian group under multiplication, i.e.,
\begin{eqnarray}
\forall r \in \mathbb{K}, r \neq 0 \Rightarrow \exists r^{-1} \in \mathbb{K} \text{ s.t. } r*r^{-1} = 1 = r^{-1}*r.
\end{eqnarray}
A field $\mathbb{K}$ is a finite field iff the underlying set $\mathbb{K}$ is finite, e.g., any quotient ring of prime $\mathbb{Z}_p$.
A field $\mathbb{K}$ is called infinite field iff the underlying set is infinite, say $\mathbb{Q,R,C}$.

\subsection{Equivalence relations}
An equivalence relation $\sim$ on a set $S$ is a binary relation which is reflexive, symmetric, and transitive:
\begin{eqnarray}
\forall a \in S, a \sim a & \text{ (reflexivity)} \\
\forall a,b \in S, (a \sim b \Rightarrow b \sim a) & \text{ (symmetry)} \\
\forall a,b,c \in S, (a \sim b, b \sim c \Rightarrow a \sim c) & \text{ (transitivity)} 
\end{eqnarray}
Then the subset
\begin{eqnarray}
[a] := \left\{\left. x \in S \right| x \sim a \right\}
\end{eqnarray}
is called the equivalence class of $a$.

\subsubsection{Partitions of disjoint subsets}
An equivalence relation $\sim$ on $S$ partitions $S$ into disjoint subsets of equivalence classes.

$\forall a \in S$ is in the equivalence class of itself, since $\sim$ is reflexive,
\begin{eqnarray}
a \sim a \Rightarrow a \in [a].
\end{eqnarray}
They are disjoint; if there exists some elements that is shared by two equivalent classes,
\begin{eqnarray}
x \in [a] \text{ and } x \in [b] \Rightarrow x \sim a \text{ and } x \sim b.
\end{eqnarray}
Since $\sim$ is transitive, we have
\begin{eqnarray}
a \sim b \Rightarrow [a] = [b].
\end{eqnarray}
$\blacksquare$

\subsubsection{Note}
This partition is a function from $S$ to the (non empty) subsets of $S$:
\begin{eqnarray}
[] : S \to (2^S - \emptyset); a \mapsto [a].
\end{eqnarray}

%%%%%%%%%%%%%%%%%%%%%%%%%%%%%%%%%%%%%%%%%%%%%%%%%

\section{Numbers; recipes without arithmetics}
Here we review $\mathbb{N, Z, Q, R},$ and $\mathbb{C}$.
If the readers are already familiar with (ZFC axiomatic) set theory and the set theoretic implementation of $\mathbb{N, Z, Q, R},$ and $\mathbb{C}$, you can skip this section.

\subsection{Natural numbers $\mathbb{N}$}
Here is a recursive implementation of natural numbers (Peano):
\begin{eqnarray}
0 &:=& \emptyset \\
n+1 &:=& n \cup \{n\}
\end{eqnarray}

\subsection{Integers $\mathbb{Z}$}
The set of integers is complete under subtraction:
\begin{eqnarray*}
\mathbb{Z} := (\mathbb{N} \times \mathbb{N})/ S,
\end{eqnarray*}
where the equivalence relation $S$ is given by
\begin{eqnarray}
(m_1, m_2)S(n_1,n_2) :\Leftrightarrow m_1 + n_2 = m_2 + n_1.
\end{eqnarray}
I.e., a negative integer $(-m) \in \mathbb{Z}$ of some $m \in \mathbb{N}$ is represented by, for example,
\begin{eqnarray}
(0,m).
\end{eqnarray}

\subsection{Rational numbers $\mathbb{Q}$}
The set of rational numbers is complete under non-zero division:
\begin{eqnarray}
\mathbb{Q} := \left(\mathbb{Z} \times (\mathbb{Z} - \{0\}) \right)/F,
\end{eqnarray}
where the equivalence relation $F$ is given by
\begin{eqnarray}
(a,b) F (c, d) : \Leftrightarrow a * d = b * c.
\end{eqnarray}
Usually we denote
\begin{eqnarray}
\frac{a}{b} := (a,b) \in \mathbb{Q}.
\end{eqnarray}

\subsection{Real numbers $\mathbb{R}$}
The set of real numbers should satisfy the Axiom of continuity, but before state it, let us define an important tool, the cut.

\subsubsection{Dedekind cut of $\mathbb{Q}$}
A Dedekind cut is given by a pair
\begin{eqnarray}
(A_-, A_+)
\end{eqnarray}
of non-empty subsets of $\mathbb{Q}$:
\begin{eqnarray}
A_\mp \in 2^\mathbb{Q}-\{\emptyset\}
\end{eqnarray}
with
\begin{eqnarray}
A_- \cup A_+ = \mathbb{Q}
\end{eqnarray}
and
\begin{eqnarray}
x \in A_-, y \in A_+ \Rightarrow x < y.
\end{eqnarray}

\subsubsection{Definition of $\mathbb{R}$}
There are three possibilities
\begin{eqnarray}
\label{notExistsExists}
\nexists \max A_-, \exists \min A_+ \\
\label{existsNotExists}
\exists \max A_-, \nexists \min A_+ \\
\label{notExistsNotExists}
\nexists \max A_-,  \nexists \min A_+ 
\end{eqnarray}
since they do not share the border element.
Here is a preliminary definition of $\mathbb{R}$:\footnote{
Alternatively, we can define $\mathbb{R}$ be the set of all Dedekind cuts identifying eq.(\ref{notExistsExists}) and eq.(\ref{existsNotExists}).
}
\begin{eqnarray}
\label{preReal}
\mathbb{R} := \left\{\left. (A_-, A_+) \right| \text{Dedekind cut with} (\ref{existsNotExists}), (\ref{notExistsNotExists})\right\}
\end{eqnarray}

We define a total order $\leq$ as follows.
Let 
\begin{eqnarray}
\alpha := (A_-, A_+), \beta := (B_-, B_+) \in \mathbb{R}
\end{eqnarray}
then we define
\begin{eqnarray}
\alpha \leq \beta :\Leftrightarrow A_- \subset B_-.
\end{eqnarray}

\subsubsection{The Axiom of continuity}
A cut of $\mathbb{R}$
\begin{eqnarray}
(A,B)
\end{eqnarray}
with
\begin{eqnarray}
A \cup B = \mathbb{R}
\end{eqnarray}
and
\begin{eqnarray}
x \in A, y \in B \Rightarrow x < y
\end{eqnarray}
satisfies either
\begin{eqnarray}
\label{minA}
\exists \min A, \nexists \max B
\end{eqnarray}
or
\begin{eqnarray}
\label{maxB}
\nexists \min A, \exists \max B.
\end{eqnarray}

\subsubsection{Check}
We prove that our $\mathbb{R}$ in eq.(\ref{preReal}) satisfies the Axiom of continuity.
Let 
\begin{eqnarray}
\{\alpha_\lambda\}_{\lambda \in \Lambda}
\end{eqnarray}
be bounded subspaces of $\mathbb{R}$, and
\begin{eqnarray}
\alpha_\lambda := (A^\lambda_-, A^\lambda_+).
\end{eqnarray}
Now
\begin{eqnarray}
\alpha &:=& (A_-, A_+) \\
A_- &:=& \bigcup_{\lambda \in \Lambda} A^\lambda_- \\
A_+ &:=& \mathbb{Q} - A_-
\end{eqnarray}
is an upper bound of $\{\alpha_\lambda\}_{\lambda \in \Lambda}$, since $\forall \lambda \in \Lambda$,
\begin{eqnarray}
A^\lambda_- \subset A_- \Leftrightarrow \alpha_\lambda \leq \alpha
\end{eqnarray}
by definition.

Indeed this $\alpha$ is the minimum of upper bound of $\{\alpha_\lambda\}_{\lambda \in \Lambda}$, i.e., the supremum: if 
\begin{eqnarray}
\beta := (B^\lambda_-, B^\lambda_+)
\end{eqnarray}
is another upper bound of $\{\alpha_\lambda\}_{\lambda \in \Lambda}$, then 
\begin{eqnarray}
\forall \lambda \in \Lambda, A^\lambda_- \subset B.
\end{eqnarray}
This means
\begin{eqnarray}
A_- \subset B_- \Leftrightarrow \alpha \leq \beta.
\end{eqnarray}
Therefore, if $(A,B)$ is a cut of $\mathbb{R}$, then $A \subset \mathbb{R}$ has the supremum $\alpha \in \mathbb{R}$:
\begin{eqnarray}
\alpha := \sup A \in \mathbb{R}.
\end{eqnarray}

Finally we show that either $\alpha = \min A$ (eq.(\ref{minA})) or $\alpha = \max B$ (eq.(\ref{maxB})).
If $\alpha \neq \max A$, then
\begin{eqnarray}
\alpha \in B.
\end{eqnarray}
By definition, $\forall \beta \in B$ is an upper bound of $\alpha$:
\begin{eqnarray}
\alpha \leq \beta,
\end{eqnarray}
that is
\begin{eqnarray}
\alpha = \min B.
\end{eqnarray}
$\blacksquare$

\subsection{Complex numbers $\mathbb{C}$}
Let us define
\begin{eqnarray}
\mathbb{C} := (\mathbb{R}\times \mathbb{R}, +, *),
\end{eqnarray}
where
\begin{eqnarray}
(a,b) + (c,d) &:=& (a+c, b+d) \\
(a,b) * (c,d) &:=& (a*c - b* d, a*d + b*c).
\end{eqnarray}
Note that
\begin{eqnarray}
(0,1) * (0,1) := (-1,0).
\end{eqnarray}
Usually we denote
\begin{eqnarray}
a + \sqrt{-1} * b := (a,b) \in \mathbb{C}.
\end{eqnarray}

%%%%%%%%%%%%%%%%%%%%%%%%%%%%%%%%%%%%%%%%%%%%%%%%%

\section{The fundamental theorem in algebra (Theorem 7, \S1.1)}
Here we review the fact that $\mathbb{C}$ of complex numbers is algebraically closed.

\subsection{($\epsilon$-$N$) convergence}
Consider $\mathbb{C}$ of complex numbers and a function
\begin{eqnarray}
z: \mathbb{N} \to \mathbb{C}; z\mapsto z_n
\end{eqnarray}
so called a sequence on $\mathbb{C}$.
We usually write a sequence as
\begin{eqnarray}
\{ z_n \}_{n \in \mathbb{N}}.
\end{eqnarray}

A sequence $z$ converges to $c \in \mathbb{C}$ iff
\begin{eqnarray}
\forall \epsilon > 0, \exists N \in \mathbb{N} \text{ s.t. } (\forall n \geq N, |c - z_n| < \epsilon),
\end{eqnarray}
where $\forall a ,b \in \mathbb{R}$,
\begin{eqnarray}
|a + \sqrt{-1} b| :=  \sqrt{a^2 + b^2}
\end{eqnarray}
is the euclidean distance.

\subsection{(Sequence) continuity}
Consider a $\mathbb{C}$-valued sequence
\begin{eqnarray}
z: \mathbb{N} \to \mathbb{C}; z\mapsto z_n
\end{eqnarray}
which converges to $c \in \mathbb{C}$.
We write this as
\begin{eqnarray}
\lim_{n \to \infty} z_n = c.
\end{eqnarray}

A function $f: \mathbb{C} \to \mathbb{C}$ is continuous at $c \in \mathbb{C}$ iff
\begin{eqnarray}
f(c) = f\left( \lim_{n \to \infty} z_n \right) = \lim_{n \to \infty} f(z_n).
\end{eqnarray}
A continuous function is a function which is continuous at every point of its domain, i.e., every input.

\subsection{Lemma}
\label{aLemmaForFundamentalTheoremInAlgebra}
Let
\begin{eqnarray}
C_1 := \left\{z \left| z \in \mathbb{C}, |z|^2 \leq 1 \right.\right\}
\end{eqnarray}
be a unit circle and its inside area.
An arbitrary function
\begin{eqnarray}
f: C_1 \to \mathbb{R}
\end{eqnarray}
has the maximum and minimum.

\subsubsection{Proof}
It suffices to show the maximum case, and we prove by contradiction.

Suppose there is no maximum; that is either
\begin{enumerate}

\item (no upper limit)
\label{noUpperLimit}
it diverges, i.e.,
\begin{eqnarray}
\forall N \in \mathbb{N}, \exists z \in C_1 \text{ s.t. } f(z) > N,
\end{eqnarray}

or

\item (upper limit)
\label{upperLimit}
there is NO $z \in C_1$ s.t. $f(z) = \alpha$ but
\begin{eqnarray}
\forall n \in \mathbb{N}, f(z_n) < \alpha
\end{eqnarray}
and
\begin{eqnarray}
\lim_{n \to \infty} f(z_n) = \alpha
\end{eqnarray}

\end{enumerate}
holds.\footnote{
We sometimes write it as
\begin{eqnarray}
f(z_n) \to \alpha - 0.
\end{eqnarray}
}

Let us construct a sequence $p: \mathbb{N} \to C_1$ by, if there is no upper limit, put $p_n$ s.t.,
\begin{eqnarray}
f(p_n) > n,
\end{eqnarray}
else if $\alpha$ is the upper limit, put $p_n$ s.t.,
\begin{eqnarray}
\alpha - \frac{1}{n} < f(p_n) < \alpha.
\end{eqnarray}
Then we get a sequence\footnote{Here we have used the Axiom of choice.}
\begin{eqnarray}
p: \mathbb{N} \to C_1.
\end{eqnarray}

Next, consider four squares of $\frac{1}{2} \times \frac{1}{2}$, which are separated by two lines
\begin{eqnarray}
y = \frac{n}{2}, x = \frac{n}{2}.
\end{eqnarray}
Since $C_1$ is covered by a finite number(=16) of these squares, and at least one of which has infinite number of $p$'s, call it $S_1$.
At least one of the quadrants of $S_1$ contains infinite number of $p$'s, call it $S_2$.
By induction, we get a sequence of squares
\begin{eqnarray}
\left. S_n \right|_{n \in \mathbb{N}}
\end{eqnarray}
s.t., $\forall n \in \mathbb{N}$, $S_n$ is a $\frac{1}{2^n} \times \frac{1}{2^n}$ square and each $S_n$ contains infinite $p$'s.

Let us pick a sub sequence $q : \mathbb{N} \to C_1$ of $p$ by
\begin{enumerate}
\item choose $q_1 \in S_1$ from arbitrary $p$'s in $S_1$, say for some $n$,
\begin{eqnarray}
q_1 := p_n.
\end{eqnarray}

\item let $i \geq 1$ and from
\begin{eqnarray}
q_1, \cdots, q_{i-1} = p_{m},
\end{eqnarray}
of some $m \in \mathbb{N}$, pick $q_i$ from
\begin{eqnarray}
\{p_j  \in S_i | j > m  \}
\end{eqnarray}

\end{enumerate}

This sequence $q : \mathbb{N} \to C_1$ satisfies for $j>i$,
\begin{eqnarray}
q_j, q_i \in S_i
\end{eqnarray}
and its euclidean distance is smaller than 
\begin{eqnarray}
\sqrt{2} \frac{1}{2^i}
\end{eqnarray}
of diagonal length of the $i$-th square.
So, $q$ converges to a point in $C_1$
\begin{eqnarray}
\lim_{n\to \infty} q_n =: q_\infty \in C_1,
\end{eqnarray}
and since $f$ is continuous,
\begin{eqnarray}
f(q_\infty) = \lim_{n\to \infty} f(q_n).
\end{eqnarray}

If \ref{noUpperLimit}st (divergent) case holds, the right hand side is infinity, contradiction.
Else \ref{upperLimit}nd (upper limit) case holds, the right hand side is $\alpha$, contradiction.

Therefore, all function $f : C_1 \to \mathbb{R}$ has the maximum, and minimum.\\
$\blacksquare$

\subsection{The fundamental theorem in algebra}
Let $a_n \neq 0, \left. a_i\right|_i$ be $\mathbb{C}$.
An arbitrary $\mathbb{C}$-coefficients degree $n>0$ polynomial
\begin{eqnarray}
\label{nDegreePolynomial}
a_n * x^n + a_{n-1} * x^{n-1} + \cdots + a_0 = 0%, a_n \neq 0, \left. a_i\right|_i \in \mathbb{C}
\end{eqnarray}
has a solution in $\mathbb{C}$.
This means that $\mathbb{C}$ is algebraically closed.

\subsubsection{Proof}
We prove this theorem by contradiction; suppose there is no solution.
Define
\begin{eqnarray}
F: \mathbb{C} \to \mathbb{R}
\end{eqnarray}
by
\begin{eqnarray}
F(x) := \left| a_n x^n + a_{n-1} x^{n-1} + \cdots + a_0 \right|.
\end{eqnarray}
We can choose a large $R'>0$ and $|x|>R'$ s.t.\footnote{
Since this left hand side is decreasing if $|x|$ becomes larger and larger.
}
\begin{eqnarray}
\left|\frac{a_{n-1}}{x} + \cdots + \frac{a_0}{x^n} \right| < \frac{|a_n|}{2}.
\end{eqnarray}
Then $\forall |x| > R'$,
\begin{eqnarray}
F(x) &=& |x|^n * \left|a_n + \frac{a_{n-1}}{x} + \cdots + \frac{a_0}{x^n} \right| \\
&\geq& |x|^n *\left( \left|a_n \right| - \left| \frac{a_{n-1}}{x} + \cdots + \frac{a_0}{x^n} \right| \right) \\
&>& |x|^n \frac{|a_n|}{2} 
\end{eqnarray}
and\footnote{
We have used a triangle inequality
\begin{eqnarray}
|a+b| \leq |a| + |b|
\end{eqnarray}
and its consequence
\begin{eqnarray}
|a+b| - |b| \leq |b| \Leftrightarrow |c-b| \geq |c| -|b|.
\end{eqnarray}
}
we also can find $R \geq R'$ s.t.
\begin{eqnarray}
\forall |x| > R, F(x) > F(0),
\end{eqnarray}
since $F(0) = a_0$ is a constant.

So if we define 
\begin{eqnarray}
C_R := \{x \in \mathbb{C} | |x| \leq R\},
\end{eqnarray}
then
\begin{eqnarray}
x \notin C_R \Rightarrow F(x) > F(0).
\end{eqnarray}
Therefore, if $F(x)$ has a minimum in $C_R$, this is indeed the minimum in $\mathbb{C}$.

Above lemma \S\ref{aLemmaForFundamentalTheoremInAlgebra} guarantees that $F(x)$ has a minimum in $C_R$, and $F(x)$ has the minimum in $\mathbb{C}$:
\begin{eqnarray}
\alpha := \min \left. F(x) \right|_x.
\end{eqnarray}
Since this is not a solution for eq.(\ref{nDegreePolynomial}), $\alpha$ is positive definite.
Now suppose $F$ has this minimum $\alpha>0$ at $p$:
\begin{eqnarray}
F(p) = \alpha > 0.
\end{eqnarray}

Consider
\begin{eqnarray}
g(x) := a_n (x+p)^n + a_{n-1} (x+p)^{n-1} + \cdots + a_0.
\end{eqnarray}
Then $|g(x)|$ has its minimum $g(0) = \alpha$.
Define $h(x)$ by 
\begin{eqnarray}
h(x)&:=& g(x)/\alpha\\ 
%&=& g(x) / (a_n p^n + a_{n-1} p^{n-1} + \cdots + a_0) \\
&=& 1 + b_1 x + \cdots b_n x^n.
\end{eqnarray}
Consider $b_1, \cdots, b_n$, and let
\begin{eqnarray}
b_s
\end{eqnarray}
be the first nonzero coefficient.
Define
\begin{eqnarray}
K := \max \{ |b_{s+1}|, \cdots, |b_n| \}.
\end{eqnarray}

Now, we can find a small $r>0$ which satisfies both
\begin{eqnarray}
1 - |b_s| r^s > 0
\end{eqnarray}
and
\begin{eqnarray}
0 < |b_s| r^s - \frac{K r^{s+1}}{1-r} < 1
\end{eqnarray}
for later arguments.

If we write in a polar coordinate
\begin{eqnarray}
b_s = |b_s| * e^{i \theta}
\end{eqnarray}
and define
\begin{eqnarray}
q := r * e^{i (\pi - \theta)/s}
\end{eqnarray}
then $b_s q^s = |b_s| e^{i\theta} * r^s e^{i(\pi-\theta)} = |b_s| r^s e^{i\pi} = - |b_s| r^s$ and
\begin{eqnarray}
|h(q)| &=& |1 + b_s q^s + \cdots + b_n q^n| \\
&\leq& 1 + |b_s q^s| + |b_{s+1} q^{s+1} + \cdots + b_n q^n| \\
&<& 1 - |b_s| r^s + |b_{s+1}| r^{s+1} + \cdots + |b_n| r^n \\
&\leq& 1 - |b_s| r^s + K\left( r^{s+1} + \cdots + r^n \right) \\
&=& 1 - |b_s| r^s + K r^{s+1} \frac{1 - r^{n-s-1}}{1-r} \\
&<& 1 - |b_s| r^s + \frac{K r^{s+1}}{1-r} \\
&&< 1.
\end{eqnarray}

Since 
\begin{eqnarray}
|h(q)| = \frac{F(q+p)}{F(q)} < 1,
\end{eqnarray}
we get
\begin{eqnarray}
F(q+p) < F(p) = \alpha,
\end{eqnarray}
but it contradicts that $\alpha$ is the minimum of $F$.\\
$\blacksquare$

\subsection{Corollary}
We can reduce
\begin{eqnarray}
a_n * x^n + a_{n-1} * x^{n-1} + \cdots + a_0
=
a_n * (x-s_1) * \cdots * (x-s_n).
\end{eqnarray}

\subsubsection{Proof}
Induction on the degree $n$.
Base case; $n=1$:
\begin{eqnarray}
a_1*x +a_0 = a_1*\left(x - (-\frac{a_0}{a_1})\right).
\end{eqnarray}
Induction step; let us assume $n-1$ case and consider
\begin{eqnarray}
f(x) := a_n * x^n + a_{n-1} * x^{n-1} + \cdots + a_0, a_n \neq 0.
\end{eqnarray}
Let $s$ be a solution for $f(x) =0$;
\begin{eqnarray}
f(s) &=& a_n * x^n + a_{n-1} * x^{n-1} + \cdots + a_0 \\
&=& 0.
\end{eqnarray}
Then
\begin{eqnarray}
f(x) &=& f(x) - f(s)\\
&=& a_n * x^n + a_{n-1} * x^{n-1} + \cdots + a_0 - 0 \\
&=& a_n * (x^n-s^n) + a_{n-1} * (x^{n-1}-x^{n-1}) + \cdots + a_1*(x-s) + 0 \qquad\qquad \\
&=& a_n * (x-s) * h(x)
\end{eqnarray}
where $h(x)$ has degree at most $(n-1)$, and its highest coefficient is one (1).\\
$\blacksquare$

%%%%%%%%%%%%%%%%%%%%%%%%%%%%%%%%%%%%%%%%%%%%%%%%%

\chapter{Geometry, Algebra, and algorithms}
Let $\mathbb{K}$ be an arbitrary fields, and $\forall x_1, \cdots, x_n \in \mathbb{K}$.
We will treat polynomial ring in $n$ variables $(x_1, \cdots, x_n)$ with  $\mathbb{K}$ coefficients:\footnote{ We will define this set in eq.(\ref{polynomialRing}).}
\begin{eqnarray}
\mathbb{K}[x_1,\cdots,x_n].
\end{eqnarray}
We will introduce affine varieties, which are solution sets defined by polynomial equations.

\section{Polynomials and Affine space}
\subsection{Monomials}
A monomial in $x_1, \cdots, x_n$ is a product of the form
\begin{eqnarray}
x_1^{\alpha_1} * \cdots * x_n^{\alpha_n},
\end{eqnarray}
where
\begin{eqnarray}
\alpha_1, \cdots, \alpha_n \in \mathbb{N}.
\end{eqnarray}
The total degree of this monomial is the sum $\alpha_1+ \cdots + \alpha_n$.

When every $\alpha$s zero, then we write
\begin{eqnarray}
x_1^{0} * \cdots * x_n^{0} = 1.
\end{eqnarray}

\subsection{Multi index notation}
We write a monomial using multi index notation
\begin{eqnarray}
x^\alpha := x_1^{\alpha_1} * \cdots * x_n^{\alpha_n},
\end{eqnarray}
and the total degree
\begin{eqnarray}
|\alpha| := \alpha_1+ \cdots + \alpha_n.
\end{eqnarray}

\subsection{Polynomials}
A polynomial $f(x) = f(x_1, \cdots, x_n)$ is a finite linear combination of monomials:
\begin{eqnarray}
f(x) = \sum_\alpha c_\alpha * x^\alpha,
\end{eqnarray}
where $c_\alpha \in \mathbb{K}$ is called coefficient of monomial $x^\alpha$.

For nonzero coefficient $c_\alpha \neq 0$,
\begin{eqnarray}
c_\alpha * x^\alpha
\end{eqnarray}
is called a term (of $f(x)$).

The total degree of $f(x)$ is given by the maximum of $|\alpha|$ (of nonzero coefficients).

We write
\begin{eqnarray}
\label{polynomialRing}
\mathbb{K}[x_1,\cdots,x_n] := \left\{\left. f(x) = \sum_\alpha c_\alpha * x^\alpha \right| \forall c_\alpha \in \mathbb{K} \right\}
\end{eqnarray}
as a set of all polynomial in $n$ variables $(x_1, \cdots, x_n)$ with  $\mathbb{K}$ coefficients.

\subsection{Affine spaces}
Let $\mathbb{K}$ be a field and $n \in \mathbb{N}$.
The $n$ dim affine space over $\mathbb{K}$ is the set
\begin{eqnarray}
\mathbb{K}^n := \left\{ \left. (a_1, \cdots, a_n) \right| a_1, \cdots, a_n \in \mathbb{K} \right\}.
\end{eqnarray}

\subsection{Polynomials as functions}
A polynomial
\begin{eqnarray}
f(x) = \sum_\alpha c_\alpha * x^\alpha
\end{eqnarray}
can be seen as a function
\begin{eqnarray}
f : \mathbb{K}^n \to \mathbb{K}
\end{eqnarray}
which is defined by
\begin{eqnarray}
a:=(a_1, \cdots, a_n) \stackrel{f}{\mapsto} f(a) := \sum_\alpha c_\alpha * a^\alpha.
\end{eqnarray}
That is, the function $f$ replace every $x_i$ by $a_i$ in the expression for $f(x)$.

\section{Affine Varieties}
Affine varieties are curves, surfaces, and their higher dimensional generalizations defined by polynomial equations.

\subsection{Definition of affine varieties}
\label{DefOfAV}
Let $f_1(x), \cdots, f_s(x) \in \mathbb{K}[x_1,\cdots,x_n]$.
Then
\begin{eqnarray}
\mathbb{V}(f_1, \cdots, f_s) := \left\{\left. a := (a_1, \cdots, a_n) \in \mathbb{K}^n \right| f_i(a) = 0, 1 \leq \forall i \leq s \right\}.
\end{eqnarray}
is the affine variety defined by $f_1, \cdots, f_s$.

Thus, an affine variety $\mathbb{V}(f_1, \cdots, f_s)$ is the set of all solutions of the system of equations (for $x$):
\begin{eqnarray}
f_1(x) =0, \cdots, f_s(x) =0.
\end{eqnarray}

\subsection{Intersection and union of varieties (Lemma 2 \S1.2)}
\label{intersectionAndUnionOfVarieties}
Let $V, W \subset \mathbb{K}^n$ be 
\begin{eqnarray}
V &:=& \mathbb{V}(f_1, \cdots, f_s)\\
W &:=& \mathbb{V}(g_1, \cdots, g_t)
\end{eqnarray}
of varieties.
Then $V \cap W$ and $V \cup W$ are also affine varieties, and indeed
\begin{eqnarray}
V \cap W &=& \mathbb{V}(f_1, \cdots, f_s, g_1, \cdots, g_t) \\
V \cup W &=& \mathbb{V}\left( \left. f_i * g_j \right| 1 \leq i \leq s, 1 \leq j \leq t \right) 
\end{eqnarray}

\subsubsection{Proof}
\begin{enumerate}

\item $(V \cap W)$
The 1st equation holds since
\begin{eqnarray}
a \in V \cap W &\Leftrightarrow& f(a) = 0 \text{ and } g(a) = 0 \\
&\Leftrightarrow& a \in \mathbb{V}(f_1, \cdots, g_t).
\end{eqnarray}

\item $(V \cup W)$
Let us start $\subset$-direction.
If $a \in V$, i.e.,
\begin{eqnarray}
f_1(a) = \cdots = f_s(a) = 0.
\end{eqnarray}
Then $\forall i,j$,
\begin{eqnarray}
f_i * g_j = 0.
\end{eqnarray}
That is 
\begin{eqnarray}
a \in \mathbb{V}\left( \left. f_i * g_j \right| 1 \leq i \leq s, 1 \leq j \leq t \right),
\end{eqnarray}
and
\begin{eqnarray}
V \subset \mathbb{V}\left( \left. f_i * g_j \right| 1 \leq i \leq s, 1 \leq j \leq t \right).
\end{eqnarray}
Similarly, $W \subset \mathbb{V}\left( \left. f_i * g_j \right| 1 \leq i \leq s, 1 \leq j \leq t \right)$ and these imply that
\begin{eqnarray}
V \cup W \subset \mathbb{V}\left( \left. f_i * g_j \right| 1 \leq i \leq s, 1 \leq j \leq t \right).
\end{eqnarray}

For $\supset$-direction.
Take
\begin{eqnarray}
\label{aInProductOfVarieties}
a \in \mathbb{V}\left( \left. f_i * g_j \right| 1 \leq i \leq s, 1 \leq j \leq t \right).
\end{eqnarray}
If $a \in V$ then
\begin{eqnarray}
a \in V \subset V \cup W
\end{eqnarray}
and done.
If $a \not\in V$, then there exists some $i_0$ s.t.,
\begin{eqnarray}
f_{i_0}(a) \neq 0.
\end{eqnarray}
However, from eq.(\ref{aInProductOfVarieties}), 
\begin{eqnarray}
\forall i,j, f_i(a) * g_j(a) = 0
\end{eqnarray}
and we get
\begin{eqnarray}
\forall j, g_j(a) = 0.
\end{eqnarray}
This means $a \in W$.
Therefore
\begin{eqnarray}
a \in W \subset V \cup W.
\end{eqnarray}

\end{enumerate}
$\blacksquare$

\section{Parameterizations of Affine Varieties}
\subsection{Rational functions}
Let $f(x)$ be an arbitrary, and $g(x)$ be non-zero polynomials.
\begin{eqnarray}
f(x),g(x) (\neq 0) \in \mathbb{K}[x_1,\cdots,x_n].
\end{eqnarray}
A rational function in $(x_1,\cdots,x_n)$ with $\mathbb{K}$ coefficients is a quotient
\begin{eqnarray}
f/g : \mathbb{K}^n \to \mathbb{K}; a \mapsto \frac{f(a)}{g(a)}.
\end{eqnarray}

The equality is given
\begin{eqnarray}
f/g = h/k &:\Leftrightarrow& k*f = g*h \\
&\Leftrightarrow& k(x)*f(x) = g(x)*h(x), \forall x \in \mathbb{K}.
\end{eqnarray}
in $\mathbb{K}[x_1, \cdots, x_n]$.
The set of all ration function in $x_1, \cdots, x_n$ is denoted
\begin{eqnarray}
\mathbb{K}(x_1, \cdots, x_n).
\end{eqnarray}

\section{Ideals}
\subsection{Ideals}
\label{DefOfIdeal}
A subset $I \subset \mathbb{K}[x_1,\cdots,x_n]$ is an ideal (or a polynomial ideal) iff
\begin{eqnarray}
0 &\in& I \\
\forall f(x),g(x) \in I, f(x) + g(x) &\in& I \\
\forall f(x) \in I, h(x) \in \mathbb{K}[x_1,\cdots,x_n], h(x)*f(x) &\in& I.
\end{eqnarray}

\subsection{Generators of an ideal}
A set of polynomials generate an ideal:
\begin{eqnarray}
\left< f_1, \cdots, f_s \right> := \left\{ \left. \sum_i h_i(x) * f_i(x) \right| h_i \in \mathbb{K}[x_1,\cdots,x_n] \right\}
\end{eqnarray}
We call it an ideal generated by $f(x)$'s.

\subsubsection{Check}
$\left< f_1, \cdots, f_s \right> $ is indeed an ideal.
\begin{enumerate}
\item 0
\begin{eqnarray}
0 = \sum_i 0 * f_i(x) \in \left< f_1, \cdots, f_s \right>.
\end{eqnarray}

\item sum
\begin{eqnarray}
g_1(x) + g_2(x) &=& \sum_i g_{1i}(x) * f_i(x) + \sum_i g_{2i}(x) * f_i(x) \\
&=& \sum_i (g_{1i}(x) + g_{2i}(x)) * f_i(x) \\
&& \in \left< f_1, \cdots, f_s \right>.
\end{eqnarray}

\item 
\begin{eqnarray}
h(x) * \sum_i g_{i}(x) * f_i(x) &=& \sum_i \left( h(x) *g_{i}(x) \right) * f_i(x) \\
&& \in \left< f_1, \cdots, f_s \right>.
\end{eqnarray}

\end{enumerate}
$\blacksquare$

\subsection{(Proposition 4 \S1.4)}
\begin{eqnarray}
\left< f_1, \cdots, f_s \right> = \left< g_1, \cdots, g_t \right> \Rightarrow \mathbb{V}(f_1, \cdots, f_s) = \mathbb{V}(g_1, \cdots, g_t).
\end{eqnarray}

\subsubsection{Proof}
It suffices to show $\mathbb{V}(f) \subset \mathbb{V}(g)$.
By definition, for each $g_i(x)$, $\exists M_{ij}(x) \in \mathbb{K}[x_1,\cdots,x_n]$ s.t.
\begin{eqnarray}
g_i(x) = \sum_j M_{ij}(x) * f_j(x).
\end{eqnarray}
So, if $a \in \mathbb{V}(f)$, then
\begin{eqnarray}
g_i(a) = \sum_j M_{ij}(a) * f_j(a) = 0.
\end{eqnarray}
Therefore $a \in \mathbb{V}(g)$.\\
$\blacksquare$

\subsection{Definition}
For an arbitrary variety $V \subset \mathbb{K}^n$, define
\begin{eqnarray}
\mathbb{I}(V) := \left\{\left. f(x) \in \mathbb{K}[x_1,\cdots,x_n] \right| \forall a \in V, f(a) = 0 \right\}
\end{eqnarray}

\subsection{(Lemma 6 \S1.4)}
$\mathbb{I}(V)$ is an ideal.

\subsubsection{Proof}
\begin{enumerate}
\item 0
By definition, 0 of zero polynomial is in $\mathbb{I}(V)$

\item sum
If $f(x), g(x) \in \mathbb{I}(V)$, then the sum
\begin{eqnarray}
(f + g)(x) := f(x) + g(x)
\end{eqnarray}
satisfies
\begin{eqnarray}
\forall a \in V, (f + g)(a) = f(a) + g(a) = 0.
\end{eqnarray}

\item
For an arbitrary polynomial $h(x) \in \mathbb{K}[x_1,\cdots,x_n]$ and $f(x) \in \mathbb{I}(V)$,
\begin{eqnarray}
\forall a \in V, (h*f)(a) = h(a) * f(a) = 0
\end{eqnarray}

\end{enumerate}
$\blacksquare$

\subsection{Lemma 7 \S1.4}
\begin{eqnarray}
\left< f_1, \cdots, f_s \right> \subsetneq \mathbb{I}\left( \mathbb{V} \left( f_1, \cdots, f_s \right) \right)
\end{eqnarray}

\subsubsection{Proof}
$(\subset)$; $\forall f(x) \in \left< f_1, \cdots, f_s \right>, \exists h_i(x) \in \mathbb{K}[x_1,\cdots,x_n]$, s.t.,
\begin{eqnarray}
f(x) = \sum_{i=1}^s h_i(x) * f_i(x)
\end{eqnarray}
Therefore, $\forall a \in \mathbb{V} \left( f_1, \cdots, f_s \right)$,
\begin{eqnarray}
f(a) = \sum_{i=1}^s h_i(x) * f_i(a) = 0
\end{eqnarray}
and $f(x) \in \mathbb{I}\left( \mathbb{V} \left( f_1, \cdots, f_s \right) \right)$.

(Counter example for the equality)
Consider
\begin{eqnarray}
\left< x^2, y^2 \right>
\end{eqnarray}
and 
\begin{eqnarray}
\mathbb{I}\left( \mathbb{V} \left( x^2, y^2 \right) \right).
\end{eqnarray}
The equations
\begin{eqnarray}
x^2 = 0 = y^2
\end{eqnarray}
imply
\begin{eqnarray}
x = 0 = y
\end{eqnarray}
i.e.,
\begin{eqnarray}
\mathbb{V} \left( x^2, y^2 \right) = \{(0,0)\}
\end{eqnarray}
However we also have
\begin{eqnarray}
\mathbb{V} \left( x, y \right) = \{(0,0)\} \Rightarrow \mathbb{I}\left( \mathbb{V} \left( x^2, y^2 \right) \right) = \left< x, y \right>.
\end{eqnarray}
The fact
\begin{eqnarray}
x \neq \left< x^2, y^2 \right>
\end{eqnarray}
indicates that
\begin{eqnarray}
\left< x^2, y^2 \right> \neq \mathbb{I}\left( \mathbb{V} \left( x^2, y^2 \right) \right).
\end{eqnarray}
$\blacksquare$

\subsection{Proposition 8 \S1.4}
\begin{eqnarray}
V \subset W \Leftrightarrow \mathbb{I}(V) \supset \mathbb{I}(W).
\end{eqnarray}

\subsubsection{Proof}
$(\Rightarrow)$ Since any polynomial vanishing on $W$ must vanish on $V$,
\begin{eqnarray}
f(x) \in \mathbb{I}(V) \Rightarrow f(x) \in \mathbb{I}(W).
\end{eqnarray}

$(\Leftarrow)$
If
\begin{eqnarray}
f(x) \in \mathbb{I}(V) \Rightarrow f(x) \in \mathbb{I}(W),
\end{eqnarray}
i.e., 
\begin{eqnarray}
\forall a \in W, f(a)=0 \Rightarrow \forall b \in V, f(b) = 0,
\end{eqnarray}
then we have
\begin{eqnarray}
V \subset W.
\end{eqnarray}
$\blacksquare$

\subsection{Radical ideals (Exercise 8 \S 1.4)}
An ideal $I \subset \mathbb{K}[x_1,\cdots,x_n]$ is radical iff
\begin{eqnarray}
f(x) \in I \Leftrightarrow \exists m \text{ s.t. } \left( f(x) \right)^m \in I.
\end{eqnarray}
$\mathbb{I}(V)$ is a radical ideal.

\subsubsection{Proof}
Since
\begin{eqnarray}
f(a) = 0
\end{eqnarray}
iff
\begin{eqnarray}
\left( f(a)  \right)^m  = 0
\end{eqnarray}
for some $m$.\\
$\blacksquare$

%%%%%%%%%%%%%%%%%%%%%%%%%%%%%%%%%%%%%%%%%

\section{Polynomials of One(1) Variable}
Consider univariate polynomials
\begin{eqnarray}
\mathbb{K}[x].
\end{eqnarray}

\subsection{Leading terms}
Given nonzero polynomial $f(x) \in \mathbb{K}[x]$, let
\begin{eqnarray}
f(x) = a_m * x^m + \cdots + a_1 * x + a_0,
\end{eqnarray}
where $a_i \in \mathbb{K}$ and $a_m \neq 0$.
Then
\begin{eqnarray}
m = \deg(f)
\end{eqnarray}
and
\begin{eqnarray}
a_m * x^m
\end{eqnarray}
is the leading term of $f$:
\begin{eqnarray}
LT\left(f(x) \right) = a_m * x^m.
\end{eqnarray}

\subsection{A total order $\leq$ in one variable $ \mathbb{K}[x]$ }
If $f(x), g(x) \in \mathbb{K}[x]$ are nonzero polynomials, then
\begin{eqnarray}
\deg(f) \leq \deg(g) \Leftrightarrow LT\left(f(x) \right) \text{ divides } LT\left(g(x) \right).
\end{eqnarray}
This order is essentially the total order in $(\mathbb{N}, \leq)$, so we can define an totally ordered $(\mathbb{K}[x], \leq)$ by
\begin{eqnarray}
f(x) \leq g(x) :\Leftrightarrow \deg(f) \leq \deg(g).
\end{eqnarray}
We write
\begin{eqnarray}
f(x) > g(x)
\end{eqnarray}
for
\begin{eqnarray}
\lnot\left( f(x) \leq g(x) \right) &:\Leftrightarrow& \lnot\left( \deg(f) \leq \deg(g) \right) \\
&\Leftrightarrow&  \deg(f) \not\leq \deg(g) \\
&\Leftrightarrow&  \deg(f) > \deg(g).
\end{eqnarray}

\subsection{The Division Algorithm in 1 variable}
\label{DivisionAlgorithmIn1}
Let $g(x) \in \mathbb{K}[x]$ be a nonzero polynomial.
Then $\forall f(x) \in \mathbb{K}[x]$, $\exists q(x),r(x) \in \mathbb{K}[x]$ s.t.
\begin{eqnarray}
f(x) = q(x)*g(x) + r(x)
\end{eqnarray}
and either
\begin{eqnarray}
r = 0
\end{eqnarray}
or
\begin{eqnarray}
\label{degreeOfReminder}
\deg(r) < \deg(g).
\end{eqnarray}
There is an algorithm that can find unique $q(x)$ and $r(x)$.

\subsubsection{Pseudo code}
Here is our pseudo (Pascal like) code:
\begin{verbatim}
Input:  f,g
Output: q,r

q := 0
r := f
WHILE r /= 0 AND LT(g) divides LT(r) DO
  q := q + LT(r) / LT(g)
  r := r - (LT(r) / LT(g)) * g
\end{verbatim}

\subsubsection{Proof}
First, we shall prove that $r$ and $q$ are unique.
If we assume that there are two expressions
\begin{eqnarray}
\label{if2different}
q_1(x)*g(x) + r_1(x) = f(x) = q_2(x)*g(x) + r_2(x).
\end{eqnarray}
If $r_1 = 0$ and $r_2 \neq 0$, then
\begin{eqnarray}
r_2(x) = \left(q_1(x) - q_2(x) \right)*g(x).
\end{eqnarray}
If $q_1(x) \neq q_2(x)$ then we can divide $r_2(x)$ by $g(x)$; contradiction (see eq.(\ref{degreeOfReminder})),
So $r_1 =0$ requires $r_2=0$ and $q_1=q_2$.

Else both $r_1(x)$ and $r_2(x)$ are nonzero, without loss of generality, we can put $\deg(r_1) \leq \deg(r_2)$ and eq.(\ref{degreeOfReminder}) holds: $\deg(r_2)< \deg(g)$.
\begin{eqnarray}
\label{ourAssumptionOfDegreeIn1}
\deg(r_1) \leq \deg(r_2)< \deg(g).
\end{eqnarray}
From eq.(\ref{if2different}), we have
\begin{eqnarray}
r_2 - r_1 &=& (q_1 - q_2)*g.
\end{eqnarray}
Therefore, if $r_2 \neq r_1$, then
\begin{eqnarray}
\deg(r_2) &=& \deg(r_2 - r_1) \\
&=& \deg\left( (q_1 - q_2)*g \right) \\
&=& \deg(q_1 - q_2) + \deg(g) \\
&>& \deg(g)
\end{eqnarray}
but this contradict our assumption eq.(\ref{ourAssumptionOfDegreeIn1}).
So, both $r$ and $q$ are unique.

Next we shall prove this algorithm terminate; we shall show that above WHILE is not an infinite loop.
Since the process
\begin{eqnarray}
\verb| r -> r - (LT(r) / LT(g))* g |
\end{eqnarray}
is strict decreasing on their degree; suppose $m \geq k$ and\footnote{This is so called "big O notation".}
\begin{eqnarray}
r &=& a_0 * x^m + O(x^{m-1}) \\
g &=& b_0 * x^k + O(x^{k-1})
\end{eqnarray}
Then $r - (LT(r) / LT(g))* g$ is
\begin{eqnarray}
a_0 * x^m + O(x^{m-1}) - \frac{a_0 * x^m}{b_0 * x^k} \times \left( b_0 * x^k + O(x^{k-1}) \right)
\end{eqnarray}
and clearly the coefficient of $x^m$ is cancelled out, the result is at most $O(x^{m-1})$.
So, for finite degree of inputs, it exits WHILE loop finite steps.\\
$\blacksquare$

\subsubsection{$\text{rem}$ and $\text{quot}$}
For $f(x), g(x)(\neq 0) \in \mathbb{K}[x]$ and the consequence of division algorithm,
\begin{eqnarray}
f(x) = q(x) * g(x) + r(x),
\end{eqnarray}
let us define
\begin{eqnarray}
\text{rem}\left( f(x), g(x) \right) &:=& r(x) \\
\text{quot}\left( f(x), g(x) \right) &:=& q(x).
\end{eqnarray}

\subsubsection{Recursive definition}
Here is the recursive definition.
The input of this algorithm is two polynomials
\begin{eqnarray}
f(x), g(x)(\neq 0) \in \mathbb{K}[x]
\end{eqnarray}
and its outputs are
\begin{eqnarray}
q(x), r(x) \in \mathbb{K}[x]
\end{eqnarray}
s.t. $f(x) = q(x) * g(x) + r(x)$.

\begin{enumerate}
\item Base case.
\begin{eqnarray}
q_0(x) &:=& 0 \\
r_0(x) &:=& f(x)
\end{eqnarray}

\item Induction step.
If $r_n(x)$ is zero polynomial, or $LT\left(r_n(x) \right) \not\geq LT\left( g(x)\right)$,
then
\begin{eqnarray}
q(x) &:=& q_n(x) \\
r(x) &:=& r_n(x)
\end{eqnarray}
otherwise, i.e. $r_n(x) \neq 0$ and $LT\left(r_n(x) \right) \geq LT\left( g(x)\right)$,
\begin{eqnarray}
q_{n+1}(x) &:=& q_n(x) + \frac{LT\left( r_n(x) \right)}{LT\left( g(x) \right)} \\
r_{n+1}(x) &:=& r_n(x) - \frac{LT\left( r_n(x) \right)}{LT\left( g(x) \right)}*g(x) 
\end{eqnarray}

\end{enumerate}


\subsubsection{Haskell code}
todo: QuickCheck

Here is an actual code for one variable polynomial division.
At first, we have used Double for coefficients, but Haskell can treat "Rationals".

\begin{verbatim}
DARation.lhs

For one variable K[x].

> module DARation where

> import Data.Ratio 
> import Test.QuickCheck

> type Monomial = Int
> -- type ATerm = (Double, Monomial)
> type ATerm = (Ratio Int, Monomial)

A term is given by its coefficient and its non-negative power.

> type Poly = [ATerm]

We assume that there is no terms with same power like
  (3.0, 7) (-2.0, 7)
It's much better to implement as an instance of Num!

> polySort :: Poly -> Poly
> polySort [] = []
> polySort (f1@(c,a):polys) 
>   = if (c /= 0) then higher ++ (f1: lower)
>                 else higher ++ lower
>   where higher = polySort [(c',a') | (c',a') <- polys, c' /= 0, a' > a]
>         lower  = polySort [(c',a') | (c',a') <- polys, c' /= 0, a' < a]

> polyAdd :: Poly -> Poly -> Poly
> polyAdd f g = polyAdd' (polySort f) (polySort g) 
>   where 
>     polyAdd' :: Poly -> Poly -> Poly
>     polyAdd' [] [] = []
>     polyAdd' f [] = f
>     polyAdd' [] g = g
>     polyAdd' f@((c1,a1):fs) g@((c2,a2):gs) 
>       | a1 > a2 = (c1,a1) : (polyAdd' fs g)
>       | a1 < a2 = (c2,a2) : (polyAdd' f gs)
>       | a1 == a2 = (c1+c2,a1) : (polyAdd' fs gs)
>       | otherwise = error ":polyAdd"

> polyNegate :: Poly -> Poly
> polyNegate = map (\(c,a) -> (-c,a))

> polySub :: Poly -> Poly -> Poly
> polySub f g = polyAdd f (polyNegate g)

> polyMul :: Poly -> Poly -> Poly
> polyMul f g = polyMul' (polySort f) (polySort g)
>   where 
>     polyMul' :: Poly -> Poly -> Poly
>     polyMul' [] [] = []
>     polyMul' [] _ = []
>     polyMul' _ [] = []
>     polyMul' ((c1,a1):fs) g@((c2,a2):gs) 
>       = polyAdd first (polyMul' fs g)
>       where first = map (\(c,a) -> (c*c1, a+a1)) g

  *DivisionAlgorithm> polyMul [(2,3),(-4,1),(3,0)] [(1,1),(1,0)]
  [(2.0,4),(2.0,3),(-4.0,2),(-1.0,1),(3.0,0)]
  (%i9) f: 2*x^3-4*x+3;
  (%o9) 2*x^3-4*x+3
  (%i10) g: x+1;
  (%o10) x+1
  (%i13) g*f, expand;
  (%o13) 2*x^4+2*x^3-4*x^2-x+3

> isDivisibleBy :: ATerm -> ATerm -> Bool
> isDivisibleBy (_, a) (_, b) 
>   | a < 0 || b < 0 = error "ATerm is given by positive power"
>   | otherwise = (a >= b)

> leadingTermOf :: Poly -> ATerm
> leadingTermOf polynomial = head $ polySort polynomial 

  *DivisionAlgorithm> let f = [(2,3),(-4,1),(3,0)] :: Poly 
  *DivisionAlgorithm> leadingTermOf f
  (2.0,3)

> termDiv :: ATerm -> ATerm -> ATerm
> termDiv f@(c1,a1) g@(c2,a2)
>   | c2 == 0             
>        = error "0 division"
>   | f `isDivisibleBy` g 
>        = (c1/c2, a1-a2) -- since c1 :: Ratio Int, 1/3 = 1 % 3
>   | otherwise           
>        = error "Not divisible"

> -- polyDiv :: Poly -> Poly -> (Quot, Rem)
> polyDiv :: Poly -> Poly -> (Poly, Poly)
> polyDiv f g = polyDiv' (polySort f) (polySort g)
> polyDiv' :: Poly -> Poly -> (Poly, Poly)
> polyDiv' _ [] = error "zero division"
> polyDiv' [] g = ([],g)
> f `polyDiv'` g = div' g ([], f)
>   where
>     div' :: Poly -> (Poly, Poly) -> (Poly, Poly)
>     div' g (q, r)
>       | r /= [] && ltr `isDivisibleBy` ltg
>           = div' g (q `polyAdd` newR, r `polySub` (newR `polyMul` g)) 
>       | otherwise = (polySort q, polySort r)
>       where
>         ltr = leadingTermOf r
>         ltg = leadingTermOf g
>         newR = [ltr `termDiv` ltg] :: Poly

  (%i41) divide(x^3+2*x^2+x+1, 2*x+1,x);
  (%o41) [(4*x^2+6*x+1)/8,7/8]

  *DARation Data.Ratio> let f = [(1,3),(2,2),(1,1),(1,0)]
  *DARation Data.Ratio> let g = [(2,1),(1,0)]
  *DARation Data.Ratio> f `polyDiv` g
  ([(1 % 2,2),(3 % 4,1),(1 % 8,0)],[(7 % 8,0)])

  *DARation Data.Ratio> f `polyDiv` g
  ([(1 % 2,2),(3 % 4,1),(1 % 8,0)],[(7 % 8,0)])
  *DARation Data.Ratio> ((fst it) `polyMul` g) `polyAdd` (snd it)
  [(1 % 1,3),(2 % 1,2),(1 % 1,1),(1 % 1,0)]
  *DARation Data.Ratio> it == f
  True
\end{verbatim}

\subsection{"is divisible" $\sqsupseteq$}
Let us define
\begin{eqnarray}
f(x) \sqsupseteq g(x) :\Leftrightarrow \exists q(x), f(x) = q(x) * g(x),
\end{eqnarray}
that is, iff $f(x)$ is divisible by $g(x)$ (so $r=0$).

\subsection{Every ideal of $\mathbb{K}[x]$ is generated by one polynomial}
\label{InOneDimension}
Every ideal of $\mathbb{K}[x]$ over a field $\mathbb{K}$ can be written in the form $\left< f\right>$.

\subsubsection{Proof}
Take an ideal $I \subset \mathbb{K}[x]$.
If $I = \{ 0 \}$, then we have done since $I = \left< 0 \right>$.

Otherwise, we can take a polynomial $f(x) \in I$ which is minimum in degree in $I$.
We shall prove that $I = \left< f \right>$.
Since $I$ is an ideal, if $\forall f'(x) \in \left< f \right>$ then $f'(x) \in I$, i.e.,
\begin{eqnarray}
\left< f \right> \subset I.
\end{eqnarray}

Conversely, $\forall g(x) \in I$, by division algorithm in \S\ref{DivisionAlgorithmIn1}, we have
\begin{eqnarray}
g(x) = q(x)*f(x) + r(x)
\end{eqnarray}
where either $r=0$ or $\deg(r) < \deg(f)$.
Since $I$ is an ideal, $f(x),g(x) \in I$ and $q(x)*f(x) \in I$, we get
\begin{eqnarray}
r(x) = g(x) -q(x)*f(x) \in I.
\end{eqnarray}
If $r$ were not $0$, then from \S\ref{DivisionAlgorithmIn1}, $\deg(r) < \deg(f)$, which would contradict our minimum assumption.
So $r=0$ and this means
\begin{eqnarray}
g(x) \in \left< f \right>,
\end{eqnarray}
and this means
\begin{eqnarray}
\left< f \right> \supset I.
\end{eqnarray}

Therefore we get
\begin{eqnarray}
\left< f \right> = I.
\end{eqnarray}
$\blacksquare$

\subsubsection{Principal ideal domain (PID)}
In general, an ideal generated by one element is called a principal ideal.
We call the univariate polynomial ring $\mathbb{K}[x]$ a principal ideal domain.

\subsection{Corollary}
\label{atMostDeg}
Let $\mathbb{K}$ be a field and $f(x) \in \mathbb{K}[x]$ be a non zero polynomial.
If $m = \deg(f)$, then $f(x) = 0$ has at most $m$ roots in $\mathbb{K}$.

\subsubsection{Proof}
We will prove this statement by induction on $m$.
When $m=0$, then $f$ is just a non zero constant, and there is no root, so we have $0$ root.

Assume $m-1$ case holds, then consider $f$ of $m$ degree.
If $f$ has no root, then done.
Else $f(x)=0$ has a root $a \in \mathbb{K}$, then by \S\ref{DivisionAlgorithmIn1},
\begin{eqnarray}
f(x) = q(x)* (x-a) + r(x), \deg(r) < \deg(x-a)=1
\end{eqnarray}
that is $r \in \mathbb{K}$.
Since $f$ is zero at $a$,
\begin{eqnarray}
r = f(a) = 0
\end{eqnarray}
and
\begin{eqnarray}
f (x)= q(x)* (x-a).
\end{eqnarray}
Since $\deg(q)$ is $m-1$, and has at most $m-1$ root in $\mathbb{K}$.
Therefore, $f(x)=0$ has at most $m$ root.\\
$\blacksquare$

\subsection{Zero function on an infinite field (Proposition 5 \S1.1)}
Consider an infinite field $\mathbb{K}$ and $f(x) \in \mathbb{K}[x_1, \cdots, x_n]$.
Then $f(x) = 0$ in $\mathbb{K}[x_1, \cdots, x_n]$, i.e., zero polynomial iff $f : \mathbb{K}^n \to \mathbb{K}$ is the zero function.

\subsubsection{Proof}
$(\Rightarrow)$ part is obvious, since if $f(x)$ is the zero polynomial, i.e., all the coefficients are zero, then the zero polynomial $f(x)$ gives zero function:
\begin{eqnarray}
f : \mathbb{K}^n \to \mathbb{K}; \forall a \mapsto 0.
\end{eqnarray}

$(\Leftarrow)$ We will use induction on $n$ to show the statement that if for every $a \in \mathbb{K}^n, f(a) = 0$ then $f(x)$ is the zero polynomial.

\begin{enumerate}
\item Base case ($n=1$).
Since $f(x) \in \mathbb{K}[x]$ has at most $\deg(f)$ roots in $\mathbb{K}$ by \S\ref{atMostDeg}, if $\deg(f)$ is finite, we can choose some
\begin{eqnarray}
a \in \mathbb{K}
\end{eqnarray}
s.t.
\begin{eqnarray}
f(a) \neq 0
\end{eqnarray}
since $\mathbb{K}$ is infinite (set).
So, if $\forall a \in \mathbb{K}, f(a) = 0$ then $f(x)=0$ has infinitely many roots, and hence $f(x)$ is the zero polynomial.

\item Induction step.
Assume $n-1$ case is true, and let $f(x) \in \mathbb{K}[x_1, \cdots, x_n]$ be a polynomial s.t. $\forall a \in \mathbb{K}^n, f(a) = 0$.
By collecting the terms which are powers of $x_n$, we can write
\begin{eqnarray}
f(x) = \sum_{i} g_i (x_1, \cdots, x_{n-1}) * x_n^i.
\end{eqnarray}

Let us fix
\begin{eqnarray}
(a_1, \cdots, a_{n-1}) \in \mathbb{K}^{n-1}
\end{eqnarray}
and treat
\begin{eqnarray}
f(a_1, \cdots, a_{n-1}, x_n) \in \mathbb{K}[x_n] 
\end{eqnarray}
as a polynomial only of $x_n$.
Since $f(x)$ is zero for every $x_n$, $f(x)$ is the zero polynomial of $x_n$.
This means the "coefficients" are zero
\begin{eqnarray}
g_i (a_1, \cdots, a_{n-1}) = 0, \forall i.
\end{eqnarray}
However, our choice of $(a_1, \cdots, a_{n-1})$ is arbitrary and this means every $g_i(x)$ is zero polynomial in $\mathbb{K}[x_1, \cdots, x_{n-1}]$.
From the induction hypothesis, $g_i (a_1, \cdots, a_{n-1}) = 0$, and
\begin{eqnarray}
f(x) = 0
\end{eqnarray}
i.e., the zero polynomial.\\
\end{enumerate}

$\blacksquare$

\subsection{Corollary 6 \S1.1}
Let $\mathbb{K}$ be an infinite field, and $f(x), g(x) \in  \mathbb{K}[x_1, \cdots, x_n]$.
Then $f(x) = g(x)$. in $ \mathbb{K}[x_1, \cdots, x_n]$ iff $f,g : \mathbb{K}^n \to \mathbb{K}$ are the same functions. 

%\subsection{Properties of GCD}
%Let $f,g \in \mathbb{K}[x]$, then
%\begin{enumerate}
%\item $\exists GCD(f,g)$ up to nonzero multiplication.
%\item $\left< f, g\right> = \left< GCD(f,g) \right>$, i.e. $GCD(f,g)$ is a generator of the ideal $\left< f, g\right>$.
%\item There is an algorithm for finding $GCD(f,g)$.
%\end{enumerate}

\subsection{GCD}
A greatest common divisor of $f(x),g(x) \in \mathbb{K}[x]$ is a polynomial $h(x) \in \mathbb{K}[x]$ s.t.
\begin{enumerate}
\item 
\begin{eqnarray}
f(x) &\sqsupseteq& h(x) \\
g(x) &\sqsupseteq& h(x) 
\end{eqnarray}
i.e., "common."

\item If $h'(x) \in \mathbb{K}[x]$ satisfies
\begin{eqnarray}
f(x) &\sqsupseteq& h'(x) \\
g(x) &\sqsupseteq& h'(x) 
\end{eqnarray}
then
\begin{eqnarray}
h(x) \sqsupseteq h'(x) 
\end{eqnarray}
i.e., "greatest."

\end{enumerate}

\subsubsection{Proof}
We show the existence of such a gcd.
Let us consider an ideal
\begin{eqnarray}
\left< f(x), g(x) \right> \subset \mathbb{K}[x].
\end{eqnarray}
Since $\forall$ ideal in $\mathbb{K}[x]$ is PID, there exists a polynomial $h(x) \in \mathbb{K}[x]$ s.t.
\begin{eqnarray}
\left< f(x), g(x) \right> = \left< h(x)\right>.
\end{eqnarray}
So, there are $a(x),b(x) \in \mathbb{K}[x]$ s.t.
\begin{eqnarray}
f(x) &=& a(x) * h(x) \\
g(x) &=& b(x) * h(x) \\
\end{eqnarray}
since $f(x), g(x) \in \left< f(x), g(x) \right> = \left< h(x)\right>$.
This is equivalent to
\begin{eqnarray}
f(x) &\sqsupseteq& h(x) \\
g(x) &\sqsupseteq& h(x) 
\end{eqnarray}
We claim that this $h(x)$ is a greatest common devisor of $f(x)$ and $g(x)$.

If we take $h'(x) \in \mathbb{K}[x]$ s.t.
\begin{eqnarray}
f(x) &\sqsupseteq& h'(x) \\
g(x) &\sqsupseteq& h'(x) 
\end{eqnarray}
i.e., $\exists c(x),d(x) \in \mathbb{K}[x]$ s.t.,
\begin{eqnarray}
f(x) &=& c(x) * h'(x) \\
g(x) &=& d(x) * h'(x).
\end{eqnarray}
Then, since $\left< f(x), g(x) \right> = \left< h(x)\right>$, there exists $i(x),j(x) \in \mathbb{K}[x]$ s.t.
\begin{eqnarray}
h(x) &=& i(x) * f(x) + j(x) * g(x) \\
&=& \left( i(x)*c(x) + j(x)*d(x) \right) * h'(x).
\end{eqnarray}
That is,
\begin{eqnarray}
h(x) \sqsupseteq h'(x).
\end{eqnarray}
$\blacksquare$

\subsection{GCD is "unique"}
GCD is unique up to overall factor.

\subsubsection{Proof}
If we have 
\begin{eqnarray}
h(x), h'(x) \in \mathbb{K}[x]
\end{eqnarray}
as two GCD of $f(x), g(x) \in \mathbb{K}[x]$, then
\begin{eqnarray}
h(x) &\sqsupseteq& h'(x) \\
h'(x) &\sqsupseteq& h(x)
\end{eqnarray}
i.e., we have $l(x), m(x) \in \mathbb{K}[x]$ s.t.
\begin{eqnarray}
h(x) &=& l(x) * h'(x) \\
h'(x) &=& m(x) * h(x) 
\end{eqnarray}
So
\begin{eqnarray}
h(x) = \left( l(x) * m(x) \right) * h(x)
\end{eqnarray}
i.e., $l(x),m(x)$ are constant polynomial:
\begin{eqnarray}
l(x) * m(x) = 1, \forall x \in \mathbb{K}.
\end{eqnarray}
$\blacksquare$

\subsection{GCD algorithm}
\subsubsection{Pseudo code and recursive definition}
\begin{verbatim}
Input: f,g
Output: h
h := f
s := g
WHILE s /= 0 DO
  rem := remainder(h,s)
  h := s
  s := rem
\end{verbatim}

Here is the recursive definition.
\begin{enumerate}
\item Base case.
\begin{eqnarray}
h_0(x) &:=& f(x) \\
s_0(x) &:=& g(x) %\\
%r_0(x) &:=& \text{rem}\left( h_0(x), s_0(x) \right)
\end{eqnarray}

\item Induction step.
If $s_n(x)$ is zero polynomial, then
\begin{eqnarray}
\text{GCD}\left( f(x), g(x) \right) &:=& h_n(x),
\end{eqnarray}
otherwise
\begin{eqnarray}
h_{n+1}(x) &:=& s_n(x) \\
s_{n+1}(x) &:=& \text{rem}\left( h_n(x), s_n(x) \right) .
\end{eqnarray}

\end{enumerate}

\subsubsection{Proof}
Let us prove that this algorithm will terminate in finite steps.
From the division algorithm, observe
\begin{eqnarray}
\deg\left( s_0(x) \right) > \deg\left( s_1(x) \right) > \cdots,
\end{eqnarray}
i.e., $\left\{ \deg\left( s_n(x) \right) \right\}_n$ is strictly decreasing, non-negative sequence.
Since our termination condition is $\deg\left( s_n(x) \right)=0$, this algorithm will stop at finite steps.

If we put
\begin{eqnarray}
h_n(x) = q_n(x)*s_n(x) + \text{rem}\left( h_n(x), s_n(x) \right)
\end{eqnarray}
as the consequence of polynomial division, then
\begin{eqnarray}
\left<h_n(x), s_n(x) \right>
&=& \left< q_n(x)*s_n(x) + \text{rem}\left( h_n(x), s_n(x) \right), s_n(x) \right> \qquad\\
&=& \left<\text{rem}\left( h_n(x), s_n(x) \right), s_n(x) \right> \\
&=& \left<s_{n+1}(x), h_{n+1}(x) \right>
\end{eqnarray}
Therefore, if $s_n(x)$ is zero polynomial, then we get
\begin{eqnarray}
\left<  f(x), g(x) \right>
&=& \left<h_0(x), s_0(x) \right> \\
\nonumber
&\vdots& \\
&=& \left<h_{n-1}(x), s_{n-1}(x) \right> \\
&=& \left<h_n(x), 0 \right> \\
&=& \left<h_n(x)\right>
\end{eqnarray}
So
\begin{eqnarray}
\text{GCD}\left( f(x), g(x) \right) = h_n(x).
\end{eqnarray}
$\blacksquare$

\subsection{B\"ezout's identity (Exercise 4 \S1.5)}
\label{BezoutLemma}
For arbitrary polynomials $f(x), g(x) \in \mathbb{K}[x]$, there are $a(x), b(x) \in \mathbb{K}[x]$ s.t.
\begin{eqnarray}
f(x) * a(x) + g(x) * b(x) = \text{GCD}\left( f(x), g(x) \right).
\end{eqnarray}
To prove this claim, we extend our GCD algorithm.

\subsection{Extended GCD}
The following algorithm is a constructive proof for B\"ezout lemma in \S\ref{BezoutLemma}.
\begin{enumerate}
\item Base case.
\begin{eqnarray}
\left( r_0(x), s_0(x), t_0(x) \right) &:=& \left( f(x), 1, 0\right) \\
\left( r_1(x), s_1(x), t_1(x) \right) &:=& \left( g(x), 0, 1\right)
\end{eqnarray}

\item Induction step.
Define $n \geq 2$,
\begin{eqnarray}
q_n(x) := \text{quot}\left( r_{n-2}(x), r_{n-1}(x) \right)
\end{eqnarray}
and
\begin{eqnarray}
r_n(x) &:=& \text{rem}\left( r_{n-2}(x), r_{n-1}(x) \right) \qquad \\
&=& r_{n-2}(x) - q_n(x) * r_{n-1}(x) \\
s_n(x) &:=& s_{n-2}(x) - q_n(x) * s_{n-1}(x) \\
t_n(x) &:=& t_{n-2}(x) - q_n(x) * t_{n-1}(x) 
\end{eqnarray}

If $r_{n+1}(x)$ is zero polynomial, then
\begin{eqnarray}
\text{GCD}\left( f(x), g(x) \right) := r_n(x).
\end{eqnarray}

\end{enumerate}

\subsubsection{Proof}
This algorithm will also terminate, since
\begin{eqnarray}
\deg\left(r_n(x) \right) > \deg\left(r_{n+1}(x) \right).
\end{eqnarray}
Observe that for $n=0,1$ they satisfy so called B\"ezout identity
\begin{eqnarray}
r_n(x) &=& s_n(x) * f(x) + t_n(x) * g(x).
\end{eqnarray}
We claim this holds for all $n$.
If we assume this is the case for $0,1, \cdots, n(\geq 1)$, then
\begin{eqnarray}
r_{n+1}(x)
&:=& r_{n-1}(x) - q_n(x) * r_{n}(x) \\
\nonumber &=& s_{n-1}(x) * f(x) + t_{n-1}(x) * g(x) \\
&&- q_n(x) * \left( s_n(x) * f(x) + t_n(x) * g(x) \right) \qquad \qquad \\
\nonumber &=& \left( s_{n-1}(x) + q_n(x) * s_n(x) \right) * f(x) \\
&&+ \left( s_{n-1}(x) + q_n(x) * s_n(x) \right) * g(x) \\
&=& s_{n+1}(x) * f(x) + t_{n+1}(x) * g(x)
\end{eqnarray}

Now
\begin{eqnarray}
\left< r_n(x), r_{n+1}(x) \right>
&=& \left< r_n(x), r_{n-1}(x) - q_{n+1}(x) * r_n(x) \right> \\
&=& \left< r_n(x), r_{n-1}(x) \right>
\end{eqnarray}
we get
\begin{eqnarray}
\left< f(x), g(x) \right> &:=& \left< r_0(x), r_1(x) \right> \\
&=& \left< r_1(x), r_2(x) \right> \\
\nonumber &\vdots& 
\end{eqnarray}
Therefore, if we meet $r_{n+1}(x) = 0$, then
\begin{eqnarray}
\left< f(x), g(x) \right>  &=& \left< r_n(x), 0 \right> \\
&=& \left< r_n(x)\right> \\
\end{eqnarray}
so
\begin{eqnarray}
\text{GCD}\left( f(x), g(x) \right) := r_n(x).
\end{eqnarray}
For this $n$ with $r_{n+1}(x) = 0$, we have
\begin{eqnarray}
r_n(x) = s_n(x) * f(x) + t_n(x) * g(x).
\end{eqnarray}
$\blacksquare$

%%%%%%%%%%%%% GCF %%%%%%%%%%%%%

\chapter{Gr\"obner Bases}
I'd like to rewrite AAC related sections from scratch.

\section{Introduction}

%\section{Orderings on the Monomials in $\mathbb{K}[x_1, \cdots, x_n]$}
\section{Orderings on the Monomials in $\mathbb{K}\left[x_1, ... , x_n\right]$}
\subsection{Definition of monomial order $(\mathbb{N}^n, >)$}
\label{DefOfMonomialOrder}
A monomial order on $\mathbb{K}[x_1, \cdots, x_n]$ is a relation $>$ on $\mathbb{N}^n$, or on a set $\{x^\alpha | \alpha \in \mathbb{N}^n \}$, satisfying
\begin{eqnarray}
\text{$(\mathbb{N}^n, >)$ is totally ordered} \\
\alpha > \beta, \gamma \in \mathbb{N}^n \Rightarrow \alpha + \gamma > \beta + \gamma \\
\label{wellOrderedMonomialOrder}
\text{$(>, \mathbb{N}^n)$ is well-ordered.}
\end{eqnarray}

The following lemma will help us understand what the well-ordering condition of the third part of above definition:

\subsection{A condition for $(\mathbb{N}^n, >)$ is well-ordered}
\label{wellOrderingCondition}
An order $>$ on $\mathbb{N}^n$ is well-ordered iff $\forall$ strictly decreasing sequence $\{\alpha(i)\}_i$ in $\mathbb{N}^n$ will terminate:
\begin{eqnarray}
\alpha(1) > \alpha(2) > \cdots > \alpha(m).
\end{eqnarray}

\subsubsection{Proof}
We shall prove this in contrapositive form.\footnote{For $P\Rightarrow Q$, its contraposition is $\lnot Q \Rightarrow \lnot P$.}

If $(\mathbb{N}^n, >)$ is not well-ordered, then there is a non-empty subset $S \subset \mathbb{N}^n$ that has no smallest element.
We can pick $\alpha(1) \in S$, but $\alpha(1)$ is not the smallest element.
Thus $\exists \alpha(2) \in S$ s.t. $\alpha(1) > \alpha(2)$.
Continuing the same way, we can get an infinite strictly decreasing sequence in $S$:
\begin{eqnarray}
\alpha(1) > \alpha(2) > \cdots
\end{eqnarray}

Conversely, given such an infinite sequence, then
\begin{eqnarray}
\left\{\alpha(1), \alpha(2), \cdots \right\}
\end{eqnarray}
is a nonempty subset in $\mathbb{N}^n$ with no smallest element.
That is, we have shown
\begin{eqnarray}
(\mathbb{N}^n, >) \text{ is not well-ordered} \Leftrightarrow \exists \text{infinite strict decreasing sequence in } \mathbb{N}^n. \quad
\end{eqnarray}
$\blacksquare$

\subsubsection{Note}
This lemma guarantees that several algorithms must terminate in a finite number of steps.
At each step of the algorithm, some monomials strictly decrease with respect to a fixed monomial order.

\subsection{Terminologies}
Let 
\begin{eqnarray}
f = \sum_{\alpha} a_\alpha x^\alpha \in \mathbb{K}[x_1, \cdots, x_n]
\end{eqnarray}
be a nonzero polynomial, and $>$ is a (fixed) monomial order.

The multi degree of $f$ is given by
\begin{eqnarray}
\label{multiDegree}
MD(f) := \max \left( \alpha \in \mathbb{N}^n, a_\alpha \neq 0 \right) = \alpha_{\max} \in \mathbb{N}^n
\end{eqnarray}
with respect to the monomial order $>$.

The leading coefficient of $f$ is
\begin{eqnarray}
LC(F) := a_{\alpha_{\max}} \in \mathbb{K}.
\end{eqnarray}
The leading monomial of $f$ is
\begin{eqnarray}
LM(f) := x^{\alpha_{\max}},
\end{eqnarray}
of course this is a term with 1 as its coefficient.
The leading term of $f$ is
\begin{eqnarray}
LT(f) := LC(F) * LM(f) = a_{\alpha_{\max}} * x^{\alpha_{\max}}.
\end{eqnarray}

\subsection{Lemma (Multiplications and summations)}
\label{multiplication}
Let $f, g \in \mathbb{K}[x_1, \cdots, x_n]$ be nonzero polynomial.
Then
\begin{eqnarray}
MD(f * g) = MD(f) + MD(g),
\end{eqnarray}
and if $f+g \neq 0$, then
\begin{eqnarray}
MD(f+g) \leq \max(MD(f), MD(g)).
\end{eqnarray}
If, in addition, $MD(f) \neq MD(g)$, then equality holds.\footnote{In this case, there is no cancellation on the leading terms.}

\subsubsection{Proof}
Let us write
\begin{eqnarray}
f &=& a_{\alpha_{\max}} * x^{\alpha_{\max}} + O(x^{\alpha_{\max}-1}) \\
g &=& b_{\beta_{\max}} * x^{\beta_{\max}} + O(x^{\beta_{\max}-1})
\end{eqnarray}
then clearly
\begin{eqnarray}
f*g = a_{\alpha_{\max}} * b_{\beta_{\max}} * x^{\alpha_{\max}+\beta_{\max}} + O(x^{\alpha_{\max}-1+\beta_{\max} -1})
\end{eqnarray}
and
\begin{eqnarray}
f + g =  a_{\alpha_{\max}} * x^{\alpha_{\max}} + O(x^{\alpha_{\max}-1}) + b_{\beta_{\max}} * x^{\beta_{\max}} + O(x^{\beta_{\max}-1}) 
\end{eqnarray}
There might be a cancellation on the leading terms.\\
$\blacksquare$

%\section{Division algorithm in}% $\mathbb{K}[x_1, \cdots, x_n]$}
\section{Division algorithm in $\mathbb{K}[x_1, ... , x_n]$}
\subsection{"is divisible by" as a binary relation}
\label{isDivisibleBy}
Define a binary relation $\sqsupseteq$ ("is divisible by") on $\mathbb{K}[x_1, \cdots, x_n]$ by $\forall r,f \in \mathbb{K}[x_1, \cdots, x_n]$,
\begin{eqnarray}
r \sqsupseteq f :\Leftrightarrow \exists q \in \mathbb{K}[x_1, \cdots, x_n], r = q*f.
\end{eqnarray}
i.e. $r$ is divisible by $f$ iff there is some polynomial $q$ and $r = q*f$.\footnote{$r|f$ is the usual notation for "is divisible", but I would like to use an asymmetric notation for such an asymmetric binary relation.}

Let us write the negation:
\begin{eqnarray}
r \sqsubset f :\Leftrightarrow \nexists q \in \mathbb{K}[x_1, \cdots, x_n], r = q*f
\end{eqnarray}
i.e. $r$ is not divisible by f.

%\subsection{Division algorithm in }%$\mathbb{K}[x_1, \cdots, x_n]$}
\subsection{Division algorithm in $\mathbb{K}[x_1, ... , x_n]$}
\label{DivisionAlgorithmInN}
Let $>$ be a (fixed) monomial order\footnote{We will introduce some monomial orders in \S\ref{MonomialOrders}.} and $(\mathbb{N}^n, >)$ be well-ordered, and
\begin{eqnarray}
F := (f_1, \cdots, f_s)
\end{eqnarray}
be an ordered $s$-tuple (or a list) of polynomials in $\mathbb{K}[x_1, \cdots, x_n]$.
Then $\forall f \in \mathbb{K}[x_1, \cdots, x_n]$,
\begin{eqnarray}
\exists a_i, r  \in \mathbb{K}[x_1, \cdots, x_n] \text{ s.t. } f = a_1*f_1 + \cdots + a_s * f_s + r,
\end{eqnarray}
and either $r=0$ or $r$ is a linear combination, with coefficients in $\mathbb{K}$, of monomials, none of which is divisible by any of $LT(f_1), \cdots, LT(f_s)$:\footnote{See \S\ref{isDivisibleBy}}
\begin{eqnarray}
\label{remainderHasSmallerMultiDegree}
r \sqsubset LT(f_1), \cdots, \text{ and } r \sqsubset LT(f_s).
\end{eqnarray}
Furthermore, if $a_i f_i \neq 0$., then we have \footnote{This is notation abuse, $\geq$ means $>$ or $=$ (as an ordered monomial $\mathbb{N}^n$).}
\begin{eqnarray}
\label{decreasingMultiDegree}
MD(f) \geq MD(a_i * f_i)
\end{eqnarray}
with respect to the fixed monomial order.

\newpage
\subsubsection{Pseudo code}
\begin{verbatim}
Input: f_1 .. f_s,f
Output: a_1 .. a_s,r
a_1 := 0; ..; a_s := 0; r := 0
p := f

WHILE p /= 0 DO
  i := 1
  divisionOccured := False
  WHILE i <= s AND divisionOccured = False DO
    IF LT(f_i) divides p THEN
      a_i := a_i + LT(p) / LT(f_i)
      p   := p - (LT(p) / LT(f_i)) * f_i
      divisionOccured := True
    Else
      i := i + 1
  IF divisionOccured = False THEN
    r := r + LT(p)
    p := p - LT(p)
\end{verbatim}

\subsubsection{Proof}
To prove that the above algorithm works, we'll first show that
\begin{eqnarray}
\label{intermediateRepresentation}
f = a_1*f_1 + \cdots + a_s * f_s + p + r
\end{eqnarray}
holds at every stage, by induction on steps.
This is clearly true for the initial values of $a_1, \cdots, a_s (= 0),p(=f),r(=0)$.
Suppose eq.(\ref{intermediateRepresentation}) holds at one step of the algorithm.
If the next step is a Division Step, i.e.,
\begin{eqnarray}
\verb|LT(f_i) divides p|
\end{eqnarray}
then the following combination
\begin{eqnarray}
a_i f_i + p = \left(a_i + \frac{LT(p)}{LT(f_i)} \right) * f_i + \left(p - \frac{LT(p)}{LT(f_i)}* f_i \right)
\end{eqnarray}
stays unchanged.
Since all other variables are unaffected, eq.(\ref{intermediateRepresentation}) remains true in this case.

On the other hand, if the next step is a Remainder Step, then both $r$ and $p$ will be changed, 
\begin{eqnarray}
\verb|r := r + LT(p)| \\
\verb|p := p - LT(p)|
\end{eqnarray}
but the sum is unchanged:
\begin{eqnarray}
r+p = \left(r + LT(p) \right) + \left( r + LT(p)\right).
\end{eqnarray}
Also eq.(\ref{intermediateRepresentation}) remains true in this case.

Note that this algorithm comes to halt when $p=0$, see the first \verb|WHILE| statement.
In this situation (when $p=0$), eq.(\ref{intermediateRepresentation}) becomes
\begin{eqnarray}
f = a_1*f_1 + \cdots + a_s * f_s + r.
\end{eqnarray}

Finally, we need to show that this algorithm will terminate.
The key observation is the rewriting process:
\begin{eqnarray}
\verb| p := p - (LT(p) / LT(f_i)) * f_i |
\end{eqnarray}
By Lemma in \S\ref{multiplication},
\begin{eqnarray}
LT \left( \frac{LT(p)}{LT(f_i)} * f_i \right) = \frac{LT(p)}{LT(f_i)} * LT(f_i) = LT(p).
\end{eqnarray}
If $p$ becomes $0$ in this process, then this algorithm halts.
Even if $p \neq 0$, the leading term will vanish, and the multi degree must decrease strictly.

If this algorithm never terminated, that is, we never meet $p =0$, then we could get an infinite decreasing sequence of multi degrees.
But since our monomial order satisfies eq.(\ref{wellOrderedMonomialOrder}), i.e. $\forall$ strict decreasing sequence will terminate (as \S\ref{wellOrderingCondition}) and eventually $p = 0$. 

Finally, consider $MD(f)$ and $MD(a_i f_i)$.
Every term in $a_i$ is of the form 
\begin{eqnarray}
\frac{LT(p)}{LT(f_i)}
\end{eqnarray}
for some $p$.
Above algorithm starts with $p = f$ and the multi degree of $p$'s are decreasing:
\begin{eqnarray}
MD(f = p) > MD(p') > \cdots.
\end{eqnarray}
This shows that for every step, either $>$ or $=$ holds
\begin{eqnarray}
MD(f) \geq MD(p)
\end{eqnarray}
and
\begin{eqnarray}
MD(a_i f_i) = MD\left(\frac{LT(p)}{LT(f_i)}* f_i \right) = MD(p) \leq MD(f) 
\end{eqnarray}
$\blacksquare$

\subsection{The ideal membership problems}
\label{membership}
As a consequence of the above division algorithm \S\ref{DivisionAlgorithmInN} in multi-variables is the followings; if after division of $f$ by the ordered tuple $F := (f_1, \cdots, f_s)$ we obtain $r=0$, then we have
\begin{eqnarray}
f = a_1 * f_1 + \cdots a_t * f_t.
\end{eqnarray}
Thus $r = 0$ is a sufficient condition for ideal membership:
\begin{eqnarray}
r = 0 \Rightarrow f \in \left< f_1, \cdots, f_s \right>.
\end{eqnarray}
However, we'll see soon, $r=0$ is not a necessary condition for being in the ideal.\footnote{We'll find the iff condition for an ideal membership in \S\ref{IMC}, using Gr\"obner basis.}

\subsubsection{Example}
Let us consider $\mathbb{K}[x,y]$ with lex order and
\begin{eqnarray}
f_1 &:=& x*y + 1\\
f_2 &:=& y^2 -1
\end{eqnarray}
Dividing
\begin{eqnarray}
f = x*y^2 -x
\end{eqnarray}
by an ordered 2-tuple $(f_1, f_2)$, the result is
\begin{eqnarray}
f &=& x*y^2 -x \\
&=& y*(f_1 -1) -x \\
&=& y*f_1 -x-y.
\end{eqnarray}
On the other hand, by $(f_2, f_1)$, the result is
\begin{eqnarray}
f &=& x*y^2 -x \\
&=& x*(f_2 + 1) -x \\
&=& x*f_2 + 0\\
\Rightarrow f &\in& \left< f_1, f_2\right>.
\end{eqnarray}
Thus, the first trial show that even if $f \in \left< f_1, f_2\right>$, the reminder can be non zero.\\
$\blacksquare$

\section{Monomial Ideals and Dickson's Lemma}
\subsection{Definition of monomial ideals}
$I \subset \mathbb{K}[x_1, \cdots, x_n]$ is a monomial ideal iff the elements can be written as a finite sum form:
\begin{eqnarray}
\exists A \subset \mathbb{N}^n \text{ s.t. } I = \left\{\left. \sum_{i=1}^s h_i * x^{\alpha(i)} \right| \alpha(i) \in A, h_i \in \mathbb{K}[x_1, \cdots, x_n] \right\}
\end{eqnarray}
Then we write as a generator form:
\begin{eqnarray}
I := \left<\left. x^\alpha \right| \alpha \in A\right>
\end{eqnarray}

\subsection{Monomial ideal memberships}
\label{Lemma2OfSection2_3}
Let $I = \left<\left. x^\alpha \right| \alpha \in A\right> \subset \mathbb{K}[x_1, \cdots, x_n]$ be a monomial ideal.
Then a monomial $x^\beta$ is in $I$ iff 
\begin{eqnarray}
\exists \alpha \in A \text{ s.t. } x^\beta \sqsupseteq x^\alpha
\end{eqnarray}
i.e., $x^\beta$ is divisible by some $x^\alpha \in I$.

\subsubsection{Proof}
($\Leftarrow$)
If $x^\beta \sqsupseteq x^\alpha$, then there is some $h \in \mathbb{K}[x_1, \cdots, x_n]$ s.t.,
\begin{eqnarray}
x^\beta = h * x^\alpha.
\end{eqnarray}
So clearly $x^\beta \in I$ by the definition of ideal.

($\Rightarrow$)
Conversely, if $x^\beta \in I$, then we have an expression for $x^\beta$:
\begin{eqnarray}
x^\beta = \sum_{i=1}^s h_i * x^{\alpha(i)}
\end{eqnarray}
Now our generating set
\begin{eqnarray}
\left\{x^{\alpha(1)}, \cdots, x^{\alpha(s)} \right\}
\end{eqnarray}
is finite by definition of monomial ideals.
From the well-ordered property of our monomial order, this generating set has the minimum element.
Without loss of generality, we can put $\alpha(1)$ is the minimum\footnote{Indeed, we can put this generating set as $\left\{x^{\alpha(1)}< \cdots < x^{\alpha(s)} \right\}$.} and $h_1 \neq 0$, then
\begin{eqnarray}
x^\beta \sqsupseteq x^{\alpha(1)}.
\end{eqnarray}
$\blacksquare$

\subsection{Dickson's Lemma}
\label{Dickson}
Let $I = \left<\left. x^\alpha \right| \alpha \in A\right> \subset \mathbb{K}[x_1, \cdots, x_n]$ be a monomial ideal.
Then
\begin{eqnarray}
\exists \text{ finite } s \in \mathbb{N}, I = \left<x^{\alpha(1)}, \cdots, x^{\alpha(s)} \right>,
\end{eqnarray}
where $x^{\alpha(1)}, \cdots, x^{\alpha(s)} \in A$.
That is, every monomial ideal has a finite number of monomial generators.

\subsubsection{Proof}
By induction on $n$.\footnote{Here we uses the well-ordering property of $\mathbb{N}$.}

$n=1$ case, $I = \left<\left. x_1^{\alpha(1)} \right| {\alpha(1)} \in A \subset \mathbb{N} \right>$.
Then just take the smallest $\beta \in A$ and 
\begin{eqnarray}
I = \left< x_1^\beta \right>.
\end{eqnarray}

Now assume that $n>1$ and this theorem holds for up to $n-1$.
Consider the following form of monomial
\begin{eqnarray}
x^\alpha * y^m \in \mathbb{K}[x_1, \cdots, x_{n-1},y],
\end{eqnarray}
where
\begin{eqnarray}
\alpha \in \mathbb{N}^{n-1}, m \in \mathbb{N}.
\end{eqnarray}
and a monomial ideal
\begin{eqnarray}
I \subset \mathbb{K}[x_1, \cdots, x_{n-1},y].
\end{eqnarray}
Define an ideal in $\mathbb{K}[x_1, \cdots, x_{n-1}]$ (not in $\mathbb{K}[x_1, \cdots, x_{n-1}, y]$):
\begin{eqnarray}
J := \left\{ \left. x^\alpha \in \mathbb{K}[x_1, \cdots, x_{n-1}] \right| \exists m \in \mathbb{N} \text{ s.t. } x^\alpha y^m \in I \right\}
\end{eqnarray}
By our inductive hypothesis, there is a finite $s$ s.t.
\begin{eqnarray}
J = \left<x^{\alpha(1)}, \cdots, x^{\alpha(s)} \right>.
\end{eqnarray}

From this construction, $\forall \left. \alpha(i) \right|_{i=1}^s, \exists m_i \in \mathbb{N}$ s.t. $x^{\alpha(i)} y^{m_i} \in I$.
Now we can take 
\begin{eqnarray}
m := \max\left(\left. m_i \right|_{i=1}^s \right).
\end{eqnarray}
$0 \leq \forall j \leq m-1$, define an ideal in $\mathbb{K}[x_1, \cdots, x_{n-1}]$:
\begin{eqnarray}
J_j := \left\{ \left. x^\beta \in \mathbb{K}[x_1, \cdots, x_{n-1}] \right| x^\beta * y^j \in I \right\}.
\end{eqnarray}
Using our inductive hypothesis, again, we get a finite $s_j$:
\begin{eqnarray}
J_j =  \left<x^{\alpha_j(1)}, \cdots, x^{\alpha_j(s_j)} \right>.
\end{eqnarray}

Define a union of above generators
\begin{eqnarray}
\tilde{J} := \{ x^{\alpha(1)}, \cdots, x^{\alpha(s)}, x^{\alpha_0(1)}, \cdots, x^{\alpha_{m-1}(s_{m-1})}\} 
\end{eqnarray}
and write $\left< \tilde{J} \right>$ is an ideal generated by the elements of $\tilde{J}$.
From above construction, $\left< \tilde{J} \right>$ is a monomial ideal and
\begin{eqnarray}
\label{JJsubsetI}
\left< \tilde{J} \right> \subset I.
\end{eqnarray}

Then we can show that $\forall $monomial in $I$ is divisible by one of the element of $\tilde{J}$.
Since, $\forall x^\alpha * y^p \in I$, if $p \geq m$,
\begin{eqnarray}
x^\alpha * y^p  \sqsupseteq \exists x^{\alpha(i)} * y^m \in \left< \tilde{J} \right>,
\end{eqnarray}
else (i.e., $0 \leq p \leq m-1$),
\begin{eqnarray}
x^\alpha y^p  \sqsupseteq \exists x^{\alpha_p(j)} y^p \in J_p.
\end{eqnarray}
(In both case, $y$ part is done, and $x$ part is from the inductive hypothesis.)

Let us prove $\left< \tilde{J} \right> = I$.
Since $\forall f \in I$ of monomial ideal is written as the finite sum of monomials with appropriate factors:
\begin{eqnarray}
f = \sum_{\alpha,p} h_{\alpha,p} * x^\alpha * y^p, h_{\alpha,p} \in \mathbb{K}[x_1, \cdots, x_{n-1},y]
\end{eqnarray}
$\forall$ monomial $x^\alpha y^p$ of above is divisible by $\left< \tilde{J} \right>$, and this shows that
\begin{eqnarray}
I \subset \left< \tilde{J} \right>.
\end{eqnarray}
%In addition to the consequence we've got in eq.(\ref{JJsubsetI}), we get
Therefore,
\begin{eqnarray}
I = \left< \tilde{J} \right>.
\end{eqnarray}
$\blacksquare$

\subsection{$(\mathbb{N}^n, >)$ is well-ordering iff "positive definite"}
\label{WellOrderingIffPD}
Consider $(\mathbb{N}^n, >)$ of a binary relation.
If two conditions hold, 
\begin{eqnarray}
> \text{ is total order} \\
\forall \alpha,\beta,\gamma \in \mathbb{N}^n, \alpha > \beta \Rightarrow \alpha + \gamma > \beta + \gamma
\end{eqnarray}
then \footnote{We abuse 0; $0 \in \mathbb{N}^n$.} the followings are equivalent:
\begin{eqnarray}
\label{simpleWOProperty}
(\mathbb{N}^n, >) \text{ is well-ordering } \Leftrightarrow \forall \alpha \in \mathbb{N}^n, \alpha \geq 0.
\end{eqnarray}

\subsubsection{Proof}
($\Rightarrow$)
Assume $(\mathbb{N}^n, >)$ be well-ordering, then we can pick a minimum element:
\begin{eqnarray}
\alpha_0 \in \mathbb{N}^n \text{ s.t. } \forall \alpha \in \mathbb{N}^n, \alpha_0 \leq \alpha
\end{eqnarray}
It suffices to show that $0 \geq \alpha_0$; if $0 > \alpha_0$ were true, using the second assumption,
\begin{eqnarray}
\alpha_0 = 0 + \alpha_0 > \alpha_0 + \alpha_0 = 2\alpha_0
\end{eqnarray}
i.e., $2\alpha_0$ became smaller than the minimum $\alpha_0$, contradiction.
So,
\begin{eqnarray}
0 \leq \alpha_0.
\end{eqnarray}

($\Leftarrow$)
Assume $\forall \alpha \in \mathbb{N}^n, \alpha \geq 0$.
Let
\begin{eqnarray}
(\emptyset \neq )A \subset \mathbb{N}^n
\end{eqnarray}
be a non-empty subset and we shall prove $\exists$ a smallest element of $A$.

Consider an ideal of A:
\begin{eqnarray}
I := \left< \left. x^\alpha \right| \alpha \in A\right> 
\end{eqnarray}
By Dickson's lemma, there is a finite generator:
\begin{eqnarray}
I = \left< x^{\alpha(1)}, \cdots, x^{\alpha(s)} \right> 
\end{eqnarray}
We can replace the order of such generators by using the monomial total\footnote{By definition, our monomial order is total.} order $<$, 
\begin{eqnarray}
I = \left< x^{\alpha(1)} < \cdots< x^{\alpha(s)} \right> 
\end{eqnarray}

Now we can proove $x^{\alpha(1)}$ of the smallest generator is the smallest element of $I$, and this is the same as $\alpha(1) \in A$ is the smallest.
Since $\forall \alpha \in A$,
\begin{eqnarray}
x^\alpha \in I = \left< x^{\alpha(1)} < \cdots< x^{\alpha(s)} \right>,
\end{eqnarray}
and from the lemma \S\ref{Lemma2OfSection2_3}, $1 \leq \exists i \leq s$,
\begin{eqnarray}
x^\alpha \sqsupseteq x^{\alpha(i)}
\end{eqnarray}
and this means
\begin{eqnarray}
\exists \gamma \in \mathbb{N}^n \text{ s.t. } \alpha = \alpha(i) + \gamma.
\end{eqnarray}
By our hypothesis, $\gamma \geq 0$, so if $\gamma > 0$,
\begin{eqnarray}
\alpha = \alpha(i) + \gamma > \alpha(i) + 0 =  \alpha(i) \geq \alpha(1).
\end{eqnarray}
else $\gamma = 0$,
\begin{eqnarray}
\alpha = \alpha(i) + \gamma = \alpha(i) + 0 =  \alpha(i) \geq \alpha(1).
\end{eqnarray}
This means $\alpha(1)$ is the smallest element of given $A$.\\
$\blacksquare$

\subsection{Another definition of monomial orders}
As a result of the proof in \S\ref{WellOrderingIffPD}, we can simplify the monomial order $(\mathbb{N}^n, >)$ in \S\ref{DefOfMonomialOrder}, we can replace the 3rd condition by eq.(\ref{simpleWOProperty}) of "positive definite":

A monomial order on $\mathbb{K}[x_1, \cdots, x_n]$ is a relation $>$ on $\mathbb{N}^n$, or on a set $\{x^\alpha | \alpha \in \mathbb{N}^n \}$, satisfying
\begin{eqnarray}
\text{$(\mathbb{N}^n, >)$ is totally ordered} \\
\alpha > \beta, \gamma \in \mathbb{N}^n \Rightarrow \alpha + \gamma > \beta + \gamma \\
\forall \alpha \in \mathbb{N}^n, \alpha \geq 0.
\end{eqnarray}


%%%%%%%%%%%%%%%%%%%%%%%%%%%%%%%%%%%%%%%%%%%%%%%%%%

%\subsection{Monomial orders in}% $\mathbb{K}[x_1, \cdots, x_n]$}
\subsection{Monomial orders in $\mathbb{K}[x_1, ... , x_n]$}
\label{MonomialOrders}
We introduce some important instances of monomial orders.

\subsubsection{Lexicographic order $>_{lex}$}
For $\alpha := (\alpha_1, \cdots, \alpha_n), \beta := (\beta_1, \cdots, \beta_n) \in \mathbb{N}^n$, we say $\alpha >_{lex} \beta$ iff, in the "vector" difference
\begin{eqnarray}
(\alpha_1- \beta_1, \cdots, \alpha_n -\beta_n)
\end{eqnarray}
the leftmost nonzero entry is positive.
\begin{eqnarray}
x * y^2 &>_{lex}& y^3 * z^4 \\
x^3 * y^2 * z^4 &>_{lex}& x^3 * y^2 * z \\
x &>_{lex}& y
\end{eqnarray}

In the following (real) code, we assume the length of the list is the same:\footnote{
We use builtin list as an expression for a monomial in Haskell:
\begin{eqnarray}
x^\alpha = x_1^{\alpha_1} * \cdots * x_n^{\alpha_n} \to [\alpha_1, \cdots, \alpha_n]
\end{eqnarray}
}

\begin{verbatim}
> type Monomial = [Int]

> lexO :: Monomial -> Monomial -> Ordering
> lexO [] [] = EQ
> lexO (a:as) (b:bs)
>   | a > b = GT
>   | a < b = LT
>   | otherwise = lexO as bs

  *MonomialOrder> lexO [1,2,0] [0,3,4]
  GT
  *MonomialOrder> lexO [3,2,4] [3,2,1]
  GT
  *MonomialOrder> lexO [1,0,0] [0,1,0]
  GT
\end{verbatim}

\paragraph{$>_{lex}$ is a monomial ordering}
Since the number of variables is finite ($n$ in $\mathbb{K}[x_1, \cdots, x_n]$), the above algorithm will terminate, i.e., all two monomials $\alpha, \beta$ are in one of
\begin{eqnarray}
\alpha >_{lex} \beta, \alpha = \beta, \beta >_{lex} \alpha.
\end{eqnarray}
So $>_{lex}$ is total.

Since for every entry of $\forall \gamma$, 
\begin{eqnarray}
(\alpha_i + \gamma_i) - (\beta_i + \gamma_i) = \alpha_i  - \beta_i,
\end{eqnarray}
we have $\forall \gamma$, 
\begin{eqnarray}
\alpha >_{lex} \beta \Leftrightarrow \alpha + \gamma >_{lex} \beta + \gamma.
\end{eqnarray}

Finally, for every entry,
\begin{eqnarray}
\alpha_i = \alpha_i - 0 \geq 0 \Leftrightarrow \alpha \geq_{lex} 0
\end{eqnarray}
since all the entry in $\alpha$ is positive definite in $\mathbb{N}$.
Therefore, $>_{lex}$ is a monomial ordering.\\
$\blacksquare$

\subsubsection{Graded Lexicographic order $>_{grlex}$}
For $\alpha, \beta \in \mathbb{N}^n$, define $\alpha >_{grlex} \beta$ iff
\begin{eqnarray}
|\alpha| > |\beta|
\end{eqnarray}
or
\begin{eqnarray}
|\alpha| = |\beta| \text{ and } \alpha >_{lex} \beta
\end{eqnarray}
where
\begin{eqnarray}
|\alpha| := \sum_{i} \alpha_i.
\end{eqnarray}
\begin{eqnarray}
x * y^2 &>_{grLex}& y^3 * z^4 \\
x^3 * y^2 * z^4 &>_{grLex}& x^3 * y^2 * z \\
x &>_{grLex}& y
\end{eqnarray}

\begin{verbatim}
> grLex :: Monomial -> Monomial -> Ordering
> grLex [] [] = EQ
> grLex a b
>   | sum a > sum b = GT
>   | sum a < sum b = LT
>   | otherwise = lexO a b
  
  *MonomialOrder> grLex [1,2,3] [3,2,0]
  GT
  *MonomialOrder> grLex [1,2,4] [1,1,5]
  GT
\end{verbatim}

\paragraph{$>_{grLex}$ is a monomial ordering}
$>_{grLex}$ is clearly total order, since the comparison process will terminate in finite steps.

$\forall \gamma$, we have already seen $\alpha >_{lex} \beta \Leftrightarrow \alpha + \gamma >_{lex} \beta + \gamma$ in $>_{lex}$ case, and
\begin{eqnarray}
|\alpha + \gamma| > |\beta + \gamma| \Leftrightarrow |\alpha| > |\beta|
\end{eqnarray}
since $|\alpha + \gamma| = |\alpha| + |\gamma|$.
So $\gamma \in \mathbb{N}^n$,
\begin{eqnarray}
\alpha >_{grLex} \beta \Leftrightarrow \alpha + \gamma >_{grLex} \beta + \gamma.
\end{eqnarray}

Finally, $\forall \alpha$ is positive definite,
\begin{eqnarray}
|\alpha| \geq 0 \Rightarrow \alpha \geq_{grLex} 0.
\end{eqnarray}
$\blacksquare$

\subsubsection{Graded Reversed Lex}
For $\alpha, \beta \in \mathbb{N}^n$, define $\alpha >_{grevlex} \beta$ iff
\begin{eqnarray}
|\alpha| > |\beta|
\end{eqnarray}
or
\begin{eqnarray}
|\alpha| = |\beta| \text{ and the rightmost nonzero entry of difference in $\mathbb{N}^n$ is negative.}  \qquad
\end{eqnarray}
\begin{eqnarray}
x^4 * y^7 * z &>_{gRevLex}& x^4 * y^2 * z^3 \\
x * y^5 * z^2 &>_{gRevLex}& x^4 * y * z^3
\end{eqnarray}

\begin{verbatim}
> gRevLex :: Monomial -> Monomial -> Ordering
> gRevLex [] [] = EQ
> gRevLex a b
>   | sum a > sum b = GT
>   | sum a < sum b = LT
>   | otherwise = helper (reverse a) (reverse b)
>   where
>     helper (a:as) (b:bs) 
>       | a < b     = GT
>       | otherwise = helper as bs

  *MonomialOrder> gRevLex [4,7,1] [4,2,3]
  GT
  *MonomialOrder> gRevLex [1,5,2] [4,1,3]
  GT
\end{verbatim}

\paragraph{$>_{gRevLex}$ is a monomial ordering}
This comparison algorithm terminate finitely, so $>_{gRevLex}$ is a total order, and the comparison does not change under $\alpha, \beta \leftrightarrow \alpha + \gamma, \beta + \gamma$.
In addition, $>_{gRevLex}$ is positive definite since
\begin{eqnarray}
|\alpha| \geq 0 \Rightarrow \alpha \geq_{grLex} 0.
\end{eqnarray}
$\blacksquare$

%%%%%%%%%%%%%%%%%%%%%%%%%%%%%%%%%%%%%%%%%%%%%%%%%%


\section{The Hilbert Basis Theorem and Gr\"obner Bases}
\subsection{Definition of the ideal of leading terms}
Let $I \subset \mathbb{K}[x_1, \cdots, x_{n}]$ be an ideal other than $\{0\}$.
Define a set of leading terms of $I$, 
\begin{eqnarray}
LT(I) := \left\{ \left. LT(f) \right| f \in I \right\} 
\end{eqnarray}
and the ideal generated by that set:
\begin{eqnarray}
\left< LT(I) \right>  := \left< \left\{ \left. LT(f) \right| f \in I \right\} \right>.
\end{eqnarray}

\subsection{Trivial inclusion}
\label{LargerLemma}
For $I = \left< f_1, \cdots, f_s \right>$, then $\left< LT(f_1), \cdots, LT(f_s) \right> \subset \left< LT(I) \right>$.

\subsubsection{Proof}
By definition, the leading term of $f_1$ is
\begin{eqnarray}
LT(f_1) \in \left< LT(I) \right>.
\end{eqnarray}
This means $\forall$ generators of the ideal $\left< LT(f_1), \cdots, LT(f_s) \right>$ is in $\left< LT(I) \right>$, and
\begin{eqnarray}
\left< LT(f_1), \cdots, LT(f_s) \right> \subset \left< LT(I) \right>
\end{eqnarray}
$\blacksquare$

\subsubsection{Example}
$\left< LT(I) \right>$ can be strictly larger than $\left< LT(f_1), \cdots, LT(f_s) \right>$.

Consider
\begin{eqnarray}
f_1 &:=& x^3-2*x*y \\
f_2 &:=& x^2*y -2*y^2 + x \\
I &:=& \left< f_1, f_2 \right>
\end{eqnarray}
and use the grlex ordering on monomial in $\mathbb{K}[x,y]$.
Then we have
\begin{eqnarray}
-y * f_1+x * f_2 &=& -y * (x^3-2*x*y) + x * (x^2*y -2*y^2 + x) \qquad\qquad\\
&=& x^2
\end{eqnarray}
so that
\begin{eqnarray}
x^2 \in I.
\end{eqnarray}
Thus
\begin{eqnarray}
LT(x^2) = x^2 \in \left< LT(I)  \right>,
\end{eqnarray}
but $x^2$ is not divisible by leading terms
\begin{eqnarray}
LT(f_1) &:=& x^3 \\
LT(f_2) &:=& x^2 * y,
\end{eqnarray}
therefore,
\begin{eqnarray}
x^2 \notin \left< LT(f_1), LT(f_2)  \right>.
\end{eqnarray}
This example shows that
\begin{eqnarray}
\left< LT(f_1), LT(f_2)  \right> \subsetneq  \left< LT(I) \right>.
\end{eqnarray}
$\blacksquare$

\subsection{$\left< LT(I) \right>$ is a monomial ideal, and finitely generated}
\label{monomialIdeals}
$\forall I \subset \mathbb{K}[x_1, \cdots, x_{n}]$ be an ideal, then
\begin{eqnarray}
\left< LT(I) \right>
\end{eqnarray}
is a monomial ideal, and there is a finite set of generators $\left< LT(I) \right> = \left< LT(g_1), \cdots, LT(g_t) \right>$.

\subsubsection{Proof}
By definition,
\begin{eqnarray}
\left< LT(I) \right>  := \left< \left\{ \left. LT(f) \right| f \in I \right\} \right>
\end{eqnarray}
and $\forall f \in I$,
\begin{eqnarray}
f \neq 0 \Rightarrow \exists c \in \mathbb{K}, \text{ s.t. } c* LT(f) \text{ is a monomial}.
\end{eqnarray}
If we write such a monomial $g$, then
\begin{eqnarray}
\left< LT(I) \right>  := \left< \left\{ \left. g \in I \right| g (\neq 0) \text{ is a monomial} \right\} \right>
\end{eqnarray}
i.e. $\left< LT(I) \right> $ is a monomial ideal.

Since we have proved that $\left< LT(I) \right>$ is a monomial ideal, Dickson's lemma in \S\ref{Dickson} tells us that there is a finite number of monomials:
\begin{eqnarray}
\left< LT(I) \right> = \left< g_1, \cdots, g_t \right>,
\end{eqnarray}
but since $g$'s are monomials
\begin{eqnarray}
\left< LT(I) \right> = \left< LT(g_1), \cdots, LT(g_t) \right>.
\end{eqnarray}
$\blacksquare$

\subsection{Hilbert Basis Theorem}
\label{HilbertBasisTheorem}
Every ideal $I \subset \mathbb{K}[x_1, \cdots, x_{n}]$ has a finite generators.

\subsubsection{Proof}
If $I = \{0\}$, this singleton set $\{0\}$ is certainly finite.\footnote{Here we have two different $\{0\}$, the first one is a trivial ideal (as a structured set), and the other one is a singleton set of zero.}

If $I$ contains some nonzero polynomial, we can construct a finite generating set for $I$ as follows.
By \S\ref{monomialIdeals}, there is a set of finite $g_1, \cdots, g_t \in I$ s.t. $\left< LT(I) \right> = \left< LT(g_1), \cdots, LT(g_t) \right>$.
We claim that $I = \left< g_1, \cdots, g_t \right>$.

Since each $g$'s are in $I$, 
\begin{eqnarray}
\label{HilbertSubset}
\left< g_1, \cdots, g_t \right> \subset I.
\end{eqnarray}

Conversely, $\forall f \in I$, by the division algorithm in \S\ref{DivisionAlgorithmInN} we can divide f by the ideal $\left< g_1, \cdots, g_t \right>$:
\begin{eqnarray}
f = a_1 * g_1 + \cdots + a_t * g_t + r,
\end{eqnarray}
where no term of $r$ is divisible by $LT(g_1), \cdots, LT(g_t)$.
Now
\begin{eqnarray}
r = f - \left( a_1 * g_1 + \cdots + a_t * g_t \right) \in I
\end{eqnarray}
If $r \neq 0$, then 
\begin{eqnarray}
LT(r) \in \left< LT(I) \right> = \left< LT(g_1), \cdots, LT(g_t) \right>,
\end{eqnarray}
but by \S\ref{Lemma2OfSection2_3}, there is some $LT(g_i)$ s.t.
\begin{eqnarray}
LT(r) \sqsupseteq LT(g_i)
\end{eqnarray}
since $\left< LT(g_1), \cdots, LT(g_t) \right>$ is an monomial ideal by \S\ref{monomialIdeals} and $LT(r)$ is a term (i.e., a monomial times a coefficient).
This clearly contradicts our definition of remainder, and consequently,
\begin{eqnarray}
\label{nullRemainder}
r = 0 \Leftrightarrow f = a_1 * g_1 + \cdots + a_t * g_t.
\end{eqnarray}
Thus
\begin{eqnarray}
f \in \left< g_1, \cdots, g_t \right>
\end{eqnarray}
and this means 
\begin{eqnarray}
\label{HilbertSupset}
I \subset \left< g_1, \cdots, g_t \right>.
\end{eqnarray}

Finally we have
\begin{eqnarray}
\label{HilbertGenerators}
I = \left< g_1, \cdots, g_t \right>.
\end{eqnarray}
$\blacksquare$

\subsection{Definition of Gr\"obner basis}
\label{DefOfGroebner}
Let us fix a monomial order $>$.
A finite subset of given ideal $I$
\begin{eqnarray}
G := \{g_1, \cdots, g_t\} \subset I
\end{eqnarray}
is a Gr\"obner basis iff 
\begin{eqnarray}
\left< LT(g_1), \cdots, LT(g_t) \right> = \left< LT(I)\right>
\end{eqnarray}
holds.\footnote{
From \S\ref{LargerLemma}, we already have $\left< LT(g_1), \cdots, LT(g_t) \right> \subset \left< LT(I) \right>$, so $\left< LT(g_1), \cdots, LT(g_t) \right> \supset \left< LT(I) \right>$ is the essential condition for $G$ is a Gr\"obner basis.
I personally call $G$ is "gr\"obner" if 
\begin{eqnarray}
\label{isGroebner}
\left< LT(g_1), \cdots, LT(g_t) \right> \supset \left< LT(I) \right>
\end{eqnarray}
holds, (indeed $\left< LT(g_1), \cdots, LT(g_t) \right> = \left< LT(I) \right>$ holds).
}

Informally, a set $G := \{g_1, \cdots, g_t\} \subset I$ of generators is a Gr\"ober basis iff the leading of all element of $I$ is divisible by one of the $LT(g_i)$.

% \subsubsection{Proof}

\subsection{Every nontrivial ideal has a Gr\"obner basis}
The proof of Hilbert Basis Theorem in \S\ref{HilbertBasisTheorem} also establishes the following result.

Fix a monomial order $>$.
All ideal $I \subset \mathbb{K}[x_1, \cdots, x_{n}]$ other than $\{0\}$ has a Gr\"obner basis.
Furthermore, any Gr\"obner basis for an ideal $I$ is a basis of $I$.

\subsubsection{Proof}
Given a nonzero ideal $I$, take
\begin{eqnarray}
G = \{g_1, \cdots, g_t\}
\end{eqnarray}
of its Hilbert Basis which generates $I = \left< g_1, \cdots, g_t \right> (\ref{HilbertGenerators})$ (for the second claim).

This Hilbert basis has the following property:
\begin{eqnarray}
\left< LT(I) \right> = \left< LT(g_1), \cdots, LT(g_t) \right>.
\end{eqnarray}
(see also \S\ref{monomialIdeals}).
This is indeed the condition for being a Gr\"obner basis.\\
$\blacksquare$

\subsection{The Ascending Chain Condition}
\label{ACC}
Let
\begin{eqnarray}
I_0 \subset I_1 \subset I_2 \subset I_3 \subset \cdots.
\end{eqnarray}
be an ascending chain of ideals in $\mathbb{K}[x_1, \cdots, x_n]$.
Then, there exists an $m \leq 1$ s.t.
\begin{eqnarray}
I_{m} = I_{m+1} = I_{m+2} \cdots.
\end{eqnarray}
That is, the ascending chain will have stabilized after finite steps $m$.

\subsubsection{Proof}
Given an ascending chain of ideal, let us define a set
\begin{eqnarray}
\label{BigI}
I := \bigcup_{j\geq 0}^\infty I_j,
\end{eqnarray}
and we shall show that $I$ is an ideal.\footnote{ See the definition in \S\ref{DefOfIdeal}.}
First, since $0 \in \forall I_j$ and $0\in I$.
Next, if $f,g \in I$, then we can put 
\begin{eqnarray}
f \in I_j, g \in I_k, j \leq k,
\end{eqnarray}
without loss of generality.
We have assumed that the chain is ascending, so $I_j \subset I_k$ and
\begin{eqnarray}
f,g \in I_k \Rightarrow f+g \in I_k
\end{eqnarray}
since $I_k$ is an ideal, and we get
\begin{eqnarray}
f+g \in I.
\end{eqnarray}
Similarly, $\forall f \in I$, there is $I_j$ s.t.
\begin{eqnarray}
f \in I_j \Rightarrow \forall h \in \mathbb{K}[x_1, \cdots, x_n], h*f \in I_j,
\end{eqnarray}
hence $\forall h \in \mathbb{K}[x_1, \cdots, x_n]$, 
\begin{eqnarray}
h*f \in I.
\end{eqnarray}
Therefore $I$ is an ideal.

By the Hilbert Basis Theorem in \S\ref{HilbertBasisTheorem}, for this ideal, there is a finite generator:
\begin{eqnarray}
I = \left<f_1, \cdots, f_s \right>,
\end{eqnarray}
and each generator is in some ideal in the ascending chain:
\begin{eqnarray}
f_i \in I_{j_i}, 1 \leq \forall i \leq s, j_i \geq 1
\end{eqnarray}
We can take 
\begin{eqnarray}
m := \left. \max j_i \right|_i
\end{eqnarray}
for $1 \leq i \leq s$, and then
\begin{eqnarray}
I = \left<f_1, \cdots, f_s \right> \subset I_m \subset I_{m+1} \subset \cdots \subset I.
\end{eqnarray}
As a result, the ascending chain stabilizes with $I_m$ and 
\begin{eqnarray}
I = \left<f_1, \cdots, f_s \right> \subset I_m = I_{m+1} = \cdots = I.
\end{eqnarray}
$\blacksquare$

\subsubsection{Note}
This "every ascending chain of ideals in $\mathbb{K}[x_1, \cdots, x_{n}]$ stabilizes in finite steps" is often called the ascending chain condition (ACC).

We have used the Hilbert basis in the proof of ACC, but ACC is, actually, equivalent to the Hilbert Basis Theorem.
In \S\ref{MoreACC}, we will treat ACC more precisely.

Here we prove Hilbert Basis Theorem without using Hilbert basis; if $I \subset \mathbb{K}[x_1, \cdots, x_{n}]$ is NOT finitely generated, then we can select an infinite generating sequence s.t., $I_i := \left<f_1, \cdots, f_i \right>$.
This is an ascending chain of ideals $I_1 \subset I_2 \subset \cdots$ which does NOT stabilize, but it contradicts our ACC.\\
$\blacksquare$


\subsection{Definition of the affine variety of an ideal}
Let $I \subset \mathbb{K}[x_1, \cdots, x_n]$ be an ideal.
The affine variety $\mathbb{V}(I)$ of the ideal $I$ is defined by
\begin{eqnarray}
\mathbb{V}(I) := \left\{ \left. a \in \mathbb{K}^n \right| \forall f \in I, f(a) = 0\right\}.
\end{eqnarray}

\subsection{Varieties of ideals is well-defined}
Let $\mathbb{V}(I)$ be an affine variety of an ideal $I$.
Then, there is a finite generating set and
\begin{eqnarray}
\mathbb{V}(I) = \mathbb{V}(f_1, \cdots, f_s)
\end{eqnarray}
holds.\footnote{The left hand side is defined in \S\ref{DefOfAV}.}
So, the varieties of ideals are well defined.

\subsubsection{Proof}
By the Hilbert Basis Theorem \S\ref{HilbertBasisTheorem}, we have a finite generator for the ideal $I$:
\begin{eqnarray}
I = \left< f_1, \cdots, f_s \right>.
\end{eqnarray}

$\forall a \in \mathbb{V}(I)$, by definition
\begin{eqnarray}
\forall f \in I, f(a) = 0,
\end{eqnarray}
and since $I = \left< f_1, \cdots, f_s \right>$, we get
\begin{eqnarray}
f_1(a) = \cdots = f_s(a) = 0.
\end{eqnarray}
Thus $a \in \mathbb{V}(f_1, \cdots, f_s)$ and
\begin{eqnarray}
\mathbb{V}(I) \subset \mathbb{V}(f_1, \cdots, f_s)
\end{eqnarray}

Conversely, $\forall a \in \mathbb{V}(f_1, \cdots, f_s)$, then by definition,
\begin{eqnarray}
f_1(a) = \cdots = f_s(a) = 0.
\end{eqnarray}
Since $I = \left< f_1, \cdots, f_s \right>$, we can write
\begin{eqnarray}
\forall f \in I, f = \sum_i h_i * f_i
\end{eqnarray}
and 
\begin{eqnarray}
\forall f \in I, f(a) = \sum_i h_i(a) * f_i(a) = \sum_i h_i(a) * 0 = 0.
\end{eqnarray}
This means that $a \in \mathbb{V}(I)$ and
\begin{eqnarray}
\mathbb{V}(I) \supset \mathbb{V}(f_1, \cdots, f_s)
\end{eqnarray}

Therefore we have
\begin{eqnarray}
\mathbb{V}(I) = \mathbb{V}(f_1, \cdots, f_s)
\end{eqnarray}
$\blacksquare$

\section{Properties of Gr\"obner Bases}
From this section, we will omit $*$ symbol to indicate the multiplication.

\subsection{Unique reminder properties}
\label{UniqueReminderProperties}
Let
\begin{eqnarray}
G := \{g_1, \cdots, g_t\} \subset I
\end{eqnarray}
be a Gr\"obner basis\footnote{This inclusion is for underlying sets, so $G$ is just a set.} for an ideal $I \subset \mathbb{K}[x_1, \cdots, x_n]$.
Then $\forall f \in \mathbb{K}[x_1, \cdots, x_n]$, there is a unique reminder $r \in \mathbb{K}[x_1, \cdots, x_n]$ with the following properies:
\begin{enumerate}
\item 
\label{property1}
No term of $r$ is divisible by any of $LT(g_1), \cdots, LT(g_t)$:
\begin{eqnarray}
r \sqsubset LT(g_1), \cdots, r \sqsubset LT(g_t).
\end{eqnarray}

\item 
\label{property2}
There is $g \in I$ s.t.
\begin{eqnarray}
f = g + r.
\end{eqnarray}
In particular, $r$ is "the" unique reminder on division of $f$ by $G$.\footnote{Here $G$ is just a set, not an ordered set, since this unique reminder does not depend on the order of $g_i$'s.}
\end{enumerate}

\subsubsection{Proof}
The division algorithm in \S\ref{DivisionAlgorithmInN} gives us
\begin{eqnarray}
f = a_1 g_1 + \cdots + a_t g_t + r,
\end{eqnarray}
where $r$ satisfies \ref{property1}st condition.
By putting
\begin{eqnarray}
g := a_1 g_1 + \cdots + a_t g_t,
\end{eqnarray}
we can also satisfy \ref{property2}nd condition.

To prove the uniqueness, let us suppose
\begin{eqnarray}
g + r = f = g' + r'
\end{eqnarray}
satisfy both properties.
Then we have
\begin{eqnarray}
g - g' = r' - r \in I.
\end{eqnarray}
If $r \neq r'$, then the leading term is in $\left< LT(I) \right>$:
\begin{eqnarray}
LT(r' - r) \in \left< LT(I) \right> = \left< LT(g_1), \cdots, LT(g_t) \right>.
\end{eqnarray}
Here we have assumed $G$ be a Gr\"obner basis (see the definition \S\ref{DefOfGroebner}), and this leads equality.
That is, the monomial $LT(r' - r)$ is in the monomial ideal $\left< LT(g_1), \cdots, LT(g_t) \right>$.
By \S\ref{Lemma2OfSection2_3}, it follows that
\begin{eqnarray}
1 \leq \exists i \leq t \text{ s.t. } LT(r' - r ) \sqsupseteq LT(g_i)
\end{eqnarray}
holds.\footnote{Pronounce it as "$LT(r' - r )$ is divisible by some $LT(g_i)$".}
This, however, contradicts our division algorithm; no term in $r, r'$ is divisible by $LT(g_1), \cdots, LT(g_t)$.
Therefore we have a unique reminder
\begin{eqnarray}
r' = r
\end{eqnarray}
and $g - g' = r' - r = 0$ implies
\begin{eqnarray}
g' = g.
\end{eqnarray}
$\blacksquare$

\subsubsection{Note}
This unique reminder properties is sometimes taken as the definition of a Gr\"ober basis.\footnote{We will summarize several equivalent statements for being Gr\"obner basis in \S\ref{severalIsGroebner}.}

\subsection{Reminder and normal forms}
\label{normalForms}
The remainder $r$ is called the normal form of $f$ with respect to the Gr\"obner basis $G := \{g_1, \cdots, g_t\} $.

We will write
\begin{eqnarray}
\bar{f}^F
\end{eqnarray}
as the remainder on division of $f$ by the ordered tuple \footnote{The remainder is also unique for the ordered tuple $F$.}
\begin{eqnarray}
F := (f_1, \cdots, f_s).
\end{eqnarray}
If $F$ is a Gr\"obner basis (for $f_1, \cdots, f_s$), then we can regard $F$ as merely a set, since we have shown the uniqueness of the reminder in \S\ref{UniqueReminderProperties}.

Sometimes, we also write
\begin{eqnarray}
f \stackrel{F}{\to} r %:= \bar{f}^F
\end{eqnarray}
as a division process, $r$ is the reminder of $f$ by the tuple $F$.
Using this notation, the property in \S\ref{UniqueReminderProperties} becomes the following statement; if $G$ is gr\"obner, then $\forall f \in \mathbb{K}[x_1, \cdots, x_n]$,
\begin{eqnarray}
\exists! r \in \mathbb{K}[x_1, \cdots, x_n] \text{ s.t. } f \stackrel{G}{\to} r.
\end{eqnarray}

\subsection{An ideal membership condition}
\label{IMC}
Let $G := \{g_1, \cdots, g_t\} $ be a Gr\"obner basis for an ideal $I \subset \mathbb{K}[x_1, \cdots, x_n]$.
Then $f \in I$ iff the reminder on division of $f$ by $G$ is zero.

\subsubsection{Proof}
Now we have
\begin{eqnarray}
I = \left< g_1, \cdots, g_t\right>
\end{eqnarray}
of Gr\"obner basis for $I$.
In \S\ref{membership} we have already proved $\Leftarrow$ part:
\begin{eqnarray}
f \in I \Leftarrow f = a_1 g_1 + \cdots + a_t g_t.
\end{eqnarray}
Conversely, given $f \in I$, then
\begin{eqnarray}
f = f+0
\end{eqnarray}
satisfies the two conditions in \S\ref{UniqueReminderProperties}.
The uniqueness implies that the reminder is zero, therefore
\begin{eqnarray}
f \in I \Rightarrow f = a_1 g_1 + \cdots + a_t g_t.
\end{eqnarray}
$\blacksquare$

\subsubsection{Note}
This property is also sometimes taken as the definition of a Gr\"obner basis, since we can show \footnote{We will complete this statement in \S\ref{isGroebner2}.
Here in \S\ref{IMC} we have proved $\Rightarrow$ part.
} that it is true iff $G$ is "gr\"obner", i.e.,
\begin{eqnarray}
\left< LT(g_1), \cdots, LT(g_t) \right> = \left< LT(I)\right>
\end{eqnarray}
holds.\footnote{See the definition of Gr\"obner basis in \S\ref{DefOfGroebner}.}

\subsection{Definition of S-polynomials}
Consider nonzero polynomials $f,g \in \mathbb{K}[x_1, \cdots, x_n]$, and multi degrees
\begin{eqnarray}
\alpha := MD(f), \beta := MD(g),
\end{eqnarray}
then clearly,
\begin{eqnarray}
x^\alpha = LM(f), x^\beta = LM(g).
\end{eqnarray}
Define a multi index
\begin{eqnarray}
\gamma = (\gamma_1, \cdots, \gamma_n), \gamma_i := \max(\alpha_i, \beta_i),
\end{eqnarray}
and call
\begin{eqnarray}
x^\gamma 
\end{eqnarray}
the least common multiple of leading monomials $LM(f), LM(g)$:
\begin{eqnarray}
x^\gamma := LCM\left( LM(f), LM(g) \right).
\end{eqnarray}

Then we can define the S-polynomial of $f$ and $g$:
\begin{eqnarray}
S(f,g) := \frac{x^\gamma}{LT(f)} f - \frac{x^\gamma}{LT(g)} g 
\end{eqnarray}
As we will see, S-polynomial is designed to cancel the leading terms, moreover, all cancellation of leading terms among polynomials of the same multi degree result from the combination of S-polynomials.

\subsection{S-polynomial of $I$ is in $I$}
\label{SofIisInI}
We will show the following property of S-polynomials for later use.
\begin{eqnarray}
\forall f,g \in I, S(f,g)\in I.
\end{eqnarray}

\subsubsection{Proof}
Since $LG(f), LG(g) \sqsubseteq x^\gamma$ by definition of $\gamma$, and
\begin{eqnarray}
\frac{x^\gamma}{LT(f)}, \frac{x^\gamma}{LT(g)} \in \mathbb{K}[x_1, \cdots, x_n],
\end{eqnarray}
we get
\begin{eqnarray}
S(f, g) = \frac{x^\gamma}{LT(f)} f - \frac{x^\gamma}{LT(g)} g \in I.
\end{eqnarray}
Therefore, S-polynomial of $I$ is in $I$.\footnote{This if the sum of elements in $I$ with $\mathbb{K}[x_1, \cdots, x_n]$ "coefficients" form.}\\
$\blacksquare$

\subsection{Lemma}
\label{buchbergerLemma}
Suppose we have a sum
\begin{eqnarray}
\sum_{i=1}^s c_i f_i, c_i \in \mathbb{K}, 
\end{eqnarray}
where $\forall i$,
\begin{eqnarray*}
MD(f_i) = \delta \in \mathbb{N}^n.
\end{eqnarray*}
If the multi degree of this sum is strictly smaller than $\delta$\footnote{The cancellation of leading terms do occur.}
\begin{eqnarray}
\label{strictlySmallerMD}
MD\left( \sum_{i=1}^s c_i f_i \right) < \delta,
\end{eqnarray}
then this sum $\sum_{i=1}^s c_i f_i$ is a linear combination, with $\mathbb{K}$ coefficients, of the S-polynomials
\begin{eqnarray}
S(f_j, f_k), 1 \leq j, k \leq s.
\end{eqnarray}
Furthermore, each $S(f_j, f_k)$,
\begin{eqnarray}
MD\left( S(f_j, f_k) \right) < \delta.
\end{eqnarray}

\subsubsection{Proof}
Let $d_i = LC(f_i)$, so that
\begin{eqnarray}
c_i d_i = LC(c_i f_i).
\end{eqnarray}
Since all $c_i * f_i$ have multi degree $\delta$ and their sum has strictly smaller multi degree (eq.(\ref{strictlySmallerMD})).
It follows that
\begin{eqnarray}
\sum_i c_i d_i = 0,
\end{eqnarray}
i.e., the leading coefficient is cancelled out.

Define polynomials which has 1 as the leading coefficient:\footnote{Here we used the "small o notation" $o(x^\delta)$ to indicate the terms that have smaller multi degree than $x^\delta$.}
\begin{eqnarray}
p_i := f_i/d_i = x^\delta + o(x^\delta).
\end{eqnarray}
Consider the telescoping sum\footnote{
From Wikipedia : \begin{quote}
...whose partial sums eventually only have a fixed number of terms after cancellation.
\end{quote}
}
\begin{eqnarray}
\sum_{i=1}^s c_i f_i &=& \sum_{i=1}^s c_i d_i p_i \\
\nonumber
&=& c_1 d_1 (p_1 -p_2) + (c_1d_1 + c_2 d_2)(p_2 -p_3) + \cdots \\
\nonumber
&&+ (c_1d_1 + \cdots + c_{s-1} d_{s-1})(p_{s-1} - p_s) \\
\nonumber
&&+ (c_1d_1 + \cdots + c_s d_s) p_s \\
\nonumber
&=& c_1 d_1 (p_1 -p_2) + (c_1d_1 + c_2 d_2)(p_2 -p_3) + \cdots \\
&&+ (c_1d_1 + \cdots + c_{s-1} d_{s-1})(p_{s-1} - p_s),
\end{eqnarray}
where we have used $\sum_i c_i * d_i = 0$.
From our assumption, we have, $\forall i$,
\begin{eqnarray}
LT(f_i) = d_i x^\delta,
\end{eqnarray}
which implies that the least common multiple of $LT(f_j)$ and $LT(f_k)$ is $x^\delta$.
Thus
\begin{eqnarray}
S(f_j, f_k) &=& \frac{x^\delta}{LT(f_j)}f_j- \frac{x^\delta}{LT(f_k)}f_k \\
&=& \frac{x^\delta}{d_j x^\delta}f_j- \frac{x^\delta}{d_k x^\delta}f_k \\
&=& p_j - p_k.
\end{eqnarray}
Here, the leading terms of $p_j, p_k$ are cancelled out and
\begin{eqnarray}
MD\left( S(f_j, f_k) \right) = MD(p_j -p_k) < \delta.
\end{eqnarray}

Now the above telescoping sum becomes
\begin{eqnarray}
\nonumber
\sum_{i=1}^s c_i f_i &=& c_1 d_1 S(f_1, f_2) + (c_1d_1 + c_2 d_2)S(f_2, f_3) + \cdots\\
&& + (c_1d_1 + \cdots + c_{s-1}d_{s-1}) S(f_{s-1}, f_s) \\
&=:& \sum_{j,k} c_{j,k} S(f_j, f_k).
\end{eqnarray}
This \footnote{The last equality can be seen as the definition of $\mathbb{K}$ coefficients $c_{j,k}$'s.} is clearly a linear combination of the S-polynomials of $\mathbb{K}$ coefficients.\\
$\blacksquare$

\subsection{Buchberger's Criterion}
\label{Buchberger'sCriterion}
Let $I$ be a polynomial ideal.
Then an ordered tuple of basis $G = (g_1, \cdots, g_t)$ for $I$ is a Gr\"obner basis for $I$ iff for all pairs $j \neq k$, the remainder on division $S(g_j, g_k)$ by $G$ is zero:\footnote{It seems like a Cauchy sequence.} 
\begin{eqnarray}
\text{$G$ is gr\"obner} \Leftrightarrow \forall j\neq k, S(g_j, g_k) \stackrel{G}{\to} 0.
\end{eqnarray}

\subsubsection{($\Rightarrow$) part} 
From \S\ref{SofIisInI},
\begin{eqnarray}
S(g_j, g_k) = \frac{x^\gamma}{LT(g_j)} g_k - \frac{x^\gamma}{LT(g_k)} g_k \in I.
\end{eqnarray}
If $G$ is a Gr\"obner basis for $I$, the remainder on division by $G$ is 0 by \S\ref{IMC}.

\subsubsection{($\Leftarrow$) part} 
It suffices to show that if $ \forall j\neq k, S(g_j, g_k) \stackrel{G}{\to} 0$, then $\forall f \in I$, $LT(f) \in \left< LT(g_1), \cdots, LT(g_t) \right>$, i.e., $\left< LT(I) \right> \subset \left< LT(g_1), \cdots, LT(g_t) \right>$.\footnote{See the definition of Gr\"obner basis in \S\ref{DefOfGroebner}, or eq.(\ref{isGroebner}).
}
We have assumed that $G$ is a generator of $I$, so $\forall f \in I, \exists h_i$, s.t.,
\begin{eqnarray}
f = \sum_i h_i g_i.
\end{eqnarray}

\paragraph{Construction of $\delta_0$}
Let us write
\begin{eqnarray}
m(i) &:=& MD(h_i g_i) \\
\delta &:=& \left.\max\left(m(i) \right) \right|_i
\end{eqnarray}
then clearly
\begin{eqnarray}
\label{MDfleqDelta}
MD(f) \leq \delta.
\end{eqnarray}
If $MD(f) < \delta$ then some cancellations of the leading terms must occur.

There is at least one combination of $h_i$'s s.t. $f = \sum_i h_i g_i$, we can define the following non empty set:
\begin{eqnarray}
\left\{ \delta = \delta\left(\{h_i \right\}) \left| \forall \{h_i\} \text{ s.t. } f = \sum_i h_i g_i \right. \right\} \subset \mathbb{N}^n.
\end{eqnarray}
Since our monomial order is well-ordering,\footnote{See eq.(\ref{multiDegree}), multi degrees are essentially monomials.} we can pick up the minimal element, let's call it $\delta_0$.

\paragraph{Lemma $(MD(f) = \delta_0)$}
Since we still have $MD(f) \leq \delta_0$, we can decompose $f = \sum_i h_i g_i$ as
\begin{eqnarray}
\nonumber
f &=& \sum_i h_i g_i \\
\nonumber
&=& \sum_{m(i) = \delta_0} h_i g_i + \sum_{m(i) < \delta_0} h_i g_i \\
&=& \sum_{m(i) = \delta_0} LT(h_i) g_i + \sum_{m(i) = \delta_0} \left(h_i - LT(h_i) \right) g_i + \sum_{m(i) < \delta_0} h_i g_i \qquad
\end{eqnarray}
Both 2nd and 3rd terms clearly have strictly smaller multi degrees, e.g.,
\begin{eqnarray}
MD\left( h_i - LT(h_i) \right) < \delta_0.
\end{eqnarray}
Therefore, if $MD(f) < \delta_0$, then we have
\begin{eqnarray}
\label{keyUpperBound}
MD\left( \sum_{m(i) = \delta_0} LT(h_i) g_i \right) < \delta_0
\end{eqnarray}
but we will see that $MD(f) < \delta_0$ contradicts our minimal assumption of $\delta_0$.

Assuming $MD(f) < \delta_0$, all the terms in the 1st sum $\sum_{m(i) = \delta_0} LT(h_i) g_i $ have the same multi degree, since the summation is under $m(i) = \delta_0$:\footnote{The leading term carries the maximum monomial, so $m(i) = MD(h_i g_i) = MD\left( LT(h_i) g_i  \right)$.}
\begin{eqnarray}
MD\left( LT(h_i) g_i  \right) = \delta_0
\end{eqnarray}
This is the form of \S\ref{buchbergerLemma} with $c_i =1, f_i = LT(h_i) g_i$, and the lemma implies
\begin{eqnarray}
\sum_{m(i) = \delta_0} LT(h_i) g_i = \sum_{j,k}c_{j,k} S\left(LT(h_j) g_j, LT(h_k) g_k \right).
\end{eqnarray}
By definition, since the least common multiple of $LT(h_i) g_i$'s is $\delta_0$, 
\begin{eqnarray}
S\left(LT(h_j) g_j, LT(h_k) g_k \right) &=& \frac{x^{\delta_0}}{LT\left(LT(h_j) g_j \right)}LT(h_j) g_j - (j\to k) \qquad \\
&=& \frac{x^{\delta_0}}{LT(h_j) LT\left( g_j \right)}LT(h_j) g_j - (j\to k) \\
&=& \frac{x^{\delta_0}}{LT\left( g_j \right)} g_j - (j\to k) \\
&=& x^{\delta_0 - \gamma_{j,k}} S(g_j, g_k),
\end{eqnarray}
where
\begin{eqnarray}
\gamma_{j,k} := LCM\left(LM(g_j), LM(g_k) \right).
\end{eqnarray}
The 1st sum becomes
\begin{eqnarray}
\label{semiFinalResult}
\sum_{m(i) = \delta_0} LT(h_i) g_i = \sum_{j,k}c_{j,k} x^{\delta_0 - \gamma_{j,k}} S(g_j, g_k).
\end{eqnarray}
%and
%\begin{eqnarray}
%f = \sum_{j,k}c_{j,k} x^{\delta_0 - \gamma_{j,k}} S(g_j, g_k) + \sum_{m(i) = \delta_0} \left(h_i - LT(h_i) \right) g_i + \sum_{m(i) < \delta_0} h_i g_i. \qquad
%\end{eqnarray}

Now we use our hypothesis $S(g_j, g_k) \stackrel{G}{\to} 0$, i.e., the division algorithm implies that there exists $a_{i,j,k} \in \mathbb{K}[x_1, \cdots, x_n]$ s.t.
\begin{eqnarray}
S(g_j, g_k) = \sum_{i} a_{i,j,k} g_i
\end{eqnarray}
By eq.(\ref{decreasingMultiDegree}), we have
\begin{eqnarray}
\label{MDSgeqAG}
MD\left(S(g_j, g_k) \right) \geq MD( a_{i,j,k} g_i).
\end{eqnarray}
Therefore, the right hand side of eq.(\ref{semiFinalResult}) becomes
\begin{eqnarray}
c_{j,k} x^{\delta_0 - \gamma_{j,k}} \sum_{i} a_{i,j,k} g_i = \sum_i c_{j,k} (x^{\delta_0 - \gamma_{j,k}} a_{i,j,k}) g_i
= \sum_{i} c_{j,k} b_{i,j,k} g_i,
\end{eqnarray}
where
\begin{eqnarray}
b_{i,j,k} := x^{\delta_0 - \gamma_{j,k}} a_{i,j,k}.
\end{eqnarray}
Similarly, by eq.(\ref{decreasingMultiDegree}), we have\footnote{See eq.(\ref{keyUpperBound}).}
\begin{eqnarray}
\label{keyContradiction}
\delta_0 > MD \left( x^{\delta_0 - \gamma_{j,k}} S(g_j, g_k) \right) \geq MD( b_{i,j,k} g_i)
\end{eqnarray}
for all possible combinations.

Finally, eq.(\ref{semiFinalResult}) becomes
\begin{eqnarray}
\nonumber
\sum_{m(i) = \delta_0} LT(h_i) g_i &=& \sum_{j,k}c_{j,k} x^{\delta_0 - \gamma_{j,k}} S(g_j, g_k) \\
\nonumber
&=& \sum_{j,k} \sum_{i} c_{j,k} b_{i,j,k} g_i \\
&=& \sum_{i} \left( \sum_{j,k} b_{i,j,k} c_{j,k} \right) g_i \\
\label{tildeH}
&=& \sum_{i} \tilde{h}_i g_i 
\end{eqnarray}
and eq.(\ref{keyContradiction}) implies
\begin{eqnarray}
\label{keyResult}
MD(\tilde{h}_i g_i) < \delta_0, \forall i.
\end{eqnarray}

Now, by eq.(\ref{tildeH}),
\begin{eqnarray}
\nonumber
f &=& \sum_{m(i) = \delta_0} LT(h_i) g_i + \sum_{m(i) = \delta_0} \left(h_i - LT(h_i) \right) g_i + \sum_{m(i) < \delta_0} h_i g_i \\
&=& \sum_{i} \tilde{h}_i g_i  + \sum_{m(i) = \delta_0} \left(h_i - LT(h_i) \right) g_i + \sum_{m(i) < \delta_0} h_i g_i.
\end{eqnarray}
We have shown that if we suppose $MD(f) < \delta_0$, then we get eq.(\ref{keyResult}), and this shows all three sums have strictly smaller multi degrees than $\delta_0$.
This, however, contradicts the minimal assumption of $\delta_0$, therefore we have\footnote{Actually, we show $MD(f) \nless \delta_0$, but we already have eq.(\ref{MDfleqDelta}), thus we get this equality.}
\begin{eqnarray}
\label{equality}
MD(f) = \delta_0.
\end{eqnarray}

\paragraph{$LT(f)$ is divisible by $\left.LT(g_j)\right|_j$}
Since we have shown $MD(f) = \delta_0$ (eq.(\ref{equality})), then
\begin{eqnarray}
MD(f) = \delta_0 = \left. \max\left(MD(h_i g_i) \right) \right|_i \geq MD(g_j), \forall j.
\end{eqnarray}
This tells us that the leading term of $f$ is divisible by any generators
\begin{eqnarray}
LT(f) = LC(f) * x^{\delta_0} \sqsupseteq LT(g_j), \forall j.
\end{eqnarray}
and this is equivalent to
\begin{eqnarray}
LT(f) \in \left< LT(g_1), \cdots, LT(g_t) \right>.
\end{eqnarray}
Therefore, $G$ is gr\"obner.\\
$\blacksquare$

\subsection{Ideal membership condition, with S-polynomials}
\label{isGroebner2}
With this Buchberger's Criterion, we can show the the divisibility is the property of Gr\"obner basis as we stated in \S\ref{IMC}; if $G$ is a basis for $I$ with $\forall f \in I$,
\begin{eqnarray}
f \stackrel{G}{\to} 0,
\end{eqnarray}
then $G$ is gr\"obner.

\subsubsection{Proof}
Let $G = \{g_1, \cdots, g_t\}$, then it suffices to show $S(g_j, g_k) \stackrel{G}{\to} 0$.
By \S\ref{SofIisInI}
\begin{eqnarray}
S(g_j, g_k) = \frac{x^\gamma}{LT(g_j)}g_k - \frac{x^\gamma}{LT(g_k)}g_k \in I
\end{eqnarray}
where $\gamma = LCM \left( LM(g_j), LM(g_k)\right)$, and by our assumption,
\begin{eqnarray}
S(g_j, g_k) \stackrel{G}{\to} 0.
\end{eqnarray}
This is iff condition for gr\"obner $G$.\\
$\blacksquare$

\subsection{Several "isGr\"obner"'s}
\label{severalIsGroebner}
We have several conditions for gr\"obner property of $G = \{g_1, \cdots, g_t\}$ in a polynomial ideal $I$.
By definition, eq.(\ref{isGroebner}); if $G$ satisfies
\begin{eqnarray}
\nonumber
\left< LT(g_1), \cdots, LT(g_t) \right> \supset \left< LT(I) \right>,
\end{eqnarray}
then we call this $G$ gr\"obner.\footnote{As we saw, this inclusion provides
\begin{eqnarray}
\nonumber
\left< LT(g_1), \cdots, LT(g_t) \right> = \left< LT(I) \right>.
\end{eqnarray}
}
From Buchberger's Criterion \S\ref{Buchberger'sCriterion}:
\begin{eqnarray}
\nonumber
\text{$G$ is gr\"obner}\Leftrightarrow S(g_j, g_k) \stackrel{G}{\to} 0.
\end{eqnarray}
From \S\ref{IMC} and \S\ref{isGroebner2},
\begin{eqnarray}
\text{$G$ is gr\"obner}\Leftrightarrow \forall f \in I, f \stackrel{G}{\to} 0.
\end{eqnarray}

In addition, from \S\ref{UniqueReminderProperties}, $\forall f \in \mathbb{K}[x_1, \cdots, x_n]$, we have proved the sufficient condition\footnote{From our source book,
\begin{quote}
In fact, Groebner bases can be characterized by the uniqueness of the remainder -- see Theorem 5.35 of BECKER and WEISPFENNING(1993) for this and other conditions equivalent to being a Groebner basis.
\end{quote}
}
\begin{eqnarray}
\nonumber
\text{$G$ is gr\"obner}\Rightarrow \exists! r \in \mathbb{K}[x_1, \cdots, x_n] \text{ s.t. } f \stackrel{G}{\to} r.
\end{eqnarray}

\subsection{Unique remainder among different Gr\"obner bases}
%We can slightly generalize the statement in \S\ref{UniqueReminderProperties}.
%
%Fix a monomial ordering and let $I \subset \mathbb{K}[x_1, \cdots, x_n]$.
%\begin{enumerate}
%\item $\forall f \in I$ can be written in the form $f = g + r$, where $g \in I$ and no term of $r \in \mathbb{K}[x_1, \cdots, x_n]$ is divisible by any element of $LT(I)$.
%
%\item Given two expression $g + r = f = g' + r'$ satisfy above property.
%Then $r = r'$, i.e. $r$ is uniquely determined.
%
%\end{enumerate}
Let us fix a monomial order $<$.
Consider $\forall f \in I$ and two Gr\"obner bases $G, G'$ for $I$, then the remainders of $f$ by $G$ and that of $G'$ are the same.

\subsubsection{Proof}
From \S\ref{UniqueReminderProperties}, there exists $g, g' \in I$ s.t.
\begin{eqnarray}
g + r = f = g' + r',
\end{eqnarray}
where 
\begin{eqnarray}
f \stackrel{G}{\to} r, f \stackrel{G'}{\to} r',
\end{eqnarray}
and $r,r'$ satisfy eq.(\ref{remainderHasSmallerMultiDegree}), i.e., $r \sqsubset \left. LT(g_j) \right|_j$.
Let us suppose $r' - r \neq 0$.
Since the differences are in $I$
\begin{eqnarray}
g - g' = r' - r \in I,
\end{eqnarray}
their normal forms become 0 by \S\ref{isGroebner2}.
This contradicts the condition for the remainder (eq.(\ref{remainderHasSmallerMultiDegree})), they are not divisible by any leading terms of generators.
Therefore we have 
\begin{eqnarray}
r = r'.
\end{eqnarray}
$\blacksquare$

\section{Buchberger's Algorithm}
\S\ref{Buchberger'sCriterion} gives us the function of the following type
\begin{eqnarray}
\verb|Basis -> Bool|
\end{eqnarray}
So, we will find a constructive algorithm of Gr\"obner basis:
\begin{eqnarray}
\verb|Basis -> GroebnerBasis|
\end{eqnarray}

\subsection{Buchberger's Algorithm}
Let $I = \left< f_1, \cdots, f_s \right> \neq \{0\}$ be a polynomial ideal.
Then a Gr\"obner basis for $I$ can be constructed in a finite steps by the following algorithm:

\subsubsection{Pseudo code}
\begin{verbatim}
Input:  F = (f_1, .., f_s)
Output: G = (g_1, .., g_t)

G := F
REPEAT
  G' := G
  
  FOR each pair (p,q), p \= q in G' DO
    S(p,q) -> s
    IF s \= 0 THEN G := G \cup {s}
UNTIL G == G'
\end{verbatim}

That is, for arbitrary generator of an ideal $I = \left< f_1, \cdots, f_s \right>$, the out put is a Gr\"obner basis for $I$.

\subsubsection{Proof}
Let us define, for a generating set $G = \{g_1, \cdots, g_t\}$,
\begin{eqnarray}
\left< G \right> &:=& \left< g_1, \cdots, g_t \right> \\
\left<LT(G) \right> &:=& \left< LT(g_1), \cdots, LT(g_t) \right>
\end{eqnarray}
for later uses.
Using this notation, the generator $G$ is a Gr\"obner basis for $I$ iff $\left<LT(I)\right> = \left<LT(G) \right>$.

At every stage of the algorithm, we shall show $G' \subset I$, where 
\begin{eqnarray}
I = \left< f_1, \cdots, f_s \right>.
\end{eqnarray}
This is true at the first step:
\begin{eqnarray}
G \subset I.
\end{eqnarray}
In some intermediate step, assume $G' \subset I$.
Since $\forall p,q \in G'$,
\begin{eqnarray}
S(p,q) \in G'
\end{eqnarray}
we get
\begin{eqnarray}
S(p,q) \stackrel{G'}{\to} s \in G'
\end{eqnarray}
since $S(p,q) \stackrel{G'}{\to} s$ is equal to
\begin{eqnarray}
\exists g' \in G' \text{ s.t. } S(p,q) = g' + s,
\end{eqnarray}
therefore
\begin{eqnarray}
s = S(p,q) -g' \in G'.
\end{eqnarray}
If $s = 0$, then the next $G'$ is unchanged $G' \mapsto G'$, else $(s \neq 0)$, let us add $s$ into $G'$:
\begin{eqnarray}
G' \mapsto G' \cup \{ s \} \subset I.
\end{eqnarray}

When this algorithm terminates, we'll get $G=G'$.
This holds when
\begin{eqnarray}
\forall p, q \in G', S(p,q) \stackrel{G'}{\to} 0,
\end{eqnarray}
i.e., $G'$ is gr\"obner (see \S\ref{Buchberger'sCriterion}).

Finally, we claim that the algorithm terminates finitely.
After one step of main loop, if $G' \neq G$ then 
\begin{eqnarray}
G' \mapsto G'' := G' \cup \{s \}.
\end{eqnarray}
Since the reminder $s$ is strictly smaller than any generators of $G'$ $s \sqsubset LT(g'_i), g'_i \in G'$ (see eq.(\ref{remainderHasSmallerMultiDegree})),
\begin{eqnarray}
LT(s) \notin LT(G')
\end{eqnarray}
but by definition
\begin{eqnarray}
LT(s) \in LT(G'').
\end{eqnarray}
So
\begin{eqnarray}
\left<LT(G') \right> \subsetneq \left<LT(G'') \right>.
\end{eqnarray}
The ACC in \S\ref{ACC} implies that after a finite number of iterations, this chain will stabilize in finite steps, so that
\begin{eqnarray}
\left<LT(G) \right> \subset \cdots \subset \left<LT(G^{(m)}) \right> = \left<LT(G^{(m+1)}) \right>
\end{eqnarray}
happen eventually.
This occur when we meet $G^{(m+1)} = G^{(m)}$, therefore this algorithm will terminate finitely.\\
$\blacksquare$

\subsection{An elimination method}
Let $G$ be a Gr\"obner basis for the polynomial ideal $I$.
$\forall p \in G$ s.t. 
\begin{eqnarray}
LT(p) \in \left< LT(G - \{p\}) \right>,
\end{eqnarray}
then the set $G - \{p\}$ is also a Gr\"obner basis.

\subsubsection{Proof}
By definition, this Gr\"obner basis $G$ satisfies
\begin{eqnarray}
\left<LT(I)\right> = \left<LT(G) \right>.
\end{eqnarray}
We can write
\begin{eqnarray}
G &:=& \{g_1, \cdots, g_t, p\} \\
G - \{p\} &:=& \{g_1, \cdots, g_t\}
\end{eqnarray}

If $q \in \left< LT(G - \{p\}) \right>$, then we have $q \in \left< LT(G) \right>$ automatically, and
\begin{eqnarray}
\left< LT(G - \{p\}) \right> \subset \left<LT(G) \right>.
\end{eqnarray}
Conversely, if $LT(p) \in \left< LT(G - \{p\}) \right>$, $\forall q \in \left<LT(G) \right>$ we have an expression
\begin{eqnarray}
q = \sum_i h_i * LT(g_i) + h * LT(p), 
\end{eqnarray}
where $h,h_i \in \mathbb{K}[x_1, \cdots, x_n]$.
But since $LT(p) \in \left< LT(G - \{p\}) \right>$, we can decompose $LT(p)$ in $G - \{p\}$ and
\begin{eqnarray}
q = \sum_i h'_i * LT(g_i),
\end{eqnarray}
and this implies $q \in \left< LT(G - \{p\}) \right>$:
\begin{eqnarray}
\left< LT(G - \{p\}) \right> \supset \left<LT(G) \right>
\end{eqnarray}

Therefore, we have
\begin{eqnarray}
\left< LT(G - \{p\}) \right> = \left<LT(G) \right>
\end{eqnarray}
and
\begin{eqnarray}
\left<LT(I)\right> = \left< LT(G - \{p\}) \right>.
\end{eqnarray}
$\blacksquare$

\subsection{Definition of minimal Gr\"obner bases}
A minimal Gr\"obner basis for a polynomial ideal $I$ is a Gr\"obner basis $G$ for $I$ s.t. the leading coefficients are normalized; $\forall p \in G$,
\begin{eqnarray}
LC(p) = 1 
\end{eqnarray}
and
\begin{eqnarray}
\label{minimalCondition}
LT(p) \notin \left< LT(G - \{p\}) \right>.
\end{eqnarray}

By adjusting the leading coefficients 1 and removing $LT(p) \in \left< LT(G - \{p\}) \right>$ from an arbitrary Gr\"obner basis, we will arrive at this minimal Gr\"obner basis.

\subsection{Definition of reduced Gr\"obner bases}
A reduced Gr\"obner basis for a polynomial ideal $I$ is a Gr\"obner basis $G$ for $I$ s.t. $\forall p \in G$,
\begin{eqnarray}
LC(p) = 1
\end{eqnarray}
and
\begin{eqnarray}
\text{no monomial of $p$ lies in} \left< LT(G - \{p\}) \right>.
\end{eqnarray}

In general, $g \in G$ is reduced for a Gr\"obner basis $G$ iff no monomial of $g$ is in $\left< LT(G - \{g\}) \right>$.

\subsection{A unique reduced Gr\"obner basis}
\label{theUniquenessOfReducedGB}
In general, reduced Gr\"obner bases have the following nice property.

Let $I \neq \{0 \}$ be a polynomial ideal.
Then, for a given monomial order, $I$ has a unique reduced Gr\"obner basis.

\subsubsection{Proof}
Let $G$ be a minimal Gr\"obner basis for $I$.
Our goal is to modify $G$ until all of elements are reduced.

If $g \in G$ is reduced for $G$, then $g$ is also reduced for any other minimal Gr\"obner basis $G'$ of $I$ that contains $g$ and has the same set of leading terms, 
\begin{eqnarray}
LT(G) = LT(G'),
\end{eqnarray}
since the process of reducing only sees the leading terms $\left< LT(G - \{g\}) \right>$.

Next, given $g \in G$, let $g'$ be the remainder by $G - \{g\}$
\begin{eqnarray}
g \stackrel{G - \{g\}}{\to} g'
\end{eqnarray}
and let
\begin{eqnarray}
G' := (G-\{g\}) \cup \{g'\}
\end{eqnarray}
be a new set.\footnote{This new set can be seen as the same $G$ with $g'$ replaced by $g$.}
We claim that this new $G'$ is also a minimal Gr\"obner basis for $I$.

First,
\begin{eqnarray}
LT(g') = LT(g)
\end{eqnarray}
since when we divide $g$ by $G-\{g\}$, $LT(g)$ goes to the remainder because $LT(g)$ is not divisible by any element of $\left< LT(G - \{g\}) \right>$.\footnote{We have assumed $G$ is a minimal Gr\"obner basis for $I$, so $LT(g) \notin \left< LT(G - \{g\}) \right> $ eq.(\ref{minimalCondition}).}
This shows that the generators and the monomial ideals are the same:
\begin{eqnarray}
LT(G) &=& LT(G') \\
\left< LT(G) \right> &=& \left< LT(G') \right>.
\end{eqnarray}
Since $G$ is a Gr\"obner basis for $I$, $G$ satisfies $\left<LT(I)\right> = \left<LT(G) \right>$ and so
\begin{eqnarray}
\left<LT(I)\right> = \left<LT(G') \right>,
\end{eqnarray}
i.e., $G'$ is also a Gr\"obner basis, and is minimum.
Note that $g' \in G'$ is reduced for $G'$ by construction.
 
Now, reduce all the element in $G$ by using above process until they are all reduced.
Since once an element is reduced, it stays reduced even if $G$ changes due to other reducing processes.
Thus, we end up with a reduced Gr\"obner basis.

Finally, we'll prove the uniqueness.
Suppose $G, \tilde{G}$ be reduced Gr\"obner bases for $I$.
Under reducing process $G \mapsto G' := (G-\{g\}) \cup \{g' \}$, the leading terms of generators are unchanged:
\begin{eqnarray}
LT(G) = LT(\tilde{G}),
\end{eqnarray}
i.e. $\forall g \in G$, 
\begin{eqnarray}
\exists \tilde{g} \in \tilde{G} \text{ s.t. } LT(g) = LT(\tilde{g}).
\end{eqnarray}
Consider the difference $g - \tilde{g}$.
Since this is in $I$,
\begin{eqnarray}
g -\tilde{g} \stackrel{G}{\to} 0.
\end{eqnarray}
But we also know $LT(g) = LT(\tilde{g})$, the leading terms cancel in the difference.
Since $G,\tilde{G}$ are reduced, we have
\begin{eqnarray}
g -\tilde{g} \stackrel{G}{\to} g -\tilde{g}
\end{eqnarray}
but this is $0$.
Thus we get $g = g'$ and 
\begin{eqnarray}
G = \tilde{G},
\end{eqnarray}
since we have shown that $\forall g \in G, \exists! g \in \tilde{G}$.\\
$\blacksquare$










\subsubsection{Note}
A consequence of the uniqueness of the reduced Gr\"obner basis in \S\ref{theUniquenessOfReducedGB} is we have an ideal equality algorithm.

Consider two sets of generators
\begin{eqnarray}
\{f_1, \cdots, f_s\}, \{g_1, \cdots, g_t\}
\end{eqnarray}
and the ideals which are generated by $f$'s and $g$'s.
Compute the reduced Gr\"obner bases for $f$'s and $g$'s.
Then the ideals are equal iff the Gr\"obner bases are the same.\footnote{These Gr\"obner bases are merely sets, so we can compare them simply element by element.}

%\section{First Applications of Gr\"obner bases}
%\subsection{? Ideal Membership Problem}
%From  \ref{isGroebner2}, given ideal $I$ and its Gr\"obner basis $G$,
%\begin{eqnarray}
%f \in I \Leftrightarrow f \stackrel{G}{\to} 0.
%\end{eqnarray}
%
%\subsection{? The Problem of Solving Polynomial Equations}
%\subsection{? The Implicitization Problem}

\section{Improvements on Buchberger's Algorithm}
The most heaviest part of computation in Buchberger's Algorithm is the polynomial divisions.
Hence, a good way to improve the efficiency of the algorithm would be to reduce the number of S-polynomials need to be considered.

\subsection{Definition of $\to_G$}
Fix a monomial order and let $G = \{g_1, \cdots, g_s\} \subset \mathbb{K}[x_1, \cdots, x_n]$.
Given $f \in \mathbb{K}[x_1, \cdots, x_n]$, we say that $f$ reduces to zero modulo $G$,\footnote{This is different from $\stackrel{G}{\to}$ of division algorithm in \S\ref{normalForms}.}
\begin{eqnarray}
f \to_G 0,
\end{eqnarray}
if $f$ can be written in the form,
\begin{eqnarray}
f = \sum_{i=1}^t a_i g_i
\end{eqnarray}
s.t., whenever $a_i g_i \neq 0$, we have
\begin{eqnarray}
MD(f) \geq MD(a_i g_i).
\end{eqnarray}

\subsection{$f \stackrel{G}{\to} 0$(division) $\Rightarrow f \to_G 0$(reduction to 0 modulo $G$)}
\label{ModuloR0}
Let $G = (g_1, \cdots, g_s)$ be an ordered tuple of elements in $\mathbb{K}[x_1, \cdots, x_n]$.
Then $f \stackrel{G}{\to} 0$ implies $f \to_G 0$, though the converse is false in general.

\subsubsection{Proof}
Here we will do essentially the same discussion in \S\ref{membership}.
If $f \stackrel{G}{\to} 0$, by definition
\begin{eqnarray}
f = \sum_{i=1}^t a_i g_i + 0,
\end{eqnarray}
and for $a_i g_i \neq 0$, they satisfy eq.(\ref{decreasingMultiDegree}),
\begin{eqnarray}
MD(f) \geq MD(a_i g_i).
\end{eqnarray}
This shows that $f \to_G 0$.

The counterexample for $\Leftarrow$ direction is in \S\ref{membership}.\footnote{If $G$ is gr\"obner, both $\stackrel{G}{\to}$ and $\to_G$ are equivalent, see \S\ref{IMC}.}
\\
$\blacksquare$

\subsection{Gr\"obner basis criterion, a more general version}
\label{efficientGCriterion}
A basis $G = \{g_1, \cdots, g_s\}$ for an ideal $I \subset \mathbb{K}[x_1, \cdots, x_n]$ is a Gr\"obner basis iff $\forall i \neq j, S(g_i, g_j) \to_G 0$.

\subsubsection{Proof}
In the proof of \S\ref{Buchberger'sCriterion}, what we have used are the expressions
\begin{eqnarray}
S(g_j, g_k) = \sum_{i} a_{i,j,k} g_i
\end{eqnarray}
with the conditions eq.(\ref{MDSgeqAG})
\begin{eqnarray}
MD\left(S(g_j, g_k) \right) \geq MD( a_{i,j,k} g_i)
\end{eqnarray}
for nonzero $a_{i,j,k}$.
This is the same as $S(g_i, g_j) \to_G 0$ in \S\ref{ModuloR0}, and
\begin{eqnarray}
G \text{ is gr\"obenr} \Leftrightarrow S(g_i, g_j) \to_G 0.
\end{eqnarray}
$\blacksquare$

\subsection{Proposition}
Given a finite set $G \subset \mathbb{K}[x_1, \cdots, x_n]$, suppose that we have $f,g \in G$ s.t.
\begin{eqnarray}
\label{relativelyPrime}
LCM( LM(f), LM(g)) = LM(f) * LM(g).
\end{eqnarray}
This means that the leading monomials of $f$ and $g$ are relatively prime.
Then $S(f,g) \to_G 0$.

If this condition eq.(\ref{relativelyPrime}) holds, then this S-polynomial is guaranteed to reduce to zero, so we don't need to polynomial division on this particular $S(f,g)$.

\subsubsection{Proof}
For simplicity, we can assume $LC(f) = 1 = LC(g)$, i.e. $LT(f) = LM(f), LT(g) = LM(g)$ and
\begin{eqnarray}
f &=& LM(f) + p \\
g &=& LM(g) + q
\end{eqnarray}
Clearly,
\begin{eqnarray}
\label{strictSmallnessConditions1}
MD\left( LM(f) \right) &>& MD(p) \\
\label{strictSmallnessConditions2}
MD\left( LM(g) \right) &>& MD(q).
\end{eqnarray}

By definition of S-polynomial,
\begin{eqnarray}
S(f,g) = \frac{x^\gamma}{LT(f)} f - \frac{x^\gamma}{LT(g)} g ,
\end{eqnarray}
where $x^\gamma = LCM( LM(f), LM(g))$ satisfies eq.(\ref{relativelyPrime}) and then
\begin{eqnarray}
S(f,g) &=& \frac{LM(f) * LM(g)}{LT(f)} f - \frac{LM(f) * LM(g)}{LT(g)} g \\
&=& LM(g) f - LM(f) g \\
&=& (g-q)f - (f-p)g \\
&=& pg - qf.
\end{eqnarray}
Since $f,g \in G$, we have
\begin{eqnarray}
S(f,g) \to_G 0.
\end{eqnarray}

If the leading terms in $pg$ and $gf$ are the same, then
\begin{eqnarray}
LM(p)LM(g) = LM(q)LM(f).
\end{eqnarray}
Since we have assumed eq.(\ref{relativelyPrime}), this means
\begin{eqnarray}
\exists h (\neq 0) \in \mathbb{K}[x_1, \cdots, x_n] \text{ s.t. } LM(p) = h*LM(f), LM(q) = h* LM(g).  \qquad
\end{eqnarray}
It contradicts eq.(\ref{strictSmallnessConditions1}) and eq.(\ref{strictSmallnessConditions2}).
Thus the leading terms in the last expression for $S(f,g)$ cannot cancel:
\begin{eqnarray}
MD\left(S(f,g) \right) = \max\left( MD(pg), MD(qf) \right).
\end{eqnarray}
$\blacksquare$

\subsubsection{Note}
This proof gives us a more efficient version of \S\ref{efficientGCriterion}; to test for a Gr\"obner basis, we need only have
\begin{eqnarray}
\forall i<j, S(g_i, g_j) \to_G 0,
\end{eqnarray}
where $LM(g_i),LM(g_j)$ are not relatively prime, see eq.(\ref{relativelyPrime}).

Here is another way to improve, a kind of generalization of S-polynomials.

\subsection{Definition of syzygies}
Let $F = (f_1, \cdots, f_s)$ be a tuple of polynomials.
A syzygy\footnote{
From our source book;
\begin{quotation}
This word is used in astronomy to indicate an alignment of three planets or other heavenly bodies.
The root is a Greek word meaning "yoke."
In an astronomical syzygy, planets are "yoked together"; in a mathematical syzygy, it is polynomials that are "yoked."
\end{quotation}
} on the leading terms $LG(f_1), \cdots, LT(f_s)$ of $F$ is an s-tuple of polynomials
\begin{eqnarray}
(h_1, \cdots, h_s) \in (\mathbb{K}[x_1, \cdots, x_n])^s
\end{eqnarray}
s.t.,
\begin{eqnarray}
\sum_{i=1}^s h_i * LT(f_i) = 0.
\end{eqnarray}
We let $S(F)$ be the subset of $(\mathbb{K}[x_1, \cdots, x_n])^s$ consisting of all syzygies on the leading terms of $F$.

Let
\begin{eqnarray}
e_i = (0, \cdots, 0,\underbrace{1}_{\text{i-th}},0, \cdots, 0) \in (\mathbb{K}[x_1, \cdots, x_n])^s
\end{eqnarray}
be a unit tuple, then a syzygy $S \in S(F)$ can be written as
\begin{eqnarray}
S = \sum_{i=1}^s h_i e_i.
\end{eqnarray}
Given a pair, $i<j$,
\begin{eqnarray}
(f_i, f_j) \subset F,
\end{eqnarray}
then 
\begin{eqnarray}
S_{ij} =  \frac{x^\gamma}{LT(f_i)} e_i - \frac{x^\gamma}{LT(f_j)} e_j
\end{eqnarray}
gives a syzygy on the leading terms of $F$, where $x^\gamma$ is the least common multiple of $LT(f_i), LT(f_j)$.
In fact, the name S-polynomial is actually an abbreviation for "syzygy polynomial."

\subsection{Definition of homogeneous of multidegree}
An element $S \in S(F)$ is homogeneous of multidegree $\alpha$, where $\alpha \in \mathbb{N}^n$, provided that
\begin{eqnarray}
S = (c_1x^{\alpha(1)}, \cdots, c_sx^{\alpha(s)}),
\end{eqnarray}
where $c_i \in \mathbb{K}$ and $\alpha(i) + MD(f_i) = \alpha$, whenever $c_i \neq 0$.

\subsection{? Lemma}
Every element of $S(F)$ can be written uniquely as a sum of homogeneous elements of $S(F)$.
That is, $S(F)$ has a finite basis.

\subsubsection{Proof}
Let $S = (h_1, \cdots, h_s) \in S(F)$, i.e. $F = (f_1, \cdots, f_s)$ with
\begin{eqnarray}
\sum_{i=1}^s h_i * LT(f_i) = 0.
\end{eqnarray}


\subsection{? Proposition}
Given $F = (f_1, \cdots, f_s)$, every syzygy $S \in S(F)$ can be written as
\begin{eqnarray}
S = \sum_{i<j} u_{ij} S_{ij},
\end{eqnarray}
where $u_{ij} \in \mathbb{K}[x_1, \cdots, x_n]$ and the syzygy $S_{ij}$ is defined in eq.(?)

\subsection{? Theorem}
A basis $G = (g_1, \cdots, g_t)$ for an ideal $I$ is a Gr\"obner basis iff for every element $S = (h_1, \cdots, h_t)$ in a homogeneous basis for the syzygies $S(G)$, we have
\begin{eqnarray}
S \cdot G = \sum_{i = 1}^t h_i g_i \to_G 0.
\end{eqnarray}

\subsection{? Proposition} 
Give $G = (g_1, \cdots, g_t)$, suppose that $S \subset \{S_{ij} | 1 \leq i < j \leq t\}$ is a basis of $S(G)$.
In addition, suppose we have distinct elements $g_i, g_j, g_k \in G$ s.t.
\begin{eqnarray}
LT(g_k) \text{ divides } LCM( LT(g_i), LT(g_j))
\end{eqnarray}
IF $S_{ij}, S_{jk} \in \mathcal{S}$, then $\mathcal{S} -\{ S_{ij} \}$ is also a basis of $S(G)$.
(Note: If $i > j$, we set $S_{ij} = S_{ji}$.)

\subsection{? Theorem}
Let $I = \left< f_1, \cdots, f_s \right>$ be a polynomial ideal.
Then a Gr\"obner basis for $I$ can be constructed in a finite number of steps by the following algorithm: $\cdots$

\section{A side story on ACC}
We have used Hilbert Basis Theorem in the proof of ACC \S\ref{ACC}, but ACC is independent from Hilbert Basis Theorem.
First of all, we should give more precise definitions.

\subsection{Chains, and ascending chains}
Let $(S, \leq)$ be a non-empty partially ordered set.
The subset $C \subset S$ is called a chain iff $C$ is totally ordered, i.e.,
\begin{eqnarray}
\forall c,c' \in C, \text{ either } c \leq c' \text{ or } c' \leq c.
\end{eqnarray}

The chain $C$ is an ascending chain iff the elements of $C$ are $\mathbb{N}$ indexed s.t.,
\begin{eqnarray}
\forall k \in \mathbb{N}, c_k \leq c_{k+1}, \lnot (\exists d \in C \text{ s.t. } c_k < d < c_{k+1}),
\end{eqnarray}
where $<$ is equivalent to ($\leq$ and $\neq$, i.e., $\lneq$).\footnote{Descending chains are defined similarly.}

\subsection{Two equivalent conditions for ACC}
\label{MoreACC}
Let $(S, \leq)$ be a non-empty partially ordered set.
Then the following conditions are equivalent.
\begin{enumerate}
\item (maximal) $\forall T \subset S$ of a partially ordered subset,
\begin{eqnarray}
\exists m \in T \text{ s.t. } \lnot( \exists n \in T \text{ s.t. } m \leq n).
\end{eqnarray}

\item (strict upper bound)\footnote{The upper bound is in $S$, in general.} $\forall C \subset S$ of a chain,
\begin{eqnarray}
\exists u \in C \text{ s.t. } \forall c \in C, c \leq u.
\end{eqnarray}
 
\item (ACC) $\forall C \subset S$ of an ascending chain, with $\left. c_k \right|_{k \in \mathbb{N}} \in C$, then
\begin{eqnarray}
\exists n \in \mathbb{N} \text{ s.t. } c_{n+1} = c_{n}.
\end{eqnarray}
That is, every ascending chain will terminate finitely:
\begin{eqnarray}
C = \{c_0 \leq c_1 \leq \cdots \leq c_n\}
\end{eqnarray}

\end{enumerate}

\subsubsection{Proof}
We prove $1. \Rightarrow 2. \Rightarrow 3. \Rightarrow 1.$
 
(maximal) $\Rightarrow $ (strict upper bound)
Take
\begin{eqnarray}
u := \max (C).
\end{eqnarray}

(strict upper bound) $\Rightarrow$ (ACC)
For
\begin{eqnarray}
u := \text{upper bound}(C)
\end{eqnarray}
there should be $n \in \mathbb{N}$ s.t.
\begin{eqnarray}
u = c_n.
\end{eqnarray}

(ACC) $\Rightarrow$ (maximal)
Consider $\forall T (\neq \emptyset) \subset S$.
Suppose $T$ has NO maximal element, then take an element $t_0 \in T$.
We can take
\begin{eqnarray}
t_1 \in (T - \{t_0\}) \text{ s.t. } t_0 < t_1,
\end{eqnarray}
since $t_0$ is not maximal.
Inductively, we can take a subset
\begin{eqnarray}
\{t_0 < t_1 < \cdots \} \subset T,
\end{eqnarray}
by Axiom of Choice.\footnote{
More rigorously, for non empty sets
\begin{eqnarray}
T_0 := T, T_1 := T_0 - \{t_0\}, \cdots, T_n := T_{n-1}-\{t_{n-1}\}
\end{eqnarray}
there exists a choice function
\begin{eqnarray}
t: \mathbb{N} \to \bigcup_{n \in \mathbb{N}} T_i = T; n \mapsto t_n \in T_n.
\end{eqnarray}

}
However, this is an infinite (strictly) ascending chain, and contradicts ACC.
Thus $T$ has a maximal element.\\
$\blacksquare$


\chapter{Elimination Theory}

the Geometric Extension Theorem for (weak) Nullstellensatz

\section{The Elimination and Extension Theorems}
\setcounter{subsection}{-1}
\subsection{Example}
Here is the list of equations that we will solve:
\begin{eqnarray}
x^2 + y + z &=& 1 \\
x + y^2 + z &=& 1 \\
x + y + z^2 &=& 1
\end{eqnarray}
Putting them into Maxima, we get the Gr\"obner basis for the ideal that is generated by above equations.
\begin{verbatim}
(%i1) load(grobner)$
(%i2) display2d:false$
(%i3) poly_reduced_grobner ([x^2+y+z-1, x+y^2+z-1, x+y+z^2-1],[x,y,z]);
(%o3) [z^2+y+x-1,(-z^2)+z+y^2-y,z^4+2*y*z^2-z^2,z^6-4*z^4+4*z^3-z^2]
\end{verbatim}
where the default settings are
\begin{verbatim}
(%i5) poly_monomial_order;
(%o5) lex
(%i6) poly_grobner_algorithm;
(%o6) buchberger
\end{verbatim}

If we set
\begin{verbatim}
(%i14) g: poly_reduced_grobner ([x^2+y+z-1, x+y^2+z-1, x+y+z^2-1],[x,y,z]);
(%o14) [z^2+y+x-1,(-z^2)+z+y^2-y,z^4+2*y*z^2-z^2,z^6-4*z^4+4*z^3-z^2]
(%i15) g[4], factor;
(%o15) (z-1)^2*z^2*(z^2+2*z-1)
\end{verbatim}
we can solve the 4th Gr\"obner basis \verb|g[4]| for $z$:
\begin{verbatim}
(%i16) solve(%,z);
(%o16) [z = (-sqrt(2))-1,z = sqrt(2)-1,z = 0,z = 1]
\end{verbatim}
Substituting these solution in the rest equations, we'll get the full solutions.
\begin{verbatim}
(%i46) g, z=0;
(%o46) [y+x-1,y^2-y,0,0]
(%i47) solve(%, [x,y]);
(%o47) [[x = 1,y = 0],[x = 0,y = 1]]

(%i49) g, z=1;
(%o49) [y+x,y^2-y,2*y,0]
(%i50) solve(%, [x,y]);
(%o50) [[x = 0,y = 0]]

(%i52) g, z = (-sqrt(2))-1;
(%o52) [y+x+((-1)-sqrt(2))^2-1,y^2-y-sqrt(2)-((-1)-sqrt(2))^2-1,
        2*((-1)-sqrt(2))^2*y+((-1)-sqrt(2))^4-((-1)-sqrt(2))^2,
        ((-1)-sqrt(2))^6-4*((-1)-sqrt(2))^4+4*((-1)-sqrt(2))^3
                        -((-1)-sqrt(2))^2]
(%i53) solve(%, [x,y]);
(%o53) [[x = (-sqrt(2))-1,y = (-sqrt(2))-1]]

(%i55) g, z = sqrt(2)-1;
(%o55) [y+x+(sqrt(2)-1)^2-1,y^2-y+sqrt(2)-(sqrt(2)-1)^2-1,
        2*(sqrt(2)-1)^2*y+(sqrt(2)-1)^4-(sqrt(2)-1)^2,
        (sqrt(2)-1)^6-4*(sqrt(2)-1)^4+4*(sqrt(2)-1)^3-(sqrt(2)-1)^2]
(%i56) solve(%, [x,y]);
(%o56) [[x = sqrt(2)-1,y = sqrt(2)-1]]
\end{verbatim}
Maxima can also solve these equations directly, and it's consistent:
\begin{verbatim}
(%i57) solve([x^2+y+z-1, x+y^2+z-1, x+y+z^2-1],[x,y,z]);
(%o57) [[x = 0,y = 0,z = 1],[x = sqrt(2)-1,y = sqrt(2)-1,z = sqrt(2)-1],
        [x = (-sqrt(2))-1,y = (-sqrt(2))-1,z = (-sqrt(2))-1],
        [x = 1,y = 0,z = 0],[x = 0,y = 1,z = 0]]
\end{verbatim}

Here is the sketch of above process.
\begin{enumerate}
\item (Elimination)
Find a polynomial which involved only one variable, say $z$.

\item (Extension)
Putting each solution for above eliminated equation(s), extend these solutions of the original equations.
\end{enumerate}

\subsection{The elimination ideals}
Given $I = \left< f_1, \cdots, f_s \right> \subset \mathbb{K}[x_1, \cdots, x_n]$, the $l$-th elimination ideal $I_l$ is the ideal defined by
\begin{eqnarray}
I_l := I \cap \mathbb{K}[x_{l+1}, \cdots, x_n].
\end{eqnarray}
That is $I_l$ is an ideal which generated by $f_1 = \cdots = f_s=0$ expressed in only $x_{l+1}, \cdots, x_n$.

\subsubsection{Check}
Since $0\in I$ can be seen as a constant polynomial,
\begin{eqnarray}
0 \in I_l.
\end{eqnarray}
$\forall f,g \in I_l \subset I$,
\begin{eqnarray}
f + g \in I
\end{eqnarray}
and both $f$ and $g$ are functions only on $x_{l+1}, \cdots, x_n$, so $f+g \in \mathbb{K}[x_{l+1}, \cdots, x_n]$, i.e.,
\begin{eqnarray}
f + g \in I_l.
\end{eqnarray}
Finally, $\forall f \in I_l \subset I, r \in \mathbb{K}[x_{l+1}, \cdots, x_n]$,
\begin{eqnarray}
r * f \in I,
\end{eqnarray}
and $r * f \in \mathbb{K}[x_{l+1}, \cdots, x_n]$, that is
\begin{eqnarray}
r * f \in I_l.
\end{eqnarray}
Therefore, $I_l$ is an ideal of $ \mathbb{K}[x_{l+1}, \cdots, x_n]$.\\
$\blacksquare$

\subsubsection{Note}
More generally, the first elimination of $I_l \subset \mathbb{K}[x_{l+1}, \cdots, x_n]$ is $I_{l+1}$, since the first elimination of $I_l$ is given by
\begin{eqnarray}
I_l \cap \mathbb{K}[x_{l+2}, \cdots, x_n]
\end{eqnarray}
but
\begin{eqnarray}
I_{l+1} &:=& I \cap \mathbb{K}[x_{l+2}, \cdots, x_n] \\
&=& \left( I \cap \mathbb{K}[x_{l+1}, \cdots, x_n] \right) \cap \mathbb{K}[x_{l+2}, \cdots, x_n] \\
&=& I_l \cap \mathbb{K}[x_{l+2}, \cdots, x_n].
\end{eqnarray}
$\blacksquare$

\subsubsection{The Elimination Step}
Thus a solution of the Elimination Step means giving a systematic procedure for finding elements of $I_l$.
With the proper term ordering, Gr\"obner bases allow us to do this instantly.

\subsection{The Elimination Theorem}
Let $I \subset \mathbb{K}[x_1, \cdots, x_n]$ be an ideal and $G$ be a Gr\"obner basis for $I$ with $>_{lex}$, s.t.,
\begin{eqnarray}
x_1 > x_2 > \cdots > x_n.
\end{eqnarray}
Then $ 0 \leq \forall l \leq n$, the set
\begin{eqnarray}
G_l := G \cap \mathbb{K}[x_{l+1}, \cdots, x_n]
\end{eqnarray}
is a Gr\"obner basis of the l-th elimination ideal $I_l$.

\subsubsection{Proof}
It suffices to show $\left< LT(G_l) \right> \supset \left< LT(I_l)\right>$ of eq.(\ref{isGroebner}) "isGr\"obner".

Consider $\forall f \in I_l \subset I$.
Since $G$ is a Gr\"obner basis for $I$, there is some $g \in G$ whose leading term divides $LT(f)$:
\begin{eqnarray}
LT(f) \sqsupseteq LT(g) \Leftrightarrow : LT(f) \in \left< LT(G) \right>.
\end{eqnarray}

Now we have
\begin{eqnarray}
f \in I_l := I \cap \mathbb{K}[x_{l+1}, \cdots, x_n] \Rightarrow f \in \mathbb{K}[x_{l+1}, \cdots, x_n]
\end{eqnarray}
and since we are using $>_{lex}$ with
\begin{eqnarray}
x_1 > x_2 > \cdots > x_n,
\end{eqnarray}
any monomial involving $x_1, \cdots, x_l$ is greater than all monomials in $\mathbb{K}[x_{l+1}, \cdots, x_n]$, say $f$:
\begin{eqnarray}
\forall h(x_1, \cdots, x_l) > f(x_{l+1}, \cdots, x_n).
\end{eqnarray}

Since $LT(f)$ is divisible by $LT(g)$ and from the above observation, we get
\begin{eqnarray}
LT(g) \in \mathbb{K}[x_{l+1}, \cdots, x_n]
\end{eqnarray}
This implies that
\begin{eqnarray}
g \in \mathbb{K}[x_{l+1}, \cdots, x_n],
\end{eqnarray}
and
\begin{eqnarray}
g \in G \cap \mathbb{K}[x_{l+1}, \cdots, x_n] =: G_l.
\end{eqnarray}
Therefore this $g$ is indeed in $G_l$ and
\begin{eqnarray}
g \in G_l, LT(f) \sqsupseteq LT(g) \Leftrightarrow LT(f) \in \left< LT(G_l) \right>.
\end{eqnarray}
So $\forall f \in I_l$,
\begin{eqnarray}
LT(f) \in \left< LT(I_l) \right> \Rightarrow LT(f) \in \left< LT(G_l) \right>.
\end{eqnarray}
$\blacksquare$

\subsection{? The Extension Theorem}
Let $I \subset \mathbb{C}[x_1, \cdots, x_n]$ be an ideal and $I_1$ be the 1st elimination ideal of $I$.
$1 \leq \forall i \leq s$, write $f_i$ in the form
\begin{eqnarray}
f_i = g_i(x_2, \cdots, x_n)x_1^{N_i} + o(x_1^{N_i}),
\end{eqnarray}
where $N_i \geq 0$ and $g_i \in \mathbb{C}[x_2, \cdots, x_n]$ is nonzero, and $o(x_1^{N_i})$ stands for the terms in which $x_1$ has degree $< N_i$.
Suppose that we have a partial solution $(a_2, \cdots, a_n) \in \mathbb{V}(I_1)$.
If $(a_2, \cdots, a_n) \notin \mathbb{V}(g_1, \cdots, g_s)$, then there exists $a_1 \in \mathbb{C}$ s.t. $(a_1, \cdots, a_n) \in \mathbb{V}(I)$.

\subsubsection{Note}
The proof of this theorem uses resultant in \S.

This extension theorem is especially easy to use when one of the leading coefficients is constant.

\subsection{? Corollary}
Let $I = \left< f_1, \cdots, f_s \right> \subset \mathbb{C}[x_1, \cdots, x_n]$, and assume that for some $i$, $f_i$ is the form
\begin{eqnarray}
f_i = c x_1^N + o(x_1^N),
\end{eqnarray}
where $c \in \mathbb{C}$ is nonzero and $N>0$.
If $I_1$ is the 1st elimination ideal of $I$ and $(a_2, \cdots, a_n) \in \mathbb{V}(I_1)$, then there is $a_1 \in \mathbb{C}$ s.t. $(a_1, \cdots, a_n) \in \mathbb{V}(I)$.

\subsubsection{Proof}
This follows from the Extension Theorem; since this $g_i = c \neq 0$ implies
\begin{eqnarray}
\mathbb{V}(g_1, \cdots, g_s) = \left\{a \left| g_1(a) = \cdots = g_s(a) = 0\right. \right\}= \emptyset,
\end{eqnarray}
the if-statement
\begin{eqnarray}
(a_2, \cdots, a_n) \notin \mathbb{V}(g_1, \cdots, g_s)
\end{eqnarray}
is True for all partial solutions.\\
$\blacksquare$

\section{The Geometry of Elimination}
For simplicity, we will work over $\mathbb{C}$.

\subsection{Projections of an affine variety}
Suppose we have
\begin{eqnarray}
V := \mathbb{V}(f_1, \cdots, f_s) \subset \mathbb{C}^n.
\end{eqnarray}
To eliminate the first $l$ variables, consider the projection map
\begin{eqnarray}
\pi_l : \mathbb{C}^n \to \mathbb{C}^{n-l}; (a_1, \cdots, a_n) \mapsto (a_{l+1}, \cdots, a_n).
\end{eqnarray}
The image of $V \subset \mathbb{C}^n$ is in
\begin{eqnarray}
\pi_l(V) \subset \mathbb{C}^{n-l}.
\end{eqnarray}

\subsection{Lemma}
\label{pi_lVIsInVI_l}
Let
\begin{eqnarray}
I_l := \left< f_1, \cdots, f_s\right> \cap \mathbb{C}[x_{l+1}, \cdots, x_n]
\end{eqnarray}
be the $l$-th elimination ideal.
Then
\begin{eqnarray}
\pi_l(V) \subset \mathbb{V}(I_l) (\subset \mathbb{C}^{n-l})
\end{eqnarray}

\subsubsection{Proof}
Let $\forall f \in I_l \subset \left< f_1, \cdots, f_s\right>$ vanishes at $\forall (a_1, \cdots, a_n) \in V := \mathbb{V}(f_1, \cdots, f_s)$,
\begin{eqnarray}
f(a_1, \cdots, a_n) = 0.
\end{eqnarray}
However, since $f \in I_l := I \cap \mathbb{K}[x_{l+1}, \cdots, x_n]$, $f$ depends only on $x_{l+1}, \cdots, x_n$,
\begin{eqnarray}
f(a_{l+1}, \cdots, a_n) = f\circ \pi_l(a_1, \cdots, a_n) = 0.
\end{eqnarray}
That is, $f$ vanishes at all points in $\pi_l(V)$.\\
$\blacksquare$

\subsection{? Geometric Extension Theorem}
\label{GeometricExtensionTheorem}
Geometrical interpretation of the Extension Theorem; given $V := \mathbb{V}(f_1, \cdots, f_s) \subset \mathbb{C}^n$, let $g_i \in \mathbb{C}[x_2, \cdots, x_n]$ be as in the Extension Theorem.
If $I_1$ is the 1st elimination ideal of $\left< f_1, \cdots, f_s\right>$, then we have
\begin{eqnarray}
\mathbb{V}(I_1) = \pi_1(V) \cup \left( \mathbb{V}(g_1, \cdots, g_s) \cap \mathbb{V}(I_1) \right) (\subset \mathbb{C}^{n-1}),
\end{eqnarray}
where $\pi_1: \mathbb{C}^n \to \mathbb{C}^{n-1}$ is the projection onto the last $n-1$ components.

\subsubsection{Proof}
Since we have shown that $\pi_1(V) \subset \mathbb{V}(I_1)$, 
\begin{eqnarray}
\mathbb{V}(I_1) \supset \pi_1(V) \cup \left( \mathbb{V}(g_1, \cdots, g_s) \cap \mathbb{V}(I_1) \right) 
\end{eqnarray}

Conversely, consider an arbitrary partial solution in $\mathbb{V}(I_1)$
\begin{eqnarray}
(a_2, \cdots, a_n) \in \mathbb{V}(I_1).
\end{eqnarray}
From Fig.\ref{vennDiagram1}, this partial solution is either in
\begin{eqnarray}
(a_2, \cdots, a_n) \in \mathbb{V}(g_1, \cdots, g_s) \cap \mathbb{V}(I_1)
\end{eqnarray}
or in
\begin{eqnarray}
(a_2, \cdots, a_n) \notin \mathbb{V}(g_1, \cdots, g_s) \cap \mathbb{V}(I_1).
\end{eqnarray}
%%%%%%%%%%%%%%%%%%%%%%%%%%VennDiagram%%%%%%%%%%%%%%%%%%%%%%%%%%
\begin{figure}[htbp]
\begin{center}

\def\firstcircle{(90:1.75cm) circle (2.5cm)}
\def\secondcircle{(210:1.75cm) circle (2.5cm)}
\def\thirdcircle{(330:1.75cm) circle (2.5cm)}

\begin{tikzpicture}
\begin{scope}
    \clip \secondcircle;
%    \fill[gray] \thirdcircle;
\end{scope}

\begin{scope}
    \clip \firstcircle;
    \fill[gray] \thirdcircle;
\end{scope}

\draw \firstcircle node[text=black,above] {$V(I_1)$};
\draw \secondcircle node [text=black,below left] {$\pi_1(V)$};
\draw \thirdcircle node [text=black,below right] {$V(g)$};
\end{tikzpicture}

\caption{Three sets of $V(I_1), V(g)$, and $\pi_1(V)$}
\label{vennDiagram1}
\end{center}
\end{figure}
%%%%%%%%%%%%%%%%%%%%%%%%%%VennDiagram%%%%%%%%%%%%%%%%%%%%%%%%%%
If $(a_2, \cdots, a_n) \in \mathbb{V}(g_1, \cdots, g_s) \cap \mathbb{V}(I_1)$, then clearly
\begin{eqnarray}
(a_2, \cdots, a_n) \in \pi_1(V) \cup \left( \mathbb{V}(g_1, \cdots, g_s) \cap \mathbb{V}(I_1) \right).
\end{eqnarray}
Else $(a_2, \cdots, a_n) \notin \mathbb{V}(g_1, \cdots, g_s) \cap \mathbb{V}(I_1)$, then we also have
\begin{eqnarray}
(a_2, \cdots, a_n) \notin \mathbb{V}(g_1, \cdots, g_s)
\end{eqnarray}
and we can apply the Extension Theorem:
\begin{eqnarray}
\exists a_1 \in \mathbb{C} \text{ s.t. } (a_1, a_2, \cdots, a_n) \in \mathbb{V}(I),
\end{eqnarray}
and
\begin{eqnarray}
(a_2, \cdots, a_n) = \pi_1(a_1, a_2, \cdots, a_n) \in \pi_1(V).
\end{eqnarray}
Therefore, in either case
\begin{eqnarray}
a \in V(I_1) \Rightarrow a \in \pi_1(V) \cup \left( \mathbb{V}(g_1, \cdots, g_s) \cap \mathbb{V}(I_1) \right) 
\end{eqnarray}
that is
\begin{eqnarray}
\mathbb{V}(I_1) \subset \pi_1(V) \cup \left( \mathbb{V}(g_1, \cdots, g_s) \cap \mathbb{V}(I_1) \right) 
\end{eqnarray}
$\blacksquare$

\subsection{? The Closure Theorem, a special case}
Let $V := \mathbb{V}(f_1, \cdots, f_s) \subset \mathbb{C}^n$ and let $I_1$ be the 1st elimination ideal of $\left< f_1, \cdots, f_s \right>$.
Then
When $V \neq \emptyset$, there exists an affine variety $W \subsetneq \mathbb{V}(I_1)$ s.t., $\mathbb{V}(I_1) - W \subset \pi_1(V).$\footnote{
This is an instance of the following general statement; let $V := \mathbb{V}(f_1, \cdots, f_s) \subset \mathbb{C}^n$ and let $I_l$ be the $l$th elimination ideal of $\left< f_1, \cdots, f_s \right>$.
Then
\begin{enumerate}
\item $\mathbb{V}(I_l)$ is the smallest affine variety containing $\pi_l(V) \subset \mathbb{C}^{n-l}$.

\item When $V \neq \emptyset$, there exists an affine variety $W \subsetneq \mathbb{V}(I_l)$ s.t. $\mathbb{V}(I_l) - W \subset \pi_l(V).$
\end{enumerate}
}

\subsubsection{Proof}
Let 
\begin{eqnarray}
W' := \mathbb{V}(g_1, \cdots, g_s) \cap \mathbb{V}(I_1)
\end{eqnarray}
and note that $W$ is an affine variety by \S\ref{intersectionAndUnionOfVarieties}.
The decomposition in \S\ref{GeometricExtensionTheorem}
\begin{eqnarray}
\mathbb{V}(I_1) &=& \pi_1(V) \cup \left( \mathbb{V}(g_1, \cdots, g_s) \cap \mathbb{V}(I_1) \right)\\
&=& \pi_1(V) \cup W'
\end{eqnarray}
implies 
\begin{eqnarray}
\mathbb{V}(I_1) - W' \subset \pi_1(V),
\end{eqnarray}
and if $W' \neq \mathbb{V}(I_1)$, then we are done.

If $W' = \mathbb{V}(I_1)$, we need to change the equations defining $V$ s.t. $W$ becomes smaller.

\paragraph{$W' = \mathbb{V}(I_1) \Rightarrow V = \mathbb{V}(f_1, \cdots, f_s, g_1, \cdots, g_s)$}
$\because$ Since
\begin{eqnarray}
f_1(a) = \cdots= f_s(a) = g_1(a) = \cdots = g_s = 0 \Rightarrow f_1(a) = \cdots= f_s(a) = 0 \qquad
\end{eqnarray}
one direction follows
\begin{eqnarray}
\mathbb{V}(f_1, \cdots, f_s, g_1, \cdots, g_s) \subset V := \mathbb{V}(f_1, \cdots, f_s) 
\end{eqnarray}
by definition in \S\ref{DefOfAV}.

For the opposite direction, let
\begin{eqnarray}
(a_1,a_2, \cdots, a_n) \in V := \mathbb{V}(f_1, \cdots, f_s).
\end{eqnarray}
Then from \S\ref{pi_lVIsInVI_l},
\begin{eqnarray}
(a_2, \cdots, a_n) \in \pi_1(V) \subset \mathbb{V}(I_1) = W':= \mathbb{V}(g_1, \cdots, g_s) \cap \mathbb{V}(I_1)
\end{eqnarray}
and $\forall j$,
\begin{eqnarray}
g_j(a_2, \cdots, a_n) = 0.
\end{eqnarray}
Thus
\begin{eqnarray}
(a_1,a_2, \cdots, a_n) \in \mathbb{V}(f_1, \cdots, f_s, g_1, \cdots, g_s)
\end{eqnarray}
and
\begin{eqnarray}
\mathbb{V}(f_1, \cdots, f_s, g_1, \cdots, g_s) \supset V := \mathbb{V}(f_1, \cdots, f_s).
\end{eqnarray}

Therefore
\begin{eqnarray}
\mathbb{V}(f_1, \cdots, f_s, g_1, \cdots, g_s) = V := \mathbb{V}(f_1, \cdots, f_s).
\end{eqnarray}
$\blacksquare$

Let
\begin{eqnarray}
I := \left< f_1, \cdots, f_s \right>
\end{eqnarray}
be our original ideal and
\begin{eqnarray}
\tilde{I} := \left< f_1, \cdots, f_s, g_1, \cdots, g_s \right>.
\end{eqnarray}
Then in general $I \subset \tilde{I}$, but we already have
\begin{eqnarray}
\mathbb{V}(I) = \mathbb{V}(\tilde{I}).
\end{eqnarray}

\paragraph{? $\mathbb{V}(I_1) = \mathbb{V}(\tilde{I_1})$}





%%%%%%%%%%%%%%%%%%%%%%%%%%%%%%%%%%%%%%%%%%%%%%%%%

\chapter{The Algebra-Geometry Dictionary}
\section{Hilbert's Nullstellensatz}
Null(=Zero), Stellen(=Places), Satz(=Theorem).

\subsection{? The Weak Nullstellensatz}
Let $\mathbb{K}$ be an algebraically closed filed and let $I \subset \mathbb{K}[x_1, \cdots, x_n]$ be an ideal.
Then
\begin{eqnarray}
\mathbb{V}(I) = \emptyset \Rightarrow I = \mathbb{K}[x_1, \cdots, x_n].
\end{eqnarray}

\subsubsection{Proof}
The standard strategy is to show
\begin{eqnarray}
1 \in I.
\end{eqnarray}
It is because, if $1 \in I$ then $\forall f \in \mathbb{K}[x_1, \cdots, x_n]$,
\begin{eqnarray}
f = f * 1 \in I \Leftrightarrow \mathbb{K}[x_1, \cdots, x_n] \subset I.
\end{eqnarray}

We prove by induction.

\paragraph{$n=1$ case}
From \S\ref{InOneDimension}, every ideal $I$ of $\mathbb{K}[x]$ is generated by a single polynomial $f$:
\begin{eqnarray}
I = \left< f\right>
\end{eqnarray}
Then the variety is the set of roots
\begin{eqnarray}
\left\{ \left. a \in \mathbb{K} \right| f(a) = 0 \right\}.
\end{eqnarray}

Since $\mathbb{K}$ is algebraically closed, i.e., $\forall f \in \mathbb{K}[x]$,
\begin{eqnarray}
\text{$f$ is not a const. poly.} \Rightarrow \exists a \in \mathbb{K} \text{ s.t. } f(a) = 0,
\end{eqnarray}
if $\forall a \in \mathbb{K}, f(a) \neq 0$ then $f$ is a non zero constant polynomial.
\begin{eqnarray}
\mathbb{V}(\left< f\right>) \Rightarrow f = c (\neq 0) \in \mathbb{K}.
\end{eqnarray}
Since $c \neq 0$, 
\begin{eqnarray}
\exists c^{-1} \in \mathbb{K}.
\end{eqnarray}
Therefore,
\begin{eqnarray}
c^{-1} * f = c^{-1} * c = 1 \in I
\end{eqnarray}
and 
\begin{eqnarray}
\forall I \subset \mathbb{K}[x], \mathbb{V}(I) = \emptyset \Rightarrow I = \mathbb{K}[x].
\end{eqnarray}

\paragraph{$(n-1)$ case $\Rightarrow n$ case}
Assume $(n-1)$ case
\begin{eqnarray}
\forall J \subset \mathbb{K}[x_2, \cdots, x_n], \mathbb{V}(J) = \emptyset \Rightarrow J = \mathbb{K}[x_2, \cdots, x_n]
\end{eqnarray}
and consider an arbitrary ideal
\begin{eqnarray}
I := \left< f_1, \cdots, f_s \right> \subset \mathbb{K}[x_1, \cdots, x_n]
\end{eqnarray}
for which $\mathbb{V}(I) = \emptyset $.
Since can assume $f_1$ is not a constant,\footnote{
If $f_1$ is a non zero constant polynomial, then using the same argument in $n=1$ case, we can show $1 \in I$.
} suppose $f_1$ has total degree $N \geq 1$.

Consider the following linear transformation:
\begin{eqnarray}
\nonumber
x_1 &=& y_1, \\
\nonumber
x_2 &=& y_2 + a_2*y_1, \\
\nonumber
&\vdots&\\
\label{LinearTransformInHWN}
x_n &=& y_n + a_n*y_1,
\end{eqnarray}
where $a_2, \cdots, a_n$ are constants in $\mathbb{K}$.
Then $f_1$ bocomes
\begin{eqnarray}
f_1(x_1, \cdots, x_n) &=& f_1(y_1, y_2 + a_2*y_1, \cdots, y_n + a_n*y_1) \\
&=& c(a_2, \cdots, a_n) * y_1^N + O(y_1^{N-1})
\end{eqnarray}

\paragraph{? Lemma ($c(a_2, \cdots, a_n)$ is not constantly 0)}
Consider a polynomial $f \in \mathbb{K}[x_1, \cdots, x_n]$ under the linear transform in eq.(\ref{LinearTransformInHWN}).
When we write $f$ as a sum
\begin{eqnarray}
f = h_N + \cdots + h_1 + h_0
\end{eqnarray}
of homogeneous polynomials
\begin{eqnarray}
h_m(x_1, \cdots, x_n) = \sum_{|\alpha| = m} c_\alpha x^\alpha, c_\alpha \in \mathbb{K}.
\end{eqnarray}
Note that, under the linear transform in eq.(\ref{LinearTransformInHWN}), the degree of above $h_m$ is unchanged.

Define
\begin{eqnarray}
\tilde{f}(y_1, \cdots, y_n) &:=& f(y_1, y_2 + a_2y_1, \cdots, y_n + a_n y_1)\\
&=& c(a_2, \cdots, a_n) * y_1^N + O(y_1^{N-1})
\end{eqnarray}
If we put $y_2 = \cdots = y_n = 0$,
\begin{eqnarray}
\tilde{f}(y_1, y_2 = \cdots = y_n = 0) &=& f(y_1, a_2y_1, \cdots, a_n y_1) \\
&=& h_N(y_1, a_2y_1, \cdots, a_n y_1) + O(y_1^{N-1}) \\
&=& h_N(1, a_2, \cdots, a_n) * y_1^N + O(y_1^{N-1}),
\end{eqnarray}
since $h_N$ has homogeneous degree $N$.














\chapter{Polynomial and Rational Functions on a Variety}
\section{Polynomial Mappings}
\subsection{Definition of polynomial mappings}
\label{polynomialMappings}
Let $V \subset \mathbb{K}^m, W \subset \mathbb{K}^n$ be varieties.
A function
\begin{eqnarray}
\phi: V \to W
\end{eqnarray}
is a polynomial mapping (or regular mapping) iff there exist polynomials
\begin{eqnarray}
f_1, \cdots, f_n \in \mathbb{K}[x_1, \cdots, x_m]
\end{eqnarray}
s.t., $\forall (a_1, \cdots, a_m) \in V$,
\begin{eqnarray}
\phi(a_1, \cdots, a_m) = \left( f_1(a_1, \cdots, a_m), \cdots, f_n(a_1, \cdots, a_m) \right).
\end{eqnarray}

We call this n-tuple of polynomials
\begin{eqnarray}
(f_1, \cdots, f_n) \in \left( \mathbb{K}[x_1, \cdots, x_m] \right)^n
\end{eqnarray}
represents $\phi$.

\subsection{Proposition}
Let $V \subset \mathbb{K}^m$ be an affine variety.
Then
\begin{enumerate}
\item $f,g \in \mathbb{K}[x_1, \cdots, x_m]$ represent the same polynomial function of $V$ iff $f-g \in \mathbb{I}(V)$.

\item $(f_1, \cdots, f_n)$ and $(g_1, \cdots, g_n)$ represent the same polynomial mapping from $V$ to $\mathbb{K}^n$ iff $f_i - g_i \in \mathbb{I}(V)$ for each $1 \leq i \leq n$.
\end{enumerate}

\subsubsection{Proof}
It suffices to show the 1st case.

If $f$ and $g$ represent the same function, then $\forall a \in V$,
\begin{eqnarray}
f(a) = g(a) \Leftrightarrow (f - g)(a) = 0.
\end{eqnarray}
Thus 
\begin{eqnarray}
f - g \in \mathbb{I}(V)
\end{eqnarray}

Conversely, if
\begin{eqnarray}
f - g \in \mathbb{I}(V),
\end{eqnarray}
then $\forall a = (a_1, \cdots, a_m) \in V$,
\begin{eqnarray}
(f-g)(a) = f(a) - g(a) = 0.
\end{eqnarray}
Hence, $f$ and $g$ represent the same function on $V$.\\
$\blacksquare$

\subsection{Definition of $\mathbb{K}[V]$}
Let us define
\begin{eqnarray}
\mathbb{K}[V] := \left\{\phi : V \to \mathbb{K} \left| \phi \text{ is a polynomial function} \right. \right\}
\end{eqnarray}
the set of polynomial functions.\footnote{
I prefer the exponential notion
\begin{eqnarray}
\mathbb{K}^V := \mathbb{K}[V].
\end{eqnarray}
}

\subsubsection{$\mathbb{K}[V]$ is a commutative ring}
$\forall \phi, \psi \in \mathbb{K}[V]$, the addition
\begin{eqnarray}
(\phi + \psi) (p) := \phi(p) + \psi(p), &\forall p \in V,
\end{eqnarray}
and the multiplication
\begin{eqnarray}
(\phi * \psi) (p) := \phi(p) * \psi(p), &\forall p \in V.
\end{eqnarray}
Identities are constant functions, e.g.,
\begin{eqnarray}
0 : V \to \mathbb{K}; p \to 0.
\end{eqnarray}
Associativity and distributive laws come from that of $\mathbb{K}$, e.g., $\forall p \in V$,
\begin{eqnarray}
(\phi + (\psi + \chi) ) (p) &=& \phi(p) + (\psi + \chi)(p)\\
&=& \phi(p) + \psi(p) + \chi(p)\\
&=& (\phi(p) + \psi(p)) + \chi(p)\\
&=& ((\phi + \psi) + \chi)(p).
\end{eqnarray}

\subsection{? Proposition}
Let $V \subset \mathbb{K}^n$ be an affine variety.
The following statements are equivalent.
\begin{enumerate}
\item $V$ is irreducible.
\item $\mathbb{I}(V)$ is a prime ideal.
\item $\mathbb{K}[V]$ is an integral domain.
\end{enumerate}






\section{Quotients of polynomial Rings}
\subsection{Congruent modulo $I$}
Let $I \subset \mathbb{K}[x_1, \cdots, x_m]$ be an ideal, and let $f,g \in \mathbb{K}[x_1, \cdots, x_m]$.
We say $f$ and $g$ are congruent modulo $I$,
\begin{eqnarray}
f \equiv g \mod I :\Leftrightarrow f-g \in I.
\end{eqnarray}

This congruence modulo $I$ is an equivalence relation on $\mathbb{K}[x_1, \cdots, x_n]$.
\begin{enumerate}
\item (reflexive) $\forall f \in I$,
\begin{eqnarray}
f - f = 0 \in I.
\end{eqnarray}

\item (symmetry) $\forall f,g \in I$,
\begin{eqnarray}
f - g \in I \Rightarrow g - f = -(f-g) \in I.
\end{eqnarray}

\item (transitivity) $\forall f,g,h \in I$,
\begin{eqnarray}
f-g \in I, g-h \in I \Rightarrow f-h = (f-g) + (g-h) \in I.
\end{eqnarray}

\end{enumerate}
We denote the equivalence class which is represented by $f \in I$,
\begin{eqnarray}
[f] := \left\{ \left. g \in \mathbb{K}[x_1, \cdots, x_m] \right| g \equiv f \mod I \right\}
\end{eqnarray}

\subsection{? Proposition}
\label{121Correspondence}
The distinct polynomial functions in $\mathbb{K}[V]$ are in one-to-one correspondence with the equivalence classes of polynomials under congruence modulo $\mathbb{I}(V)$.

\subsubsection{Proof}
Consider an arbitrary polynomial functions $\phi : V \to \mathbb{K}$, and its representation
\begin{eqnarray}
f \in \left( \mathbb{K}[x_1, \cdots, x_m] \right)
\end{eqnarray}
s.t., $\forall (a_1, \cdots, a_m) \in V$,
\begin{eqnarray}
\phi(a_1, \cdots, a_m) = f(a_1, \cdots, a_m).
\end{eqnarray}
We claim that
\begin{eqnarray}
[f] \stackrel{\Phi}{\mapsto} \phi
\end{eqnarray}
is the one-to-one correspondence.

Since every polynomial functions have its representation (see \S\ref{polynomialMappings}), this correspondence is surjective:
\begin{eqnarray}
\forall \phi : V \to \mathbb{K}, \exists f \in \left( \mathbb{K}[x_1, \cdots, x_m] \right) \text{ is a rep.} \Rightarrow \exists [f].
\end{eqnarray}

Next, if we suppose
\begin{eqnarray}
\Phi([f]) = \Phi([g]) : V \to \mathbb{K},
\end{eqnarray}
then both $f$ and $g$ is a representation of this map.
Therefore, $\forall p \in V$,
\begin{eqnarray}
f(p) = g(p) \Rightarrow (f-g)(p) = 0
\end{eqnarray}
i.e., the difference is in $\mathbb{I}(V)$:
\begin{eqnarray}
f-g \in \mathbb{I}(V) \Leftrightarrow [f] = [g].
\end{eqnarray}
So this correspondence is injective.\\
$\blacksquare$

\subsection{Quotients}
\label{Quotients}
The quotient of $\mathbb{K}[x_1, \cdots, x_m]$ modulo $I$,
\begin{eqnarray}
\mathbb{K}[x_1, \cdots, x_m] / I,
\end{eqnarray}
is the set of equivalence classes of congruence modulo $I$:
 \begin{eqnarray}
\mathbb{K}[x_1, \cdots, x_m] / I := \left\{ [f] \left| f \in \mathbb{K}[x_1, \cdots, x_m] \right. \right\}
\end{eqnarray}

$\mathbb{K}[x_1, \cdots, x_m] / I $ is a commutative ring.
The addition
\begin{eqnarray}
[f] + [g] := [f+g]
\end{eqnarray}
is well defined, since $\forall f' \in [f], g' \in [g]$, there exists some elements in $I$ and
\begin{eqnarray}
f' = f + i, g' = g + j, i,j \in I.
\end{eqnarray}
So
\begin{eqnarray}
f' + g' &=& f + g + (i+j)
\end{eqnarray}
and $(i+j) \in I$.
This means
\begin{eqnarray}
f' + g' \in [f+g].
\end{eqnarray}

Similarly, the multiplication
\begin{eqnarray}
[f] * [g] := [f*g]
\end{eqnarray}
is also well defined:
\begin{eqnarray}
(f' + i) * (g' + j) = f'*g' + f'*j + i*g' + i*j,
\end{eqnarray}
and $f'*j + i*g' + i*j \in I$, and
\begin{eqnarray}
f' * g' \in [f*g].
\end{eqnarray}

Identities
\begin{eqnarray}
[0], [1]
\end{eqnarray}
are also well defined, and associativity and the distributive laws are from that of $\mathbb{K}[x_1, \cdots, x_n]$.\\
$\blacksquare$

\subsection{Theorem}
The one-to-one correspondence between two commutative rings
\begin{eqnarray}
\Phi : \mathbb{K}[x_1, \cdots, x_n]/ \mathbb{I}(V) \stackrel{\cong}{\to} \mathbb{K}[V] 
\end{eqnarray}
given in \S\ref{121Correspondence}, preserves sums and products.
That is, $\Phi$ is a ring homomorphism.
One-to-one homomorphisms are called isomorphisms.

\subsubsection{Proof}
Let $[f], [g] \in \mathbb{K}[x_1, \cdots, x_n]/ \mathbb{I}(V)$, then
\begin{eqnarray}
\Phi([f]) = \exists! \phi \in \mathbb{K}[V], \Phi([g]) = \exists! \psi \in \mathbb{K}[V]
\end{eqnarray}
hold.
Therefore
\begin{eqnarray}
\Phi([f+g]) := \phi + \psi = \Phi([f]) + \Phi([g])
\end{eqnarray}
is well defined.

Similarly, we can prove the multiplication law, and that of inverse $\Phi^{-1}$.\\
$\blacksquare$

\subsection{Definition of ring isomorphisms}
Let $R,S$ be commutative rings.
\begin{enumerate}
\item A mapping $\phi: R \to S$ is a ring homomorphism iff $\phi$ preserves sums and products:
\begin{eqnarray}
\phi(r+r') = \phi(r) + \phi(r'), & \phi(r*r') = \phi(r) * \phi(r').
\end{eqnarray}
In addition,
\begin{eqnarray}
1_R \stackrel{\phi}{\mapsto} 1_S.
\end{eqnarray}

\item A mapping $\phi: R \to S$ is a ring isomorphism iff

\begin{enumerate}
\item $\phi: R \to S$ is a ring homomorphism.

\item $\phi: R \to S$ is bijective.

\end{enumerate}

\item Two rings $R,S$ are isomorphic iff there exists an isomorphism,
\begin{eqnarray}
R \stackrel{\phi}{\cong} S.
\end{eqnarray}

\end{enumerate}

\subsection{Definition of ideal}
A subset $I$ of a commutative ring $R$ is an ideal in R iff
\begin{enumerate}
\item $0_R \in I$

\item $\forall a, b \in I \Rightarrow a+b \in I$

\item $\forall a \in I, r \in R \Rightarrow r*a \in I$
\end{enumerate}

\subsection{Proposition}
Let $I$ be an ideal in $\mathbb{K}[x_1, \cdots, x_n]$.
The ideals in the quotient ring $\mathbb{K}[x_1, \cdots, x_n] / I$ are in one-to-one correspondence with the ideals of $\mathbb{K}[x_1, \cdots, x_n]$ containing $I$, i.e. the ideal $J$ s.t. $I \subset J \subset \mathbb{K}[x_1, \cdots, x_n]$.

\subsubsection{Proof}
Consider $I \subset \forall J \subset \mathbb{K}[x_1, \cdots, x_n]$.
We claim that
\begin{eqnarray}
J \mapsto J/I := \left\{ \left. [j] \in \mathbb{K}[x_1, \cdots, x_n] / I \right| j \in J \right\}
\end{eqnarray}
is the one-to-one correspondence.

We prove $J/I$ is an ideal; since $0 \in J$,
\begin{eqnarray}
[0] \in J/I
\end{eqnarray}
$\forall [j],[k] \in J/I$, the addition is guaranteed by that of $\mathbb{K}[x_1, \cdots, x_n] / I$:
\begin{eqnarray}
[j] + [k] := [j+k] \in J/I, 
\end{eqnarray}
since $\forall j,k \in J, j+ k \in J$.
Similarly,
\begin{eqnarray}
[j] * [k] := [j*k] \in J/I.
\end{eqnarray}

Finally, if $[j] \in J/I, [r] \in \mathbb{K}[x_1, \cdots, x_n] / I$, then
\begin{eqnarray}
[r] * [j] := [r*j],
\end{eqnarray}
but $\forall j \in J, r \in \mathbb{K}[x_1, \cdots, x_n], r*j \in J$ and
\begin{eqnarray}
[r] * [j] := [r*j] \in J/I.
\end{eqnarray}

Thus, $J/I$ is an ideal.

So $J \mapsto J/I$ is surjective, and clearly injective:
\begin{eqnarray}
J = K \Rightarrow J/I = K/I.
\end{eqnarray}
Therefore
\begin{eqnarray}
J \mapsto J/I
\end{eqnarray}
is one-to-one.

Let us consider the inverse, for arbitrary $\tilde{J} \in \mathbb{K}[x_1, \cdots, x_n] / I$, define
\begin{eqnarray}
J := \left\{ \left. j \in \mathbb{K}[x_1, \cdots, x_n] \right| [j] \in \tilde{J} \right\}.
\end{eqnarray}

$\forall i \in I$,
\begin{eqnarray}
[i] = [0] \in \tilde{J} \text{ i.e. } i \in J
\end{eqnarray}
so this $J$ contain $I$:
\begin{eqnarray}
I \subset J.
\end{eqnarray}

Since $I$ is an ideal,
\begin{eqnarray}
0 \in I \subset J.
\end{eqnarray}
$\forall j,k \in J$,
\begin{eqnarray}
[j], [k] \in \tilde{J}
\end{eqnarray}
by definition, and
\begin{eqnarray}
[j+k] := [j] + [k] \in \tilde{J},
\end{eqnarray}
so
\begin{eqnarray}
j+k \in J.
\end{eqnarray}

Finally, $\forall j \in J, r \in \mathbb{K}[x_1, \cdots, x_n]$,
\begin{eqnarray}
[j] \in \tilde{J}
\end{eqnarray}
and by definition in \S\ref{Quotients},
\begin{eqnarray}
[r] \in \mathbb{K}[x_1, \cdots, x_n]/I.
\end{eqnarray}
Therefore,
\begin{eqnarray}
[r*j] := [r] * [j] \in \tilde{J} \Leftrightarrow r*j \in J.
\end{eqnarray}

Thus, $J$ is an ideal, and this mapping $J \mapsfrom J$ is the inverse.\footnote{Since we have shown $J \mapsto J/I$ is bijective, if exists, the inverse is uniquely determined.}

Schematically, 
\begin{eqnarray}
\{J | I \subset J \subset \mathbb{K}[x_1, \cdots, x_n] \} && \{\tilde{J} \subset \mathbb{K}[x_1, \cdots, x_n]/ I\} \\
J &\mapsto& J/I := \left\{ \left. [j] \in \mathbb{K}[x_1, \cdots, x_n] / I \right| j \in J \right\} \qquad\qquad\\
J := \left\{ \left. j \in \mathbb{K}[x_1, \cdots, x_n] \right| [j] \in \tilde{J} \right\} &\mapsfrom& \tilde{J}
\end{eqnarray}
$\blacksquare$

\subsection{Corollary}
Every ideal in the quotient ring $\mathbb{K}[x_1, \cdots, x_n] / I$ is finitely generated.

\subsubsection{Proof}
From above, every ideal in $\mathbb{K}[x_1, \cdots, x_n] / I$ can be uniquely written as
\begin{eqnarray}
J/I
\end{eqnarray}
of some $I \subset J \subset \mathbb{K}[x_1, \cdots, x_n]$.

The Hilbert Basis Theorem in \S\ref{HilbertBasisTheorem} implies this $J$ is finitely generated:
\begin{eqnarray}
J = \left< f_1, \cdots, f_s \right>,
\end{eqnarray}
and $\forall j \in J$,
\begin{eqnarray}
\exists h_i \in \mathbb{K}[x_1, \cdots, x_n], j = \sum_i h_i * f_i.
\end{eqnarray}
Therefore, $\forall [j] \in J/I$,
\begin{eqnarray}
[j] &=& \left[ \sum_i h_i * f_i \right] \\
&=& \sum_i [h_i] * [f_i]
\end{eqnarray}
that is
\begin{eqnarray}
J/I = \left< [f_1], \cdots, [f_s] \right>.
\end{eqnarray}
So, the classes $[f_1], \cdots, [f_s]$ generate $J/I$.\\
$\blacksquare$











\end{document}